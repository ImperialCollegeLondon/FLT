\chapter{Miniproject: Adeles}\label{Adele_miniproject}

\section{Status}

This is an active miniproject.

\section{The goal}

There are several goals to this miniproject.

\begin{enumerate}
  \item Define the adeles $\A_K$ of a number field~$K$ and
    give them the structure of a $K$-algebra (status: now in mathlib thanks to
    Salvatore Mercuri);
  \item Prove that $\A_K$ is a locally compact topological ring (status:
      \href{https://github.com/smmercuri/adele-ring_locally-compact}{
      also proved by Mercuri} but not yet in mathlib);
  \item Base change: show that if $L/K$ is a finite extension of number fields then the
    natural map $L\otimes_K\A_K\to\A_L$ is an isomorphism; (status: not
    formalized yet);
  \item Prove that $K \subseteq \A_K$ is a discrete subgoup and the quotient
    is compact (status: not formalized yet, but there is a LaTeX plan);
  \item Get this stuff into mathlib (status: (1) done, (2)--(4) not done).
\end{enumerate}

We briefly go through the basic definitions. A cheap definition of the finite
adeles $\A_K^\infty$ of $K$ is $K\otimes_{\Z}\Zhat$, where $\Zhat$ is
the profinite completion of the integers. A cheap definition of the infinite adeles
$K_\infty$ of $K$ is $K\otimes_{\Q}\R$, and a cheap definition of the adeles
of $K$ is $\A_K^\infty\times K_\infty$.

However in the literature different definitions
are often given. The finite adeles of $K$ are usually defined in the books
as the so-called restricted product $\prod'_{\mathfrak{p}}K_{\mathfrak{p}}$ over the completions
$K_{\mathfrak{p}}$ of $K$ at all maximal ideals $\mathfrak{p}\subseteq\mathcal{O}_K$ of the
integers of $K$. Here the restricted product is the subset of $\prod_{\mathfrak{p}}K_{\mathfrak{p}}$
consisting of elements which are in $\mathcal{O}_{K,\mathfrak{p}}$ for all but finitely many
$\mathfrak{p}$. This is the definition given in mathlib.
Mathlib also has the proof that they're a topological ring;
furthermore the construction of the finite adeles in mathlib works for any
Dedekind domain (this was pointed out to me by Maria Ines de Frutos Fernandez; the adeles
are an arithmetic object, but the finite adeles are an algebraic object).

Similarly the infinite adeles of a number field~$K$
are usually defined as $\prod_v K_v$,
the product running over the archimedean completions of~$K$, and this is
the mathlib definition.

The adeles of a number field $K$ are the product of the finite and infinite
adeles, and mathlib knows that they're a $K$-algebra and a topological ring.

\section{Local compactness}

As mentioned above, Salvatore Mercuri has a complete formalisation of the proof
that the adele ring is locally compact. His work is in
\href{https://github.com/smmercuri/adele-ring_locally-compact}{his own repo} which
I don't want to have as a dependency of FLT, because this work should all be
in mathlib.

\begin{theorem}
  \lean{NumberField.AdeleRing.locallyCompactSpace}
  \label{NumberField.AdeleRing.locallyCompactSpace}
  \leanok
  The adeles of a number field are locally compact.
\end{theorem}
\begin{proof}
  See \href{https://github.com/smmercuri/adele-ring_locally-compact/blob/e8e34608c139ee95a1e21d9d24f138524196a2e1/AdeleRingLocallyCompact/NumberTheory/NumberField/AdeleRing.lean#L70}
  {this line} in Mercuri's repo.
\end{proof}

\section{Base change}

The ``theorem'' we want is that if $L/K$ is a finite extension of number fields,
then $\A_L=L\otimes_K\A_K$. This isn't a theorem though, this is actually a \emph{definition}
(the map between the two objects) and a theorem about
the definition (that it's an isomorphism). Before we can prove the theorem, we need to make the
definition.

Recall that the adeles $\A_K$ of a number field is a product $\A_K^\infty\times K_\infty$
of the finite adeles and the infinite adeles. So our ``theorem'' follows immediately from
the ``theorem''s that $\A_L^\infty=L\otimes_K\A_K^\infty$ and $L_\infty=L\otimes_KK_\infty$.
We may thus treat the finite and infinite results separately.

\subsection{Base change for finite adeles}

As pointed out above, the theory of finite adeles works fine for Dedekind domains.
So Let~$A$ be a Dedekind domain. Recall that the \emph{height one spectrum} of $A$ is
the nonzero prime ideals of~$A$. Note that because we stick to the literature,
rather than to common sense, fields are Dedekind domains in mathlib, and the
height one spectrum of a field is empty. The reason I don't like allowing fields
to be Dedekind domains is that geometrically the standard definition of Dedekind
domain is ``smooth affine curve, or a point''. But many theorems in algebraic geometry
begin ``let $C$ be a smooth curve'', rather than ``let $C$ be a smooth curve or a point''.

Let $K$ be the field of fractions of $A$. If $v$ is in the height one spectrum of $A$,
then we can put the $v$-adic topology on $A$ and on $K$, and consider the completions
$A_v$ and $K_v$. The finite adele ring $\mathbb{A}_{A,K}^\infty$ is defined to be
the restricted product of the $K_v$ with respect to the $A_v$, as $v$ runs over
the height one spectrum of $A$.

Now let~$L/K$ be a finite separable extension, and let $B$ be the integral closure of~$A$ in~$L$.
If $w$ is a nonzero prime ideal of $B$ lying under $v$, a prime ideal of $A$, then we can put the
$w$-adic topology on $L$ and the $v$-adic topology on~$K$.

\begin{lemma} If $i:K\to L$ denotes the inclusion then $e\times w(i(k))=v(k)$, where
  $e$ is the ramification index of $w/v$.
  \label{IsDedekindDomain.HeightOneSpectrum.valuation_comap}
  \lean{IsDedekindDomain.HeightOneSpectrum.valuation_comap}
  \leanok
\end{lemma}
\begin{proof}
  \leanok
  Standard.
\end{proof}

\begin{definition}
  \lean{IsDedekindDomain.HeightOneSpectrum.adicCompletionComapAlgHom}
  \label{IsDedekindDomain.HeightOneSpectrum.adicCompletionComapAlgHom}
  \uses{IsDedekindDomain.HeightOneSpectrum.valuation_comap}
  \leanok
  There's a natural continuous $K$-algebra homomorphism map $K_v\to L_w$. It is defined by completing
  the inclusion $K\to L$ at the finite places $v$ and $w$, which can be done
  by the previous lemma.
\end{definition}

Now say $v$ is in the height one spectrum of $A$. The map above induces a continuous
$K$-algebra homomorphism $K_v\to\prod_{w|v}L_w$, where the product runs over the height one
primes of $B$ which pull back to $v$.

\begin{theorem}
  \lean{TODO}
  \label{TODO}
  \uses{IsDedekindDomain.HeightOneSpectrum.adicCompletionComapAlgHom}
  The induced continuous $L$-algebra homomorphism $L\otimes_KK_v\to\prod_{w|v}L_w$ is an isomorphism.
\end{theorem}
\begin{proof}
  See Theorem 5.12 on p21 of \href{https://math.berkeley.edu/~ltomczak/notes/Mich2022/LF_Notes.pdf}
  {these notes}.
\end{proof}

We can take the product of the maps $K_v\to\prod_{w|v}L_w$ over $v$.

\begin{definition}
  \lean{DedekindDomain.ProdAdicCompletions.baseChange}
  \label{DedekindDomain.ProdAdicCompletions.baseChange}
  \uses{IsDedekindDomain.HeightOneSpectrum.adicCompletion_comap_algHom}
  There's a natural $K$-algebra homomorphism $\prod_v K_v\to\prod_w L_w$, where the
  products run over the height one spectra of $A$ and $B$ respectively.
\end{definition}

Note that we make no claim about continuity.

\begin{theorem} This map induces a natural continuous $K$-algebra homomorphism $\A_{A,K}^\infty\to\A_{B,L}^\infty$.
  \label{DedekindDomain.FiniteAdeleRing.baseChange}
  \lean{DedekindDomain.FiniteAdeleRing.baseChange}
  \leanok
\end{theorem}
\begin{proof}
  Note that the restricted product does not have the subspace topology, so one has to think
  a little here.
\end{proof}

\begin{theorem} The induced map $L\otimes_K\A_{A,K}^\infty\to\A_{B,L}^\infty$ is a topological
  isomorphism.
  \label{DedekindDomain.FiniteAdeleRing.baseChangeIso}
  \lean{DedekindDomain.FiniteAdeleRing.baseChangeIso}
  \leanok
\end{theorem}
\begin{proof}
  Xavier notes **TODO**
\end{proof}

\section{Discreteness and compactness}

We need that if $K$ is a number field then
$K\subseteq\mathbb{A}_K$ is discrete, and the quotient (with the
quotient topology) is compact. Here is a proposed proof.

\begin{theorem}
  \lean{Rat.AdeleRing.zero_discrete}
  \label{Rat.AdeleRing.zero_discrete}
  \leanok
  There's an open subset of $\A_{\Q}$ whose intersection with $\Q$ is $\{0\}$.
\end{theorem}
\begin{proof}
  Use $\prod_p{\Z_p}\otimes(-1,1)$. Any rational $q$ in this set is a $p$-adic
  integer for all primes $p$ and hence (writing it in lowest terms as $q=n/d$)
  satisfies $p\nmid d$, meaning that $d=\pm1$ and thus $q\in\Z$. The fact
  that $q\in(-1,1)$ implies $q=0$.
\end{proof}

\begin{theorem}
  \lean{Rat.AdeleRing.discrete}
  \label{Rat.AdeleRing.discrete}
  \leanok
  The rationals $\Q$ are a discrete subgroup of $\A_{\Q}$.
\end{theorem}
\begin{proof}
  If $q\in\Q$ and $U$ is the open subset in the previous lemma, then
  it's easily checked that $\Q\cap U=\{0\}$ implies $\Q\cap (U+q)=\{q\}$,
  and $U+q$ is open.
\end{proof}

\begin{theorem}
  \lean{NumberField.AdeleRing.discrete}
  \label{NumberField.AdeleRing.discrete}
  \leanok
  The additive subgroup $K$ of $\A_K$ is discrete.
\end{theorem}
\begin{proof}
  By a previous result, we have $\A_K=K\otimes_{\Q}\A_{\Q}$.
  Choose a basis of $K/\Q$; then $K$ can be identified with $\Q^n\subseteq(\A_{\Q})^n$
  and the result follows from the previous theorem.
\end{proof}

For compactness we follow the same approach.

\begin{theorem}
  \lean{Rat.AdeleRing.cocompact}
  \label{Rat.AdeleRing.cocompact}
  \leanok
  The quotient $\A_{\Q}/\Q$ is compact.
\end{theorem}
\begin{proof}
  The space $\prod_p\Z_p\times[0,1]\subseteq\A_{\Q}$ is a product of compact spaces
  and is hence compact. I claim that it surjects onto $\A_{\Q}/\Q$. Indeed,
  if $a\in\A_{\Q}$ then for the finitely many prime numbers $p\in S$ such that $a_p\not\in\Z_p$
  we have $a_p\in\frac{r_p}{p^{n_p}}+\Z_p$ with $r_p/p^{n_p}\in\Q$, and
  if $q=\sum_{p\in S}\frac{r_p}{p^{n_p}}\in\Q$ then $a-q\in \prod_p\Z_p\times\R$.
  Now just subtract $\lfloor a_{\infty}-q\rfloor$ to move into $\prod_p\Z_p\times[0,1)$
  and we are done.
\end{proof}

\begin{theorem}
  \lean{NumberField.AdeleRing.cocompact}
  \label{NumberField.AdeleRing.cocompact}
  \leanok
  The quotient $\A_K/K$ is compact.
\end{theorem}
\begin{proof}
  We proceed as in the discreteness proof above, by reducing to $\Q$. As before, choosing
  a $\Q$-basis of $K$ gives us $\A_K/K\cong(\A_{\Q}/\Q)^n$ so the result follows from
  the previous theorem.
\end{proof}
