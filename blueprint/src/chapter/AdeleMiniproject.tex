\chapter{Miniproject: Adeles}\label{Adele_miniproject}

\section{Status}

This is an active miniproject.

\section{The goal}

There are several goals to this miniproject.

\begin{enumerate}
  \item Define the adeles $\A_K$ of a number field~$K$ and
    give them the structure of a $K$-algebra (status: now in mathlib thanks to
    Salvatore Mercuri);
  \item Prove that $\A_K$ is a locally compact topological ring (status:
      \href{https://github.com/smmercuri/adele-ring_locally-compact}{
      also proved by Mercuri} but not yet in mathlib);
  \item Base change: show that if $L/K$ is a finite extension of number fields then the
    natural map $L\otimes_K\A_K\to\A_L$ is an isomorphism, both algebraic and
    topological; (status: not
    formalized yet, but there is a plan -- see the project dashboard);
  \item Prove that $K \subseteq \A_K$ is a discrete subgroup and the quotient
    is compact (status: not formalized yet, but there is a plan -- see the project
    dashboard);
  \item Get this stuff into mathlib (status: (1) done, (2)--(4) not done).
\end{enumerate}

We briefly go through the basic definitions. Let $K$ be a number field.
Let $\Zhat=\projlim_{N\geq1}(\Z/N\Z)$ be the profinite completion of $\Z$,
equipped with the projective limit topology.

A cheap definition of the finite adeles $\A_K^\infty$ of $K$ is $K\otimes_{\Z}\Zhat$,
equipped with the $\Zhat$-module topology.
A cheap definition of the infinite adeles
$K_\infty$ of $K$ is $K\otimes_{\Q}\R$ with the $\R$-module topology (this is a
finite-dimensional $\R$-vector space so this is just the usual topology on $\R^n$).
A cheap definition of the adeles of $K$ is $\A_K^\infty\times K_\infty$ with
the product topology. This is a commutative topological ring.

However in the literature (and in mathlib) we see different definitions.
The finite adeles of $K$ are usually defined in the books
as the so-called restricted product $\prod'_{\mathfrak{p}}K_{\mathfrak{p}}$ over the completions
$K_{\mathfrak{p}}$ of $K$ at all maximal ideals $\mathfrak{p}\subseteq\mathcal{O}_K$ of the
integers of $K$. Here the restricted product is the subset of $\prod_{\mathfrak{p}}K_{\mathfrak{p}}$
consisting of elements which are in the integers $\mathcal{O}_{K,\mathfrak{p}}$ of
$K_{\mathfrak{p}}$ for all but finitely many $\mathfrak{p}$. This is the definition given in
mathlib. Mathlib also has the proof that they're a topological ring;
furthermore the construction of the finite adeles in mathlib works for any
Dedekind domain (this was pointed out to me by Mar\'ia In\'es
de Frutos Fern\'andez; the adeles
are an arithmetic object, but the finite adeles are an algebraic object).

Similarly the infinite adeles of a number field~$K$
are usually defined as $\prod_v K_v$,
the product running over the archimedean completions of~$K$, and this is
the mathlib definition.

The adeles of a number field $K$ are the product of the finite and infinite
adeles, and mathlib knows that they're a $K$-algebra and a topological ring.

\section{Local compactness}

As mentioned above, Salvatore Mercuri was the first to give a complete formalisation of the proof
that the adele ring is locally compact as a topological space. His work is in
\href{https://github.com/smmercuri/adele-ring_locally-compact}{his own repo} which
I don't want to have as a dependency of FLT, because this work should all be
in mathlib.

\begin{theorem}
  \lean{NumberField.AdeleRing.locallyCompactSpace}
  \label{NumberField.AdeleRing.locallyCompactSpace}
  \discussion{253}
  \leanok
  The adeles of a number field are locally compact.
\end{theorem}
\begin{proof}
  The adeles of a number field are a product of the finite adeles and the infinite adeles
  so it suffices to prove that the finite and infinite adeles are locally compact.
  The infinite adeles are just isomorphic to $\R^n$ as a topological space, so they're certainly
  locally compact. As for the finite adeles,
  the mathlib theorem {\tt RestrictedProduct.locallyCompactSpace\_of\_addGroup}
  says that a restricted product of locally compact additive groups with respect to open compact
  subgroups is locally compact, so this reduces us to the claim that if $K$ is a number field
  and $v$ is a finite place then the completion $K_v$ is locally compact and its integer
  ring $\mathcal{O}_v$ is open and compact. All of these facts are either in mathlib or
  proved by Mercuri in his repo and are on the way to mathlib.
\end{proof}

\section{Base change}

The ``theorem'' we want is that if $L/K$ is a finite extension of number fields,
then $\A_L=L\otimes_K\A_K$. This isn't a theorem though, this is actually a \emph{definition}
(the map between the two objects) and a theorem about
the definition (that it's an isomorphism). In fact the full claim is that it is both a homeomorphism
and an $L$-algebra isomorphism. Before we can prove the theorem, we need to make the
definition.

Recall that the adeles $\A_K$ of a number field is a product $\A_K^\infty\times K_\infty$
of the finite adeles and the infinite adeles. So our ``theorem'' follows immediately from
the ``theorem''s that $\A_L^\infty=L\otimes_K\A_K^\infty$ and $L_\infty=L\otimes_KK_\infty$
(both of these equalities mean an algebraic and topological isomorphism).
We may thus treat the finite and infinite results separately.

\subsection{Base change for finite adeles}

As pointed out above, the theory of finite adeles works fine for Dedekind domains.
So for the time being let~$A$ be a Dedekind domain. Recall that the \emph{height one spectrum}
of $A$ is the nonzero prime ideals of~$A$. Note that because we stick to the literature,
rather than to common sense, fields are Dedekind domains in mathlib, and the
height one spectrum of a field is empty. The reason I don't like allowing fields
to be Dedekind domains is that geometrically the definition of Dedekind
domain in the literature is ``smooth affine curve, or a point''. But many theorems in algebraic
geometry begin ``let $C$ be a smooth curve'', rather than ``let $C$ be a smooth curve or a point''.

Let $K$ be the field of fractions of $A$. If $v$ is in the height one spectrum of $A$,
then we can put the $v$-adic topology on $A$ and on $K$, and consider the completions
$A_v$ and $K_v$. The finite adele ring $\mathbb{A}_{A,K}^\infty$ is defined to be
the restricted product of the $K_v$ with respect to the $A_v$, as $v$ runs over
the height one spectrum of $A$. It is topologised by making $\prod_v A_v$ open with
the product topology (here $A_v$ has the $v$-adic topology).

Now let~$L/K$ be a finite separable extension, and let $B$ be the integral closure of~$A$ in~$L$.
We want to relate the finite adeles of $K$ and of $L$. We work place by place, starting by fixing
one place $w$ of $B$ and analysing the relation of $L_w$ and $B_w$ to the completions $K_v$
and $A_v$ where $v$ is the place of $A$ dividing $w$.

So let $w$ be a nonzero prime ideal of $B$. Say $w$ lies over $v$, a prime ideal of $A$.
Then we can put the $w$-adic topology on $L$ and the $v$-adic topology on~$K$. Furthermore
we can equip $K$ with an additive $v$-adic valuation, that is,
a function also called $v$ from $K$ to $\Z\cup\{\infty\}$ normalised so that if $\pi$ is a uniformiser
for $v$ then $v(\pi)=1$. Similarly we consider $w$ as a function from $L$ to $\Z\cup\{\infty\}$.
The next lemma explains how these valuations are related.

\begin{lemma}
  \label{IsDedekindDomain.HeightOneSpectrum.valuation_comap}
  \lean{IsDedekindDomain.HeightOneSpectrum.valuation_comap}
  \leanok
  If $i:K\to L$ denotes the inclusion then for $k\in K$ we have
  $e\times w(i(k))=v(k)$, where $e$ is the ramification index of $w/v$
  (recall that valuations here are written additively, unlike in mathlib).
\end{lemma}
\begin{proof}
  \leanok
  Standard (and formalized).
\end{proof}

\begin{definition}
  \lean{IsDedekindDomain.HeightOneSpectrum.adicCompletionComapSemialgHom}
  \label{IsDedekindDomain.HeightOneSpectrum.adicCompletionComapSemialgHom}
  \uses{IsDedekindDomain.HeightOneSpectrum.valuation_comap}
  \leanok
  There's a natural ring map $K_v\to L_w$ extending the map $K\to L$.
  It is defined by completing
  the inclusion $K\to L$ at the finite places $v$ and $w$ (which can be done
  because the previous lemma shows that the map is uniformly continuous for the $v$-adic
  and $w$-adic topologies).
\end{definition}

\begin{lemma}
  \lean{IsDedekindDomain.HeightOneSpectrum.valued_adicCompletionComap}
  \label{IsDedekindDomain.HeightOneSpectrum.valued_adicCompletionComap}
  \leanok
  If $i_v:K_v\to L_w$ denotes the map of the previous definition
  then for $x\in K_v$ we have
  $e\times w(i(k))=v(k)$, where $e$ is the ramification index of $w/v$.
\end{lemma}
\begin{proof}
  \leanok
  Follows by continuity from lemma~\ref{IsDedekindDomain.HeightOneSpectrum.valuation_comap}.
\end{proof}

\begin{theorem}
  \lean{IsDedekindDomain.HeightOneSpectrum.adicCompletionComapSemialgHom.mapadicCompletionIntegers}
  \label{IsDedekindDomain.HeightOneSpectrum.adicCompletionComapSemialgHom.mapadicCompletionIntegers}
  \leanok
  The map $i_v:K_v\to L_w$ sends the integer ring $A_v$ into $B_w$.
\end{theorem}
\begin{proof}
  \leanok
  The integer ring is defined by $v\geq0$ (or $v\leq 1$ in mathlib, which uses multiplicative
  valuations) so the result follows from \ref{IsDedekindDomain.HeightOneSpectrum.valued_adicCompletionComap}.
\end{proof}
\begin{theorem}
  \lean{IsDedekindDomain.HeightOneSpectrum.adicCompletionComap_isModuleTopology}
  \label{IsDedekindDomain.HeightOneSpectrum.adicCompletionComap_isModuleTopology}
  \uses{IsDedekindDomain.HeightOneSpectrum.adicCompletionComapSemialgHom}
  \leanok
  \discussion{326}
  Giving $L_w$ the $K_v$-algebra structure coming from the natural map $K_v\to L_w$,
  the $w$-adic topology on $L_w$ is the $K_v$-module topology.
\end{theorem}
\begin{proof}
  Any basis for $L$ as a $K$-vector space spans $L_w$ as a $K_v$-module, so $L_w$ is
  finite-dimensional over $K_v$ and the module topology is the same as the product
  topology. So we need to establish that the product topology on $L_w=K_v^n$ is
  the $w$-adic topology. But the $w$-adic topology is induced by the $w$-adic norm,
  which makes $L_w$ into a normed $K_v$-vector space, and (after picking a basis)
  the product norm on $L_w=K_v^n$ also makes $L_w$ into a normed $K_v$-vector space.
  So the result follows from the standard fact (see for example the lemma on p52
  of Cassels-Froelich, formalized as {\tt ContinuousLinearEquiv.ofFinrankEq} in mathlib)
  that any two norms on a finite-dimensional vector space over
  a complete field are equivalent (and thus induce the same topology).
\end{proof}

Because of the commutative diagram
\[
\begin{tikzcd}
K_v \arrow{r} & L_w  \\
K \arrow{u} \arrow{r} & L \arrow{u}
\end{tikzcd}
\]

we can view $L_w$ as an $L\otimes_KK_v$-algebra.

Now instead of fixing $w$ upstairs, we fix $v$ downstairs and consider all $w$ lying
over it at once. So say $v$ is in the height one spectrum of $A$.

\begin{lemma}
  \lean{IsDedekindDomain.HeightOneSpectrum.Extension.finite}
  \label{IsDedekindDomain.HeightOneSpectrum.Extension.finite}
  \leanok
  There are only finitely many primes $w$ of $B$ lying above $v$.
\end{lemma}
\begin{proof}
  \leanok
  This is a standard fact about Dedekind domains. The key input is
  mathlib's theorem {\tt primesOver\_finite}.
\end{proof}

We write $w|v$ to denote the fact that $w$ is a prime of $B$ above $v$ of $A$.

\begin{definition}
  \lean{IsDedekindDomain.HeightOneSpectrum.adicCompletionComapSemialgHom'}
  \label{IsDedekindDomain.HeightOneSpectrum.adicCompletionComapSemialgHom'}
  \uses{IsDedekindDomain.HeightOneSpectrum.adicCompletionComapSemialgHom}
  \leanok
  The product of the maps $K_v\to L_w$ for $w|v$ is a natural ring map $K_v\to\prod_{w|v}L_w$
  lying over $K\to L$.
\end{definition}

Because $K_v\to\prod_{w|v}L_w$ lies over $K\to L$, there's an induced $L$-algebra
map $L\otimes_KK_v\to\prod_{w|v}L_w$. We are now able to state one of the key results
in this section. The proof is probably the hardest proof
in this section to formalize.

\begin{theorem}
  \lean{IsDedekindDomain.HeightOneSpectrum.adicCompletionComapAlgEquiv}
  \label{IsDedekindDomain.HeightOneSpectrum.adicCompletionComapAlgEquiv}
  \uses{IsDedekindDomain.HeightOneSpectrum.adicCompletionComapSemialgHom',
  IsDedekindDomain.HeightOneSpectrum.Extension.finite}
  \leanok
  The induced $L$-algebra homomorphism $L\otimes_KK_v\to\prod_{w|v}L_w$ is an
  isomorphism of rings.
\end{theorem}
\begin{proof}
  My proposal (and I'm open to other suggestions) is to follow Theorem 5.12 on p21
  of \href{https://math.berkeley.edu/~ltomczak/notes/Mich2022/LF_Notes.pdf}
  {these notes}. We may write $L=K(\alpha)$ as a finite separable extension is simple. Let $f(x)$
  be the minimum polynomial of $\alpha$. This is irreducible over $K$ but may factor over
  the bigger field $K_v$. Factor $f(x)=f_1(x)f_2(x)\cdots f_r(x)$ into
  monic irreducibles $K_v[x]$; these are distinct by separability. We have $L=K[x]/(f)$
  so $L\otimes_KK_v=K_v[x]/(f)=K_v[x]/(\prod_i f_i)=\prod_i K_v[x]/(f_i)$. Write
  $L_i=K_v[x]/(f_i)$. Informally we have shown that $L\otimes_KK_v=\prod_i L_i$. More formally
  we have constructed some $L\otimes_KK_v$-algebras $L_i$ and given an $L\otimes_KK_v$-algebra
  isomorphism $L\otimes_KK_v\equiv \prod_i L_i$, so what we need to do
  is to identify the fields $L_i$ with the fields $L_w$ as $L\otimes_KK_v$-algebras.
  In other words, we need
  to set up a bijection between the $i$'s indexing the $f_i$ and the $w$'s indexing the
  places dividing $v$, and if $i$ corresponds to $w$ then we need to establish an
  $L\otimes_K K_v$-algebra isomorphism between $L_i$ and $L_w$.

  First note that $[L_i:K_v]\leq [L:K]<\infty$ and so there's a unique extension of the $v$-adic
  norm on $K_v$ to $L_i$. The restriction of this norm to $L$ must be equivalent to the $w$-adic
  norm for some $w|v$ by Ostrowski (one does not need the full force of Ostrowski here,
  one just needs that a nonarchimedean norm on $L$ whose restriction to $K$ is equivalent to $v$
  must come from a unique place of $L$ above $v$). This gives us a function from the $i$'s
  to the $w$'s and the $L\otimes_K K_v$-algebra isomorphism $L_i=L_w$, so all that is left is
  to prove that this function from $i$'s to $w$'s is a bijection, which we do by showing that
  it is an injection and a surjection.

  We first deal with injectivity. Note that if $i\not=j$ then $L_i$ and $L_j$ cannot be isomorphic
  as $L\otimes_KK_v$-algebras, because (thinking of $L_i$ as $K_v[x]/(f_i)$) such an isomorphism
  would send $x$ to $x$ and thus show $f_i=f_j$, contradicting separability of $\alpha$. Hence the map
  from $L_i$ to the $w$ dividing $v$ is injective.

  For surjectivity, note that if $w|v$ then $L_w$ is an $L\otimes_KK_v$-algebra and hence
  admits a map from $L\otimes_K K_v$ which must factor through one of the $L_i$.
  This gives an injection $L_i\to L_w$. But $L_i$ is complete, as it's a finite extension
  of $K_v$, and the image is dense because $L$ is dense in $L_w$, hence the injection
  is an isomorphism. This shows that $i$ maps to $w$.
\end{proof}

\begin{theorem}
  \label{IsDedekindDomain.HeightOneSpectrum.prodAdicCompletionComap_isModuleTopology}
  \lean{IsDedekindDomain.HeightOneSpectrum.prodAdicCompletionComap_isModuleTopology}
  \uses{IsDedekindDomain.HeightOneSpectrum.adicCompletionComap_isModuleTopology}
  \leanok
  For $v$ fixed, the product topology on $\prod_{w|v}L_w$ is the $K_v$-module
  topology.
\end{theorem}
\begin{proof}
  \leanok
  This is a finite product of $K_v$-modules each of which has the $K_v$-module topology
  by~\ref{IsDedekindDomain.HeightOneSpectrum.adicCompletionComap_isModuleTopology},
  and the product topology is the module topology for a finite product of modules each of which
  has the module topology (this is in mathlib).
\end{proof}

\begin{theorem}
  \label{IsDedekindDomain.HeightOneSpectrum.adicCompletionComapContinuousAlgEquiv}
  \lean{IsDedekindDomain.HeightOneSpectrum.adicCompletionComapContinuousAlgEquiv}
  \uses{IsDedekindDomain.HeightOneSpectrum.prodAdicCompletionComap_isModuleTopology}
  If we give $L\otimes_KK_v$ the $K_v$-module topology then the $L$-algebra isomorphism
  $L\otimes_K K_v\cong\prod_{w|v}L_w$ is also a homeomorphism.
\end{theorem}
\begin{proof} Indeed, is a $K_v$-algebra isomorphism between two modules each of which
  have the module topology, and any module map is automorphically continuous for the
  module topologies.
\end{proof}

We now start thinking about what's going on at the integral level. We write $A_v$
for the integers of $K_v$ and $B_w$ for the integers of $L_w$.

\begin{theorem}
  \lean{IsDedekindDomain.HeightOneSpectrum.adicCompletionComapAlgEquiv_integral}
  \label{IsDedekindDomain.HeightOneSpectrum.adicCompletionComapAlgEquiv_integral}
  \uses{IsDedekindDomain.HeightOneSpectrum.adicCompletionComapAlgEquiv}
  The isomorphism $L\otimes_KK_v\to\prod_{w|v}L_w$ induces an isomorphism
  $B\otimes_AA_v\to \prod_{w|v}B_w$
  for all $v$ in the height one spectrum of $A$.
\end{theorem}
\begin{proof}
  Certainly the image of the integral elements are integral. The argument in the other
  direction is more delicate. My original plan was to follow Cassels--Froehlich,
  Cassels' article ``Global fields'', section 12 lemma, p61, which proves it for
  all but finitely many primes, but \href{https://github.com/ImperialCollegeLondon/FLT/pull/400}
  {a PR by Matthew Jasper} gives another approach which works for all primes.
  Jasper's argument is to show that the closure of $A$ in $K_v$ is $A_v$
  for a valuation on a Dedekind domain, and then that the closure of $A$ in $\prod_{v\in S}K_v$
  is $\prod_{v\in S}A_v$ for $S$ a finite set of valuations (using the Chinese
  remainder theorem). Applying this to $B$ we get that the closure of $B$ in $\prod_{w|v}L_w$
  is $\prod_{w|v}B_w$. He then shows that this closure is the image of
  $B\otimes_A\mathcal{O}_v$ (by showing that this image is closed becasue it's open),
  giving surjectivity; injectivity follows from the statement
  that $L\otimes_KK_v=\prod_{w|v}L_w$.
\end{proof}

A summary of what we have so far: for all finite places $v$ of $A$
we have shown that the natural map $L\otimes_KK_v\to\prod_wL_w$
is an isomorphism of $L$-algebras, and that if $L\otimes_KK_v$ has
the $K_v$-module topology and each $L_w$ has the valuation topology
then this map is also a homeomorphism. Furthermore we have shown
that there is an induced algebraic isomorphism $B\otimes_AA_v\equiv\prod_w B_w$
on the subrings of the left and right hand sides.

Recall that the finite adeles $\A_{A,K}^\infty$ is defined in mathlib to be
the restricted product of the $K_v$ with respect to the $A_v$, equipped with a certain
restricted product topology (which is not the subspace topology of the product
topology, indeed $\prod_v A_v$ is open in this topology). We

\begin{theorem} There's a natural homomorphism
  $\A_{A,K}^\infty\to\A_{B,L}^\infty$ lying over $K\to L$.
  \label{IsDedekindDomain.FiniteAdeleRing.mapSemialgHom}
  \lean{IsDedekindDomain.FiniteAdeleRing.mapSemialgHom}
  \leanok
\end{theorem}
\begin{proof}
  This follows immediately from the fact (ref?) that integesr map to integers
  and general nonsense.
\end{proof}

I would like to conclude the following also by general nonsense:

\begin{theorem}
  \label{IsDedekindDomain.FiniteAdeleRing.baseChangeAlgEquiv}
  \lean{IsDedekindDomain.FiniteAdeleRing.baseChangeAlgEquiv}
  \uses{IsDedekindDomain.HeightOneSpectrum.adicCompletionComapAlgEquiv_integral}
  \leanok
  If we give $L\otimes_K\A_{A,K}^\infty$ the ``module topology'', coming from the fact
  that $L\otimes_K\A_{A,K}^\infty$ is an $\A_{A,K}^\infty$-module, then the induced
  $L$-algebra morphism
  $L\otimes_K\A_{A,K}^\infty\to\A_{B,L}^\infty$ is a topological isomorphism.
\end{theorem}
\begin{proof}
  {\bf TODO} rewrite this.

  Existence of the map follows from
  theorem~\ref{IsDedekindDomain.HeightOneSpectrum.adicCompletionComapAlgEquiv_integral}.
  Surjectivity follows from
  theorem~\ref{IsDedekindDomain.HeightOneSpectrum.adicCompletionComapAlgEquiv_integral}.
  The fact that it is a topological isomorphism surely follows from the fact that
  $L\otimes_K K_v=\oplus_{w|v}L_w$ and that this identification identifies
  the subgroup $\calO_L\otimes_K K_v$ with $\oplus_{w|v}\calO_{L_w}$ for all
  but finitely many $v$. Although I'm wondering whether it's easier to prove
  that the map $\A_{A,K}^\infty\to\A_{B,L}^\infty$ makes $\A_{B,L}^\infty$
  into an $\A_{A,K}^\infty$-module and the claim (also undoubtedly true, although
  I am not sure of the best level of abstraction here) that the topology on
  $\A_{B,L}^\infty$ is the $\A_{A,K}^\infty$-module topology.
\end{proof}

\subsection{Base change for infinite adeles}

Recall that if $K$ is a number field then the infinite adeles of $K$ are defined
to be the product $\prod_{v\mid\infty} K_v$ of all the completions of $K$ at the
infinite places.

The result we need here is that if $L/K$ is a finite extension of number fields,
then the map $K\to L$ extends to a continuous $K$-algebra map $K_\infty\to L_\infty$,
and thus to a continuous $L$-algebra isomorphism $L\otimes_KK_\infty\to L_\infty$.
Perhaps a cheap proof would be to deduce it from the fact that $K_\infty=K\otimes_{\Q}\R$.

\begin{theorem}
  \label{NumberField.InfiniteAdeleRing.baseChangeEquiv}
  \lean{NumberField.InfiniteAdeleRing.baseChangeEquiv}
  If $K\to L$ is a ring homomorphism between two number fields then there is a natural isomorphism
  (both topological and algebraic) $L\otimes_KK_\infty\cong L_\infty$.
\end{theorem}
\begin{proof}
  Standard.
\end{proof}

\subsection{Base change for adeles}

From the previous results we deduce immediately that if $L/K$ is a finite extension
of number fields then there's a natural (topological and algebraic) isomorphism
$L\otimes_K\A_K\to \A_L$.

\begin{theorem}
  \label{NumberField.AdeleRing.baseChangeEquiv}
  \lean{NumberField.AdeleRing.baseChangeEquiv}
  If $K\to L$ is a ring homomorphism between two number fields then there is a natural isomorphism
  (both topological and algebraic) $L\otimes_K\A_K\cong\A_L$.
\end{theorem}
\begin{proof}
  Follows from the previous results.
\end{proof}

Something else we shall need:

\begin{theorem}
  \label{NumberField.AdeleRing.baseChange_moduleTopology}
  If $K\to L$ is a ring homomorphism between two number fields then the topology on $\A_L$
  is the $\A_K$-module topology, where the module structure comes from the
  natural map $\A_K\to\A_L$.
\end{theorem}
\begin{proof}
  Because $\A_L=\A_K\otimes_{K}L$ we know that $\A_L$ is a finite free $\A_K$-module,
  and a standard fact (being PRed to mathlib) about the module topology is that the module topology
  on a finite free module is just the product topology. It thus suffices to show that
  the topology on $\A_L$ is the product topology after picking an $\A_K$-basis for $\A_L$
  but this is standard.
\end{proof}

\section{Discreteness and compactness}

We need that if $K$ is a number field then
$K\subseteq\mathbb{A}_K$ is discrete, and the quotient (with the
quotient topology) is compact. Here is a proposed proof.

\begin{theorem}
  \lean{Rat.AdeleRing.zero_discrete}
  \label{Rat.AdeleRing.zero_discrete}
  \leanok
  There's an open subset of $\A_{\Q}$ whose intersection with $\Q$ is $\{0\}$.
\end{theorem}
\begin{proof}
  Use $\prod_p{\Z_p}\times(-1,1)$. Any rational $q$ in this set is a $p$-adic
  integer for all primes $p$ and hence (writing it in lowest terms as $q=n/d$)
  satisfies $p\nmid d$, meaning that $d=\pm1$ and thus $q\in\Z$. The fact
  that $q\in(-1,1)$ implies $q=0$.
\end{proof}

\begin{theorem}
  \lean{NumberField.AdeleRing.zero_discrete}
  \label{NumberField.AdeleRing.zero_discrete}
  \leanok
  There's an open subset of $\A_{K}$ whose intersection with $K$ is $\{0\}$.
\end{theorem}
\begin{proof}
  By a previous result, we have $\A_K=K\otimes_{\Q}\A_{\Q}$.
  Choose a basis of $K/\Q$; then $K$ can be identified with $\Q^n\subseteq(\A_{\Q})^n$
  and the result follows from the previous theorem.
\end{proof}

\begin{theorem}
  \lean{NumberField.AdeleRing.discrete}
  \label{NumberField.AdeleRing.discrete}
  \leanok
  The additive subgroup $K$ of $\A_K$ is discrete.
\end{theorem}
\begin{proof}
  If $x\in K$ and $U$ is the open subset in the previous lemma, then
  it's easily checked that $K\cap U=\{0\}$ implies $K\cap (U+x)=\{x\}$,
  and $U+x$ is open.
\end{proof}

For compactness we follow the same approach.

\begin{theorem}
  \lean{Rat.AdeleRing.cocompact}
  \label{Rat.AdeleRing.cocompact}
  \leanok
  The quotient $\A_{\Q}/\Q$ is compact.
\end{theorem}
\begin{proof}
  The space $\prod_p\Z_p\times[0,1]\subseteq\A_{\Q}$ is a product of compact spaces
  and is hence compact. I claim that it surjects onto $\A_{\Q}/\Q$. Indeed,
  if $a\in\A_{\Q}$ then for the finitely many prime numbers $p\in S$ such that $a_p\not\in\Z_p$
  we have $a_p\in\frac{r_p}{p^{n_p}}+\Z_p$ with $r_p/p^{n_p}\in\Q$, and
  if $q=\sum_{p\in S}\frac{r_p}{p^{n_p}}\in\Q$ then $a-q\in \prod_p\Z_p\times\R$.
  Now just subtract $\lfloor a_{\infty}-q\rfloor$ to move into $\prod_p\Z_p\times[0,1)$
  and we are done.
\end{proof}

\begin{theorem}
  \lean{NumberField.AdeleRing.cocompact}
  \label{NumberField.AdeleRing.cocompact}
  \leanok
  The quotient $\A_K/K$ is compact.
\end{theorem}
\begin{proof}
  We proceed as in the discreteness proof above, by reducing to $\Q$. As before, choosing
  a $\Q$-basis of $K$ gives us $\A_K/K\cong(\A_{\Q}/\Q)^n$ so the result follows from
  the previous theorem.
\end{proof}
