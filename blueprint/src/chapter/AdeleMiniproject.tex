\chapter{Miniproject: Adeles}\label{Adele_miniproject}

\section{Status}

This is an active miniproject.

\section{The goal}

There are several goals to this miniproject.

\begin{enumerate}
  \item Define the adeles $\A_K$ of a number field~$K$ and
    give them the structure of a $K$-algebra (status: now in mathlib thanks to
    Salvatore Mercuri);
  \item Prove that $\A_K$ is a locally compact topological ring (status:
      \href{https://github.com/smmercuri/adele-ring_locally-compact}{
      also proved by Mercuri} but not yet in mathlib);
  \item Base change: show that if $L/K$ is a finite extension of number fields then the
    natural map $L\otimes_K\A_K\to\A_L$ is an isomorphism, both algebraic and
    topological; (status: not
    formalized yet, but there is a plan -- see the project dashboard);
  \item Prove that $K \subseteq \A_K$ is a discrete subgroup and the quotient
    is compact (status: not formalized yet, but there is a plan -- see the project
    dashboard);
  \item Get this stuff into mathlib (status: (1) done, (2)--(4) not done).
\end{enumerate}

We briefly go through the basic definitions. Let $K$ be a number field.
Let $\Zhat=\projlim_{N\geq1}(\Z/N\Z)$ be the profinite completion of $\Z$,
equipped with the projective limit topology.

A cheap definition of the finite adeles $\A_K^\infty$ of $K$ is $K\otimes_{\Z}\Zhat$,
equipped with the $\Zhat$-module topology.
A cheap definition of the infinite adeles
$K_\infty$ of $K$ is $K\otimes_{\Q}\R$ with the $\R$-module topology (this is a
finite-dimensional $\R$-vector space so this is just the usual topology on $\R^n$).
A cheap definition of the adeles of $K$ is $\A_K^\infty\times K_\infty$ with
the product topology. This is a commutative topological ring.

However in the literature (and in mathlib) we see different definitions.
The finite adeles of $K$ are usually defined in the books
as the so-called restricted product $\prod'_{\mathfrak{p}}K_{\mathfrak{p}}$ over the completions
$K_{\mathfrak{p}}$ of $K$ at all maximal ideals $\mathfrak{p}\subseteq\mathcal{O}_K$ of the
integers of $K$. Here the restricted product is the subset of $\prod_{\mathfrak{p}}K_{\mathfrak{p}}$
consisting of elements which are in the integers $\mathcal{O}_{K,\mathfrak{p}}$ of
$K_{\mathfrak{p}}$ for all but finitely many $\mathfrak{p}$. This is the definition given in
mathlib. Mathlib also has the proof that they're a topological ring;
furthermore the construction of the finite adeles in mathlib works for any
Dedekind domain (this was pointed out to me by Mar\'ia In\'es
de Frutos Fern\'andez; the adeles
are an arithmetic object, but the finite adeles are an algebraic object).

Similarly the infinite adeles of a number field~$K$
are usually defined as $\prod_v K_v$,
the product running over the archimedean completions of~$K$, and this is
the mathlib definition.

The adeles of a number field $K$ are the product of the finite and infinite
adeles, and mathlib knows that they're a $K$-algebra and a topological ring.

\section{Local compactness}

As mentioned above, Salvatore Mercuri was the first to give a complete formalisation of the proof
that the adele ring is locally compact as a topological space. His work is in
\href{https://github.com/smmercuri/adele-ring_locally-compact}{his own repo} and proved the
result using the ``ad hoc'' topology on the adeles which we initially had. Since then,
adeles have been refactored to have the direct limit topology and mathlib has
\href{https://leanprover-community.github.io/mathlib4_docs/Mathlib/Topology/Algebra/RestrictedProduct.html\#RestrictedProduct.locallyCompactSpace_of_addGroup}
{\tt RestrictedProduct.locallyCompactSpace\_of\_addGroup}, the result
that a restricted product of topological additive groups $K_v$ over compact open
subgroups $A_v$ is locally compact.

What we need then is this (note that this is not true for a general Dedekind domain):
\begin{theorem}
  \label{NumberField.instCompactSpaceAdicCompletionIntegers}
  \lean{NumberField.instCompactSpaceAdicCompletionIntegers}
  \discussion{451}
  \leanok
  If $K$ is a number field and $v$ is a nonzero prime ideal of the integers of $K$,
  then the integers of $K_v$ is a compact open subgroup.
\end{theorem}
\begin{proof} Openness should follow from the fact that the integers are
  $\{x : v(x)<v(1/\pi)\}$ where $\pi$ is a uniformizer. Compactness needs
  finiteness of the residue field $\mathcal{O}_K/v$.
\end{proof}

Once we have this, the above result from mathlib gives us
\begin{theorem}
  \lean{NumberField.AdeleRing.locallyCompactSpace}
  \label{NumberField.AdeleRing.locallyCompactSpace}
  \uses{NumberField.instCompactSpaceAdicCompletionIntegers}
  \discussion{253}
  \leanok
  The adeles of a number field are locally compact.
\end{theorem}
\begin{proof}
  The adeles of a number field are a product of the finite adeles and the infinite adeles
  so it suffices to prove that the finite and infinite adeles are locally compact.
  The infinite adeles are just isomorphic to $\R^n$ as a topological space, so they're certainly
  locally compact. As for the finite adeles,
  the mathlib theorem {\tt RestrictedProduct.locallyCompactSpace\_of\_addGroup}
  says that a restricted product of locally compact additive groups with respect to open compact
  subgroups is locally compact, so this reduces us the previous result.
\end{proof}

\section{Base change}

The ``theorem'' we want is that if $L/K$ is a finite extension of number fields,
then $\A_L=L\otimes_K\A_K$. This isn't a theorem though, this is actually a \emph{definition}
(the map between the two objects) and a theorem about
the definition (that it's an isomorphism). In fact the full claim is that it is both a homeomorphism
and an $L$-algebra isomorphism. Before we can prove the theorem, we need to make the
definition.

Recall that the adeles $\A_K$ of a number field is a product $\A_K^\infty\times K_\infty$
of the finite adeles and the infinite adeles. So our ``theorem'' follows immediately from
the ``theorem''s that $\A_L^\infty=L\otimes_K\A_K^\infty$ and $L_\infty=L\otimes_KK_\infty$
(both of these equalities mean an algebraic and topological isomorphism).
We may thus treat the finite and infinite results separately.

\subsection{Base change for nonarchimedean completions.}

As pointed out above, the theory of finite adeles works fine for Dedekind domains.
So for the time being let~$A$ be a Dedekind domain. Recall that the \emph{height one spectrum}
of $A$ is the nonzero prime ideals of~$A$. Note that because we stick to the literature,
rather than to common sense, fields are Dedekind domains in mathlib, and the
height one spectrum of a field is empty. The reason I don't like allowing fields
to be Dedekind domains is that geometrically the definition of Dedekind
domain in the literature is ``smooth affine curve, or a point''. But many theorems in algebraic
geometry begin ``let $C$ be a smooth curve'', rather than ``let $C$ be a smooth curve or a point''.

Let $K$ be the field of fractions of $A$. If $v$ is in the height one spectrum of $A$,
then we can put the $v$-adic topology on $A$ and on $K$, and consider the completions
$A_v$ and $K_v$. The finite adele ring $\mathbb{A}_{A,K}^\infty$ is defined to be
the restricted product of the $K_v$ with respect to the $A_v$, as $v$ runs over
the height one spectrum of $A$. It is topologised by making $\prod_v A_v$ open with
the product topology (here $A_v$ has the $v$-adic topology).

Now let~$L/K$ be a finite separable extension, and let $B$ be the integral closure of~$A$ in~$L$.
We want to relate the finite adeles of $K$ and of $L$. We work place by place, starting by fixing
one place $w$ of $B$ and analysing the relation of $L_w$ and $B_w$ to the completions $K_v$
and $A_v$ where $v$ is the place of $A$ dividing $w$.

So let $w$ be a nonzero prime ideal of $B$. Say $w$ lies over $v$, a prime ideal of $A$.
Then we can put the $w$-adic topology on $L$ and the $v$-adic topology on~$K$. Furthermore
we can equip $K$ with an additive $v$-adic valuation, that is,
a function also called $v$ from $K$ to $\Z\cup\{\infty\}$ normalised so that if $\pi$ is a uniformiser
for $v$ then $v(\pi)=1$. Similarly we consider $w$ as a function from $L$ to $\Z\cup\{\infty\}$.
The next lemma explains how these valuations are related.

\begin{lemma}
  \label{IsDedekindDomain.HeightOneSpectrum.valuation_comap}
  \lean{IsDedekindDomain.HeightOneSpectrum.valuation_comap}
  \leanok
  If $i:K\to L$ denotes the inclusion then for $k\in K$ we have
  $e\times w(i(k))=v(k)$, where $e$ is the ramification index of $w/v$
  (recall that valuations here are written additively, unlike in mathlib).
\end{lemma}
\begin{proof}
  \leanok
  Standard (and formalized).
\end{proof}

\begin{definition}
  \lean{IsDedekindDomain.HeightOneSpectrum.adicCompletionComapSemialgHom}
  \label{IsDedekindDomain.HeightOneSpectrum.adicCompletionComapSemialgHom}
  \uses{IsDedekindDomain.HeightOneSpectrum.valuation_comap}
  \leanok
  There's a natural ring map $K_v\to L_w$ extending the map $K\to L$.
  It is defined by completing
  the inclusion $K\to L$ at the finite places $v$ and $w$ (which can be done
  because the previous lemma shows that the map is uniformly continuous for the $v$-adic
  and $w$-adic topologies).
\end{definition}

\begin{lemma}
  \lean{IsDedekindDomain.HeightOneSpectrum.valued_adicCompletionComap}
  \label{IsDedekindDomain.HeightOneSpectrum.valued_adicCompletionComap}
  \uses{IsDedekindDomain.HeightOneSpectrum.valuation_comap,
  IsDedekindDomain.HeightOneSpectrum.adicCompletionComapSemialgHom}
  \leanok
  If $i_v:K_v\to L_w$ denotes the map of the previous definition
  then for $x\in K_v$ we have
  $e\times w(i(k))=v(k)$, where $e$ is the ramification index of $w/v$.
\end{lemma}
\begin{proof}
  \leanok
  Follows by continuity from lemma~\ref{IsDedekindDomain.HeightOneSpectrum.valuation_comap}.
\end{proof}

\begin{lemma}
  \lean{IsDedekindDomain.HeightOneSpectrum.adicCompletionComapSemialgHom.mapadicCompletionIntegers}
  \label{IsDedekindDomain.HeightOneSpectrum.adicCompletionComapSemialgHom.mapadicCompletionIntegers}
  \uses{IsDedekindDomain.HeightOneSpectrum.valued_adicCompletionComap}
  \leanok
  The map $i_v:K_v\to L_w$ sends the integer ring $A_v$ into $B_w$.
\end{lemma}
\begin{proof}
  \leanok
  The integer ring is defined by $v\geq0$ (or $v\leq 1$ in mathlib, which uses multiplicative
  valuations) so the result follows from \ref{IsDedekindDomain.HeightOneSpectrum.valued_adicCompletionComap}.
\end{proof}
\begin{theorem}
  \lean{IsDedekindDomain.HeightOneSpectrum.adicCompletionComap_isModuleTopology}
  \label{IsDedekindDomain.HeightOneSpectrum.adicCompletionComap_isModuleTopology}
  \uses{IsDedekindDomain.HeightOneSpectrum.adicCompletionComapSemialgHom}
  \leanok
  \discussion{326}
  Giving $L_w$ the $K_v$-algebra structure coming from the natural map $K_v\to L_w$,
  the $w$-adic topology on $L_w$ is the $K_v$-module topology.
\end{theorem}
\begin{proof}
  Any basis for $L$ as a $K$-vector space spans $L_w$ as a $K_v$-module, so $L_w$ is
  finite-dimensional over $K_v$ and the module topology is the same as the product
  topology. So we need to establish that the product topology on $L_w=K_v^n$ is
  the $w$-adic topology. But the $w$-adic topology is induced by the $w$-adic norm,
  which makes $L_w$ into a normed $K_v$-vector space, and (after picking a basis)
  the product norm on $L_w=K_v^n$ also makes $L_w$ into a normed $K_v$-vector space.
  So the result follows from the standard fact (see for example the lemma on p52
  of Cassels-Froelich, formalized as {\tt ContinuousLinearEquiv.ofFinrankEq} in mathlib)
  that any two norms on a finite-dimensional vector space over
  a complete field are equivalent (and thus induce the same topology).
\end{proof}

Because of the commutative diagram
\begin{center}
\begin{tikzcd}
K_v \arrow{r} & L_w  \\
K \arrow{u} \arrow{r} & L \arrow{u}
\end{tikzcd}
\end{center}

we can view $L_w$ as an $L\otimes_KK_v$-algebra.

Now instead of fixing $w$ upstairs, we fix $v$ downstairs and consider all $w$ lying
over it at once. So say $v$ is in the height one spectrum of $A$.

\begin{lemma}
  \lean{IsDedekindDomain.HeightOneSpectrum.Extension.finite}
  \label{IsDedekindDomain.HeightOneSpectrum.Extension.finite}
  \leanok
  There are only finitely many primes $w$ of $B$ lying above $v$.
\end{lemma}
\begin{proof}
  \leanok
  This is a standard fact about Dedekind domains. The key input is
  mathlib's theorem {\tt primesOver\_finite}.
\end{proof}

We write $w|v$ to denote the fact that $w$ is a prime of $B$ above $v$ of $A$.

\begin{definition}
  \lean{IsDedekindDomain.HeightOneSpectrum.adicCompletionComapSemialgHom'}
  \label{IsDedekindDomain.HeightOneSpectrum.adicCompletionComapSemialgHom'}
  \uses{IsDedekindDomain.HeightOneSpectrum.adicCompletionComapSemialgHom}
  \leanok
  The product of the maps $K_v\to L_w$ for $w|v$ is a natural ring map $K_v\to\prod_{w|v}L_w$
  lying over $K\to L$.
\end{definition}

Because $K_v\to\prod_{w|v}L_w$ lies over $K\to L$, there's an induced $L$-algebra
map $L\otimes_KK_v\to\prod_{w|v}L_w$. We are now able to state one of the key results
in this section. The proof is probably the hardest proof
in this section to formalize.

\begin{theorem}
  \lean{IsDedekindDomain.HeightOneSpectrum.adicCompletionComapAlgEquiv}
  \label{IsDedekindDomain.HeightOneSpectrum.adicCompletionComapAlgEquiv}
  \uses{IsDedekindDomain.HeightOneSpectrum.adicCompletionComapSemialgHom',
  IsDedekindDomain.HeightOneSpectrum.Extension.finite}
  \leanok
  The induced $L$-algebra homomorphism $L\otimes_KK_v\to\prod_{w|v}L_w$ is an
  isomorphism of rings.
\end{theorem}
\begin{proof}

  My current proposal to formalize this is as follows. The map is surjective
  because the image is dense and closed; this has been formalized already.
  It is also a $K_v$-algebra homomorphism if we give $L_w$ the obvious $K_v$-algebra
  structure. Thus we can conclude the result if we can prove that both spaces are
  finite-dimensional and have the same dimension. The $K_v$-dimension of $L\otimes_KK_v$
  is equal to the $K$-dimension of $L$, which is $\sum_{w|v}e_wf_w$ using the standard
  notation that $e_w$ is the ramification index of $w$ and $f_w$ the residue degree
  (this result is in mathlib). So it suffices to prove that $[L_w:K_v]=e_wf_w$.
  We already have that $e_w$ (defined globally) is equal to the local ramification
  index (defined as the factor by which the valuations differ on $K$). So what is left
  is to prove that (i) the residue field extension induced by $L_w/K_v$ has degree is equal to the
  globally-defined $f_w$, (ii) an extension of local fields has degree $ef$. Now (i) sounds
  straightforward given what we have (the map from $A$ to $\mathcal{O}_v$ has kernel $v$ and
  dense image) and (ii) is true for any complete discretely-valued field; I am not suggesting
  we formalize the following proof, but at least it represents a rigorous justification:
  A field complete with respect to a discrete valuation is \emph{stable} in the sense
  of the book by Bosch-G\"{u}ntzer-Remmert (Prop 3.6.2.1), so every finite extension of such a field
  is cartesian (def 3.6.1.1) and thus $ef=n$ (Prop 3.6.2.4, (iii) implies (ii)). Note
  that if you weaken the hypotheses too much then there are counterexamples; it's possible
  to have $ef<n$ and BGR goes into details.

\end{proof}

\begin{theorem}
  \label{IsDedekindDomain.HeightOneSpectrum.prodAdicCompletionComap_isModuleTopology}
  \lean{IsDedekindDomain.HeightOneSpectrum.prodAdicCompletionComap_isModuleTopology}
  \uses{IsDedekindDomain.HeightOneSpectrum.adicCompletionComap_isModuleTopology}
  \leanok
  For $v$ fixed, the product topology on $\prod_{w|v}L_w$ is the $K_v$-module
  topology.
\end{theorem}
\begin{proof}
  \leanok
  This is a finite product of $K_v$-modules each of which has the $K_v$-module topology
  by~\ref{IsDedekindDomain.HeightOneSpectrum.adicCompletionComap_isModuleTopology},
  and the product topology is the module topology for a finite product of modules each of which
  has the module topology (this is in mathlib).
\end{proof}

\begin{theorem}
  \label{IsDedekindDomain.HeightOneSpectrum.adicCompletionComapContinuousAlgEquiv}
  \lean{IsDedekindDomain.HeightOneSpectrum.adicCompletionComapContinuousAlgEquiv}
  \uses{IsDedekindDomain.HeightOneSpectrum.prodAdicCompletionComap_isModuleTopology}
  If we give $L\otimes_KK_v$ the $K_v$-module topology then the $L$-algebra isomorphism
  $L\otimes_K K_v\cong\prod_{w|v}L_w$ is also a homeomorphism.
\end{theorem}
\begin{proof} Indeed, is a $K_v$-algebra isomorphism between two modules each of which
  have the module topology, and any module map is automorphically continuous for the
  module topologies.
\end{proof}

We now start thinking about what's going on at the integral level. We write $A_v$
for the integers of $K_v$ and $B_w$ for the integers of $L_w$.

\begin{theorem}
  \lean{IsDedekindDomain.HeightOneSpectrum.adicCompletionComapAlgEquiv_integral}
  \label{IsDedekindDomain.HeightOneSpectrum.adicCompletionComapAlgEquiv_integral}
  \uses{IsDedekindDomain.HeightOneSpectrum.adicCompletionComapAlgEquiv}
  \leanok
  The isomorphism $L\otimes_KK_v\to\prod_{w|v}L_w$ induces an isomorphism
  $B\otimes_AA_v\to \prod_{w|v}B_w$
  for all $v$ in the height one spectrum of $A$.
\end{theorem}
\begin{proof}
  \leanok
  Certainly the image of the integral elements are integral. The argument in the other
  direction is more delicate. My original plan was to follow Cassels--Froehlich,
  Cassels' article ``Global fields'', section 12 lemma, p61, which proves it for
  all but finitely many primes, but \href{https://github.com/ImperialCollegeLondon/FLT/pull/400}
  {a PR by Matthew Jasper} gives another approach which works for all primes.
  Jasper's argument is to show that the closure of $A$ in $K_v$ is $A_v$
  for a valuation on a Dedekind domain, and then that the closure of $A$ in $\prod_{v\in S}K_v$
  is $\prod_{v\in S}A_v$ for $S$ a finite set of valuations (using the Chinese
  remainder theorem). Applying this to $B$ we get that the closure of $B$ in $\prod_{w|v}L_w$
  is $\prod_{w|v}B_w$. He then shows that this closure is the image of
  $B\otimes_A\mathcal{O}_v$ (by showing that this image is closed because it's open),
  giving surjectivity; injectivity follows from the statement
  that $L\otimes_KK_v=\prod_{w|v}L_w$.
\end{proof}

A summary of what we have so far: for all finite places $v$ of $A$
we have shown that the natural map $L\otimes_KK_v\to\prod_wL_w$
is an isomorphism of $L$-algebras, and that if $L\otimes_KK_v$ has
the $K_v$-module topology and each $L_w$ has the valuation topology
then this map is also a homeomorphism. Furthermore we have shown
that there is an induced algebraic isomorphism $B\otimes_AA_v\equiv\prod_w B_w$
on the subrings of the left and right hand sides.

Recall that the finite adeles $\A_{A,K}^\infty$ is defined in mathlib to be
the restricted product of the $K_v$ with respect to the $A_v$, equipped with a certain
restricted product topology (which is not the subspace topology of the product
topology, indeed $\prod_v A_v$ is open in this topology). We have seen in
definition~\ref{IsDedekindDomain.HeightOneSpectrum.adicCompletionComapSemialgHom} that
there's a map $K_v\to L_w$ if $w|v$, extending $K\to L$, and we have seen in
theorem~\ref{IsDedekindDomain.HeightOneSpectrum.adicCompletionComapSemialgHom.mapadicCompletionIntegers}
that this sends $A_v$ to $B_w$. We conclude

\begin{definition}
  \label{IsDedekindDomain.FiniteAdeleRing.mapSemialgHom}
  \lean{IsDedekindDomain.FiniteAdeleRing.mapSemialgHom}
  \uses{IsDedekindDomain.HeightOneSpectrum.adicCompletionComapSemialgHom,
  IsDedekindDomain.HeightOneSpectrum.adicCompletionComapSemialgHom.mapadicCompletionIntegers}
  \leanok
  There's a natural ring homomorphism
  $\A_{A,K}^\infty\to\A_{B,L}^\infty$ lying over $K\to L$.
\end{definition}

Hence there's a natural $L$-algebra homomorphism $L\otimes_K\A_{A,K}^\infty\to\A_{B,L}^\infty$.

Our next goal in this section is the following two results. First the algebraic claim:

\begin{theorem}
  \label{IsDedekindDomain.FiniteAdeleRing.baseChangeAlgEquiv}
  \lean{IsDedekindDomain.FiniteAdeleRing.baseChangeAlgEquiv}
  \uses{IsDedekindDomain.HeightOneSpectrum.adicCompletionComapAlgEquiv_integral,
    RestrictedProduct.relabelIso,IsDedekindDomain.FiniteAdeleRing.tensorProductAlgEquiv,
    IsDedekindDomain.FiniteAdeleRing.baseChangeIntegralAlgEquiv,
    IsDedekindDomain.AKLB.tensorProduct_module_algEquiv}
  \leanok
  \discussion{243}
  This natural map $L\otimes_K\A_{A,K}^\infty\to\A_{B,L}^\infty$ is an isomorphism.
\end{theorem}

Now $L\otimes_K\A_{A,K}^\infty$ is an $\A_{A,K}^\infty$-module and hence can be given
the $\A_{A,K}^\infty$-module topology. We also claim

\begin{theorem}
  \label{IsDedekindDomain.FiniteAdeleRing.baseChangeContinuousAlgEquiv}
  \lean{IsDedekindDomain.FiniteAdeleRing.baseChangeContinuousAlgEquiv}
  \uses{IsDedekindDomain.HeightOneSpectrum.adicCompletionComapAlgEquiv_integral}
  \leanok
The induced $L$-algebra morphism
  $L\otimes_K\A_{A,K}^\infty\to\A_{B,L}^\infty$ is a topological isomorphism.
\end{theorem}

Informally, the proofs are simple: componentwise we know
that $L\otimes_KK_v$ is isomorphic both algebraically and
topologically to $\prod_{w|v}L_w$, and that this isomorphism
sends the open set $B\otimes_AA_v$ homeomorphically onto
$\prod_{w|v}B_w$, so now it's ``just a case of putting everything
together''. Formally, we really need to spell this out, as there is a lot
going on. We do this in the
next subsection.

\subsection{Base change for nonarchimedean completions.}

As usual we are in the AKLB set-up, so in particular $K$ is the field
of fractions of the Dedekind domain $A$, $L/K$ is a finite
separable extension, and $B$ is the integral closure of $A$
in $L$. The goal in this subsection is to spell out the following argument: Assume that
$L\otimes_KK_v\cong\prod_{w|v}L_w$
algebraically and topologically for all $v$, with $B\otimes_AA_v$ identified with $\prod_{w|v}B_w$.
Then $L\otimes_K\A_K^\infty\cong\A_L^\infty$, algebraically and topologically. Here
the tensor products $L\otimes_K R$ (for $R$ a $K$-algebra with a topology) are all being
given the $R$-module topology, which if we choose a basis for $L/K$ is just the product
topology.

We start with the following observation. If $M$ is a $K$-module then there's a canonical
map $B\otimes_AM\to L\otimes_KM$ sending $b\otimes m$ to $b\otimes m$ (this follows from the
universal property of the tensor product). Our first goal
is to show that this map is an isomorphism. Let us establish some lemmas first.

\begin{lemma}
  \label{IsDedekindDomain.dvd_norm}
  If $0\not=b\in B$ then there exists $0\not=a\in A$ such that $b$ divides
  the image of $a$ in $B$.
\end{lemma}
\begin{remark} Is this already in mathlib?
\end{remark}
\begin{proof} Let $a=N_{L/K}(b)$, the norm. This is known to take nonzero elements of $L$
to nonzero elements of $K$ (because the norm is the determinant of an invertible linear map)
and integral elements to integral elements. Furthermore $a/b\in L$ is the the product of the
conjugates of $b$ in some normal closure of $L$, and hence it is integral, and thus in $B$.
\end{proof}

\begin{corollary}
  \label{IsDedekindDomain.AKLB.surjective_tensorProduct_map}
  \uses{IsDedekindDomain.dvd_norm}
  The $A$-bilinear map $B\times K\to L$ sending $(b,k)$ to $bk$ is surjective.
\end{corollary}
\begin{proof} Given $\lambda\in L$ write it as $n/d$ with $0\not=d\in B$. Choose $0\not=a\in A$
  and $b\in B$ with $db=a$ and then note $\lambda=nb/a=nb\times a^{-1}$.
\end{proof}

\begin{corollary}
  \label{IsDedekindDomain.AKLB.tensorProduct_algEquiv}
  \uses{IsDedekindDomain.dvd_norm}
  The natural map $B\otimes_AK\to L$ is a $B$-algebra isomorphism.
\end{corollary}
\begin{proof}

We write down an inverse. Regard $B\otimes_AK$ as a $B$-algebra via the action on the left.
Note that at this point it's not even clear that $B\otimes_AK$ is a field. We have the
structure map $B\to B\otimes_AK$ sending $b$ to $b\otimes1$, which is $B$-linear. I claim
that every nonzero element of $B$ gets sent to an invertible element of $B\otimes_AK$.
Indeed, if $b\not=0$ and (using the previous lemma) we choose $0\not=a\in A$ such that
$bb'=a$, then $(b\otimes1)(b'\otimes\frac1a)=1$. Thus by the universal property of
localisation, the $B$-linear map $B\to B\otimes_AK$ extends to a ring homomorphism from the field
of fractions of $B$ to $B\otimes_AK$, which we claim is our desired inverse.
Checking that both composites are the identity should be straightforward. Starting
with $B\otimes_AK$ we only have to check on elements of the form $b\otimes k$;
starting with $L$ we only have to check on elements of $B$. Hopefully both are
straightforward.
\end{proof}

\begin{corollary}
  \label{IsDedekindDomain.AKLB.tensorProduct_module_algEquiv}
  \uses{IsDedekindDomain.AKLB.tensorProduct_algEquiv}
  If $M$ is any $K$-module then the canonical map $B\otimes_A M\to L\otimes_K M$
  is an isomorphism.
\end{corollary}
\begin{proof} We can factor this map as $B\otimes_AM\cong B\otimes_A(K\otimes_KM)\cong
  (B\otimes_A K)\cong_KM\to L\otimes_KM$ and we just showed that the latter map was an
  isomorphism.
\end{proof}

We now need to explain how tensor products sometimes commute with restricted products.
Something we will need along the way is

\begin{theorem} $B$ is a finitely-presented $A$-module.
  \label{IsDedekindDomain.AKLB.finitePresentation}
\end{theorem}
\begin{proof} $A$ is Noetherian as it is a Dedekind domain, so it suffices to prove that $B$ is
  finitely-generated as an $A$-module. But this is in mathlib already (a proof is
  around line 101 of {\tt BaseChange.lean} in FLT at the time of writing).
\end{proof}

The reason we care about this is the following.

\begin{theorem}
  \label{pi_tensorProduct_of_finitePresentation}
  If $R$ is a commutative ring, if $M$ is a finitely presented $R$-module
  and if $N_i$ are a collection of $R$-modules, then the canonical map
  $M\otimes_R\prod_i N_i\to\prod_i(M\otimes_R N_i)$ is an isomorphism.
\end{theorem}
\begin{proof} If $M$ is finite and free then Maddy Crim has already formalized this
  in FLT. For the general case present $M$ as $R^a\to R^b\to M\to 0$ and use that tensor
  products and arbitrary products preserve surjections.
\end{proof}

\begin{corollary}
  \label{IsDedekindDomain.pi_tensorProduct}
  \uses{IsDedekindDomain.AKLB.finitePresentation,pi_tensorProduct_of_finitePresentation}
  If $S$ is a finite set of nonzero primes of $A$ then the natural map
  $B\otimes((\prod_{v\in S}K_v)\times(\prod_{v\notin S}A_v))\to
  (\prod_{v\in S}(B\otimes_AK_v))\times(\prod_{v\notin S}(B\otimes_AA_v))$
  is an isomorphism.
\end{corollary}
\begin{proof} Follows from the previous two theorems.
\end{proof}

Recall that $\A_K^\infty$ is the finite adeles of $K$,
defined as the restricted product of the $K_v$ with respect to the $A_v$,
where $v$ runs through the nonzero primes of $A$. Let $R$ denote the restricted
product of the $B\otimes_A K_v$ with respect to the $B\otimes_A A_v$.

\begin{corollary}
  \label{IsDedekindDomain.FiniteAdeleRing.IntegraltensorProductAlgEquiv_aux1}
  \uses{IsDedekindDomain.pi_tensorProduct}
  The natural map $B\otimes_A\A_K^\infty\to R$ is a $B$-algebra isomorphism.
\end{corollary}
\begin{proof} This follows from the previous corollary and the fact that
  tensor products commute with filtered colimits.
\end{proof}

Recall from earlier in this section that if $v$ is a finite place of $A$ then the natural map from
$B\otimes_A K_v$ to $L\otimes_KK_v$ is an isomorphism, and recall from the previous section
that the natural map from $L\otimes_KK_v$ to $\prod_{w|v}L_w$ was also an isomorphism.
This isomorphism sends $B\otimes_A A_v$ to $\prod_{w|v}B_w$ (I thank Matthew Jasper for
pointing out to me that this statement was true at all primes, not just at unramified primes).
Finally, the set of $w$ of $B$ dividing a fixed place $v$ of $A$ is finite.
Let's now formalize the abstract statement which we need.

\begin{definition}
  \label{RestrictedProduct.relabelIso}
  Let $V$ and $W$ be index sets, and let $f:W\to V$ be a map with finite fibres.
  Let $R_v$ be sets, with subsets $S_v$, let $L_w$ be sets with subsets $B_w$,
  and say for all $v\in V$ we're given a bijection $R_v\to\prod_{w|v}L_w$,
  sending $S_v$ to $\prod_{w|v}B_w$. Then there's an induced bijection between
  the restricted products $\prod'_v R_v$ and $\prod'_w L_w$.
\end{definition}

\begin{corollary}
  \label{IsDedekindDomain.FiniteAdeleRing.IntegraltensorProductAlgEquiv_aux2}
  \uses{RestrictedProduct.relabelIso}
  The ring $R$ introduced above (the restricted
  product of the $B\otimes_A K_v$ with respect to the $B\otimes_A A_v$)
  is isomorphic to $\mathbb{A}_L$.
\end{corollary}
\begin{proof} Let $V$ be the finite places of $K$ and $W$ the finite places of $L$,
  and the result follows from the previous theorem.
\end{proof}
From this, we can deduce the theorem we claimed earlier:

\begin{theorem}
  \label{IsDedekindDomain.FiniteAdeleRing.baseChangeIntegralAlgEquiv}
  \uses{RestrictedProduct.relabelIso,
  IsDedekindDomain.FiniteAdeleRing.IntegraltensorProductAlgEquiv_aux1,
  IsDedekindDomain.FiniteAdeleRing.IntegraltensorProductAlgEquiv_aux2}
  The natural map $B\otimes_A\A_K^\infty\to\A_L^\infty$ is a $B$-algebra
  isomorphism.
\end{theorem}
\begin{proof}
  This map factors through the auxiliary ring~$R$ so the result follows
  from the previous two constructions.
\end{proof}

Because this map factors through the isomorphism $B\otimes_A\A_K^\infty\to L\otimes_K\A_K^\infty$
we can finally deduce that the natural map $L\otimes_K\A_K^\infty\to\A_L^\infty$ is an algebraic
  isomorphism.
\begin{proof}
  \proves{IsDedekindDomain.FiniteAdeleRing.baseChangeAlgEquiv}
  Follows immediately from theorem~\ref{IsDedekindDomain.FiniteAdeleRing.baseChangeIntegralAlgEquiv}
  and theorem~\ref{IsDedekindDomain.AKLB.tensorProduct_module_algEquiv}.
\end{proof}

We still need to talk about topologies though.

\subsection{Base change for infinite adeles}

Recall that if $K$ is a number field then the infinite adeles of $K$ are defined
to be the product $\prod_{v\mid\infty} K_v$ of all the completions of $K$ at the
infinite places.

The result we need here is that if $L/K$ is a finite extension of number fields,
then the map $K\to L$ extends to a continuous $K$-algebra map $K_\infty\to L_\infty$,
and thus to a continuous $L$-algebra isomorphism $L\otimes_KK_\infty\to L_\infty$.
Perhaps a cheap proof would be to deduce it from the fact that $K_\infty=K\otimes_{\Q}\R$.

The overall strategy is to first establish, for each infinite place $v$ of $K$, homeomorphisms
between for the completion $K_v$ and the product $\prod_{w\mid v}L_w$ of completions of $L$ at
all infinite places $w$ of $L$ lying above $v$.
We then use these homeomorphisms to construct base change for the infinite adele ring.

\subsubsection{Weak approximation at infinite places}

First, we require a preliminary result that $K$ is dense inside any product of
completions $\prod_{v\in S} K_v$ of $K$ at infinite places.

\begin{theorem}
  \label{NumberField.InfinitePlace.Completion.denseRange_algebraMap_subtype_pi}
  \lean{NumberField.InfinitePlace.Completion.denseRange_algebraMap_subtype_pi}
  \leanok
  Let $S$ be a set of infinite places of $K$. The image of $K$ under the embedding
  $K\hookrightarrow (K_v)_{v\in S}; k \mapsto (k)_v$ is dense in the product topology.
\end{theorem}
\begin{proof}
  Let $(K, v)$ denote $K$ equipped with the topology induced by the infinite place $v$.
  It suffices to show that the image of $K$ under the embedding
  $K\hookrightarrow \prod_{v\mid \infty}(K, v)$ is dense in the product topology.
  By a standard analytic argument, for each $v$ it is possible to select a sequence $(x_n^{(v)})_n$
  with the property that $x_n^{(v)} \to 1$ in $v$'s topology, while $x_n^{(v)} \to 0$ in any other
  infinite place's topology.
  Let $(z_v)_v \in \prod_{v\mid \infty}(K, v)$.
  For each $v$, the sequence $y_n := \sum_{v\mid\infty} x_n^{(v)}z_v$ in $K$ converges to
  $z_v$ in $v$'s topology.
  So the embedded image of $y_n$ in $\prod_{v\mid \infty}(K, v)$ converges to $(z_v)_v$ in the
  product topology.
\end{proof}

\subsubsection{Dimensionality of $\prod_{w\mid v}L_w$ as a $K_v$-vector space}

This subsection contains a result that the $K_v$-dimension of
$L \otimes_K K_v$ is equal to the $K_v$-dimension of $\prod_{w\mid v}L_w$.

\begin{theorem}
  \label{NumberField.InfinitePlace.Completion.finrank_pi_eq_finrank_tensorProduct}
  \lean{NumberField.InfinitePlace.Completion.finrank_pi_eq_finrank_tensorProduct}
  For a fixed infinite place $v$ of $K$, we have
  $\text{dim}_{K_v} \prod_{w\mid v} L_w = \text{dim}_{K_v} L\otimes_K K_v$.
\end{theorem}

\subsubsection{Base change at infinite places}

\begin{definition}
  \label{NumberField.InfinitePlace.Completion.piExtensionPlace}
  \lean{NumberField.InfinitePlace.Completion.piExtensionPlace}
  \leanok
  Let $v$ be an infinite place of $K$. There is a continuous $K$-algebra homomorphism
  $K_v \to \prod_{w\mid v}L_w$, whose restriction to $K$ corresponds to the global embedding
  of $K$ into $(L_w)_w$.
\end{definition}

The map in~\ref{NumberField.InfinitePlace.Completion.piExtensionPlace} can be lifted to an
$L$-algebra homomorphism defined on $L\otimes_K K_v$.

\begin{definition}
  \label{NumberField.InfinitePlace.Completion.baseChange}
  \lean{NumberField.InfinitePlace.Completion.baseChange}
  \leanok
  \uses{NumberField.InfinitePlace.Completion.piExtensionPlace}
  Let $v$ be an infinite place of $K$. There is a natural $L$-algebra homomorphism
  $L\otimes_K K_v \to \prod_{w\mid v}L_w$, whose restriction to $1\otimes_K K_v$ corresponds to
  the map in~\ref{NumberField.InfinitePlace.Completion.piExtensionPlace}.
\end{definition}

\begin{theorem}
  \label{NumberField.InfinitePlace.Completion.baseChange_surjective}
  \lean{NumberField.InfinitePlace.Completion.baseChange_surjective}
  \leanok
  \uses{NumberField.InfinitePlace.denseRange_algebraMap_subtype_pi}
  For a fixed infinite place $v$ of $K$, the map $L\otimes_K K_v \to\prod_{w\mid v}L_w$ is
  surjective.
\end{theorem}
\begin{proof}
  Let $(x_i)_i$ be a $K_v$-basis of $\prod_{w\mid v}L_w$. By the density of $L$ in
  $\prod_{w\mid v}L_w$
  (Theorem~\ref{NumberField.InfinitePlace.Completion.denseRange_algebraMap_subtype_pi}), we can
  find $\alpha_i \in L$ arbitrarily close to $x_i\prod_{w\mid v}L_w$ with respect to the sup norm
  when embedded globally in $\prod_{w\mid v}L_w$.
  In particular, it is possible to choose such $\alpha_i$ so that the matrix representing
  the vector $((\alpha_i)_{w \mid v})_i$ in the basis $(x_i)_i$ has non-zero determinant.
  Since $(\alpha_i)_{w \mid v}$ is the image of $1\otimes \alpha_i$ under base change, we have
  that $(1 \otimes \alpha_i)_i$ also forms a basis of $L\otimes_K K_v$, and base change
  is surjective.
\end{proof}

\begin{theorem}
  \label{NumberField.InfinitePlace.Completion.baseChange_injective}
  \lean{NumberField.InfinitePlace.Completion.baseChange_injective}
  \leanok
  \uses{NumberField.InfinitePlace.Completion.finrank_pi_eq_finrank_tensorProduct}
  For a fixed infinite place $v$ of $K$, the map $L\otimes_K K_v \to\prod_{w\mid v}L_w$ is
  injective.
\end{theorem}
\begin{proof}
  The $L$-algebra map $L\otimes_K K_v \to\prod_{w\mid v}L_w$ can equivalently be thought of
  as $K_v$-linear, which is injective, because it is surjective by
  Theorem~\ref{NumberField.InfinitePlace.baseChange_surjective}, and both side have the same
  $K_v$-dimension by
  Theorem~\ref{NumberField.InfinitePlace.Completion.finrank_pi_eq_finrank_tensorProduct}.
\end{proof}

We have established that the map of
Definition~\ref{NumberField.InfinitePlace.Completion.baseChange} gives an $L$-algebra isomorphism
between $L\otimes_K K_v$ and $\prod_{w\mid v}L_w$.
The left-hand side is given the $K_v$-module topology, while we show that the right-hand side also
has the $K_v$-module topology.
\begin{theorem}
  \label{NumberField.InfinitePlace.Completion.instIsModuleTopologyValEqComapAlgebraMap_fLT}
  \lean{NumberField.InfinitePlace.Completion.instIsModuleTopologyValEqComapAlgebraMap_fLT}
  \leanok
  If $w \mid v$ is an infinite place of $L$ lying above the infinite place $v$ of $K$, then
  $L_w$ has the $K_v$-module topology.
\end{theorem}
\begin{proof}
  Because $L_w$ is a finite-dimensional normed $K_v$ vector space, there exists a $K_v$-linear
  linear homeomorphism $L_w \cong K_v^n$, from which $L_w$ has the $K_v$-module topology.
\end{proof}

\begin{theorem}
  \label{NumberField.InfinitePlace.Completion.baseChangeEquiv}
  \lean{NumberField.InfinitePlace.Completion.baseChangeEquiv}
  \leanok
  \uses{NumberField.InfinitePlace.Completion.baseChange,
    NumberField.InfinitePlace.Completion.baseChange_surjective,
    NumberField.InfinitePlace.Completion.baseChange_injective,
    NumberField.InfinitePlace.Completion.instIsModuleTopologyValEqComapAlgebraMap_fLT}
  Let $v$ be an infinite place of $K$. There is a natural $L$-algebra homeomorphism
  $L\otimes_K K_v \cong_L \prod_{w\mid v}L_w$, whose restriction to $1\otimes_K K_v$ corresponds to
  the map in~\ref{NumberField.InfinitePlace.Completion.piExtensionPlace}.
\end{theorem}
\begin{proof}
  The map in~\ref{NumberField.InfinitePlace.Completion.baseChange} is an $L$-algebra
  isomorphism by Theorem~\ref{NumberField.InfinitePlace.Completion.baseChange_surjective}
  and Theorem~\ref{NumberField.InfinitePlace.Completion.baseChange_injective}.
  Every $K_v$-algebra isomorphism between two $K_v$-module topological spaces is a homeomorphism.
  Since the $L$-algebra isomorphism of
  Definition~\ref{NumberField.InfinitePlace.Completion.baseChange} can equivalently be viewed as
  a $K_v$-algebra isomorphism, it is also a homeomorphism.
\end{proof}

\begin{theorem}
  \label{NumberField.InfinitePlace.Completion.piEquiv}
  \lean{NumberField.InfinitePlace.Completion.piEquiv}
  \leanok
  \uses{NumberField.InfinitePlace.Completion.baseChangeEquiv}
  Let $v$ be an infinite place of $K$. There is a natural $K_v$-linear homeomorphism
  $K_v^{[L:K]} \cong_{K_v} \prod_{w\mid v}L_w$.
\end{theorem}
\begin{proof}
  Compose the $K_v$-linear isomorphism $K_v^{[L:K]} \cong \prod_{w\mid v}L_w$ with the $K_v$-linear
  version of base change~\ref{NumberField.InfinitePlace.Completion.baseChangeEquiv} to get the
  isomorphism in the statement.
  Since both sides have the $K_v$-module topology, then this is also a homeomorphism.
\end{proof}

\subsubsection{Base change for the infinite adele ring}

First, we induce a $K_{\infty}$-algebra on $L_{\infty}$ from the action of each $K_v$ on
$\prod_{w\mid v}L_w$.
Specifically, this means that for $x \in K_{\infty}$ and $y \in L_{\infty}$, we have
$(x \cdot y)_w = x_{v_w} \cdot y_w$, where $v_w$ is the restriction of $w$ to $K$.
We show that $L_{\infty}$ has the $K_{\infty}$-module topology.

\begin{theorem}
  \label{NumberField.InfiniteAdeleRing.piEquiv}
  \lean{NumberField.InfiniteAdeleRing.piEquiv}
  \leanok
  \uses{NumberField.InfinitePlace.Completion.piEquiv}
  There is a natural $K_{\infty}$-linear homeomorphism
  $K_{\infty}^{[L:K]} \cong_{K_{\infty}} L_{\infty}$.
\end{theorem}
\begin{proof}
  Using the isomorphisms $K_v^{[L:K]} \cong_{K_v} \prod_{w\mid v}L_w$ from
  Theorem~\ref{NumberField.InfinitePlace.Completion.piEquiv}, we clearly have a bijection
  $K_{\infty}^{[L:K]} \cong \prod_v\prod_{w \mid v} L_w \cong \prod_w L_w$.
  The $K_v$-linearity of each component isomorphism extends to $K_{\infty}$-linearity if the
  action of $\prod_v K_v$ on $\prod_w L_w$ is constant on the fibers of the restriction map on
  infinite places.
  In other words, if, for all $x \in K_{\infty}$ and $y \in L_{\infty}$, we have
  $(x \cdot y)_w = x_{v_w} \cdot y_w$, which is true by definition.
\end{proof}

\begin{theorem}
  \label{NumberField.InfiniteAdeleRing.instIsModuleTopology_fLT}
  \lean{NumberField.InfiniteAdeleRing.instIsModuleTopology_fLT}
  \leanok
  \uses{NumberField.InfiniteAdeleRing.piEquiv}
  $L_{\infty}$ has the $K_{\infty}$-module topology.
\end{theorem}
\begin{proof}
  Since $L_{\infty}$ is homeomorphic to a finite product of $K_{\infty}$ as a $K_{\infty}$-vector
  space, it has the $K_{\infty}$-module topology.
\end{proof}

\begin{theorem}
  \label{NumberField.InfiniteAdeleRing.baseChangeEquivAux}
  \lean{NumberField.InfiniteAdeleRing.baseChangeEquivAux}
  \leanok
  \uses{NumberField.InfinitePlace.Completion.baseChangeEquiv}
  There is a natural $L$-algebra isomorphism
  $L \otimes_K K_{\infty} \cong_L L_{\infty}$.
\end{theorem}
\begin{proof}
  This follows from the following chain of isomorphisms:
  \[
    L \otimes_K K_{\infty} \cong_L \prod_v (L \otimes_K K_v) \cong_L
      \prod_v \prod_{w\mid v}L_w \cong_L L_{\infty} .
  \]
  The first isomorphism is the standard $L$-algebra isomorphism
  $L \otimes_K \prod_v K_v \cong_L \prod_v (L \otimes_K K_v)$.
  The second isomorphism is given by the component $L$-algebra isomorphisms
  $L \otimes_K K_v \cong_L \prod_{w\mid v}L_w$ from
  Theorem~\ref{NumberField.InfinitePlace.Completion.baseChangeEquiv}.
\end{proof}

It remains to show that the map in~\ref{NumberField.InfiniteAdeleRing.baseChangeEquivAux} is a
homeomorphism.
Since both sides have the $K_{\infty}$-module topology, and since the $L$-algebra isomorphism
of~\ref{NumberField.InfiniteAdeleRing.baseChangeEquivAux} can equivalently be viewed as a
$K_{\infty}$-linear isomorphism, it is also a homeomorphism.

\begin{theorem}
  \label{NumberField.InfiniteAdeleRing.baseChangeEquiv}
  \lean{NumberField.InfiniteAdeleRing.baseChangeEquiv,
    NumberField.InfiniteAdeleRing.instIsModuleTopology_fLT}
  \leanok
  \uses{NumberField.InfiniteAdeleRing.baseChangeEquivAux}
  If $K\to L$ is a ring homomorphism between two number fields then there is a natural isomorphism
  (both topological and algebraic) $L\otimes_KK_\infty\cong L_\infty$.
\end{theorem}
\begin{proof}
  Since both sides of the $L$-algebra isomorphism
  in~\ref{NumberField.InfiniteAdeleRing.baseChangeEquivAux} have the $K_{\infty}$-module topology,
  and since the isomorphism can equivalently be viewed as a $K_{\infty}$-linear isomorphism,
  it is also a homeomorphism.
\end{proof}


\subsection{Base change for adeles}

From the previous results we deduce immediately that if $L/K$ is a finite extension
of number fields then there's a natural (topological and algebraic) isomorphism
$L\otimes_K\A_K\to \A_L$.

\begin{theorem}
  \label{NumberField.AdeleRing.baseChangeEquiv}
  \lean{NumberField.AdeleRing.baseChangeEquiv}
  \uses{NumberField.InfiniteAdeleRing.baseChangeEquiv,
  IsDedekindDomain.FiniteAdeleRing.baseChangeAlgEquiv}
  If $K\to L$ is a ring homomorphism between two number fields then there is a natural isomorphism
  (both topological and algebraic) $L\otimes_K\A_K\cong\A_L$.
\end{theorem}
\begin{proof}
  Follows from the previous results.
\end{proof}

Something else we shall need:

\begin{theorem}
  \label{NumberField.AdeleRing.baseChange_moduleTopology}
  \uses{NumberField.AdeleRing.baseChangeEquiv}
  If $K\to L$ is a ring homomorphism between two number fields then the topology on $\A_L$
  is the $\A_K$-module topology, where the module structure comes from the
  natural map $\A_K\to\A_L$.
\end{theorem}
\begin{proof}
  Indeed $\A_L\cong L\otimes_K\A_K$ is a homeomorphism, and
  the right hand side has the $\A_K$-module topology.
\end{proof}

\section{Discreteness and compactness}

We need that if $K$ is a number field then
$K\subseteq\mathbb{A}_K$ is discrete, and the quotient (with the
quotient topology) is compact. Here is a proposed proof.

\begin{theorem}
  \lean{Rat.AdeleRing.zero_discrete}
  \label{Rat.AdeleRing.zero_discrete}
  \leanok
  There's an open subset of $\A_{\Q}$ whose intersection with $\Q$ is $\{0\}$.
\end{theorem}
\begin{proof}
  Use $\prod_p{\Z_p}\times(-1,1)$. Any rational $q$ in this set is a $p$-adic
  integer for all primes $p$ and hence (writing it in lowest terms as $q=n/d$)
  satisfies $p\nmid d$, meaning that $d=\pm1$ and thus $q\in\Z$. The fact
  that $q\in(-1,1)$ implies $q=0$.
\end{proof}

\begin{theorem}
  \lean{NumberField.AdeleRing.zero_discrete}
  \label{NumberField.AdeleRing.zero_discrete}
  \uses{Rat.AdeleRing.zero_discrete,NumberField.AdeleRing.baseChangeEquiv}
  \leanok
  There's an open subset of $\A_{K}$ whose intersection with $K$ is $\{0\}$.
\end{theorem}
\begin{proof}
  By a previous result, we have $\A_K=K\otimes_{\Q}\A_{\Q}$.
  Choose a basis of $K/\Q$; then $K$ can be identified with $\Q^n\subseteq(\A_{\Q})^n$
  and the result follows from the previous theorem.
\end{proof}

\begin{theorem}
  \lean{NumberField.AdeleRing.discrete}
  \label{NumberField.AdeleRing.discrete}
  \uses{NumberField.AdeleRing.zero_discrete}
  \leanok
  The additive subgroup $K$ of $\A_K$ is discrete.
\end{theorem}
\begin{proof}
  If $x\in K$ and $U$ is the open subset in the previous lemma, then
  it's easily checked that $K\cap U=\{0\}$ implies $K\cap (U+x)=\{x\}$,
  and $U+x$ is open.
\end{proof}

For compactness we follow the same approach.

\begin{theorem}
  \lean{Rat.AdeleRing.cocompact}
  \label{Rat.AdeleRing.cocompact}
  \leanok
  The quotient $\A_{\Q}/\Q$ is compact.
\end{theorem}
\begin{proof}
  The space $\prod_p\Z_p\times[0,1]\subseteq\A_{\Q}$ is a product of compact spaces
  and is hence compact. I claim that it surjects onto $\A_{\Q}/\Q$. Indeed,
  if $a\in\A_{\Q}$ then for the finitely many prime numbers $p\in S$ such that $a_p\not\in\Z_p$
  we have $a_p\in\frac{r_p}{p^{n_p}}+\Z_p$ with $r_p/p^{n_p}\in\Q$, and
  if $q=\sum_{p\in S}\frac{r_p}{p^{n_p}}\in\Q$ then $a-q\in \prod_p\Z_p\times\R$.
  Now just subtract $\lfloor a_{\infty}-q\rfloor$ to move into $\prod_p\Z_p\times[0,1)$
  and we are done.
\end{proof}

\begin{theorem}
  \lean{NumberField.AdeleRing.cocompact}
  \label{NumberField.AdeleRing.cocompact}
  \uses{Rat.AdeleRing.cocompact,NumberField.AdeleRing.baseChangeEquiv}
  \leanok
  The quotient $\A_K/K$ is compact.
\end{theorem}
\begin{proof}
  We proceed as in the discreteness proof above, by reducing to $\Q$. As before, choosing
  a $\Q$-basis of $K$ gives us $\A_K/K\cong(\A_{\Q}/\Q)^n$ so the result follows from
  the previous theorem.
\end{proof}
