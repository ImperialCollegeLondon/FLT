\chapter{Miniproject: Adeles}\label{Adele_miniproject}

\section{Status}

This is an active miniproject.

\section{The goal}

At the time of writing, mathlib does not have a definition of the adeles of
a number field. There are several goals to this miniproject.

\begin{itemize}
  \item Define the adeles $\A_K$ of a number field~$K$ and
  give them the structure of a $K$-algebra;
  \item Prove that $\A_K$ is a locally compact topological ring;
  \item Show that if $L/K$ is a finite extension of number fields then the
    natural map $L\otimes_K\A_K\to\A_L$ is an isomorphism;
  \item Prove that $K \subseteq \A_K$ is a discrete subgoup and the quotient is compact.
\end{itemize}

We briefly go through the basic definitions. A cheap definition of the finite
adeles $\A_K^\infty$ of $K$ is $K\otimes_{\Z}\Zhat$, where $\Zhat$ is
the profinite completion of the integers. A cheap definition of the infinite adeles
$K_\infty$ of $K$ is $K\otimes_{\Q}\R$, and a cheap definition of the adeles
of $K$ is $\A_K^\infty\times K_\infty$. However in the literature different definitions
are often given. The finite adeles of $K$ are usually defined
as the so-called restricted product $\prod'_{\mathfrak{p}}K_{\mathfrak{p}}$ over the completions
$K_{\mathfrak{p}}$ of $K$ at all maximal ideals $\mathfrak{p}\subseteq\mathcal{O}_K$ of the
integers of $K$. Here the restricted product is the subset of $\prod_{\mathfrak{p}}K_{\mathfrak{p}}$
consisting of elements which are in $\mathcal{O}_{K,\mathfrak{p}}$ for all but finitely many
$\mathfrak{p}$.
Mathlib already has the finite adeles and the proof that they're a topological ring;
in fact the construction in mathlib works for any Dedekind domain.

Similarly the infinite adeles of a number field
are usually defined as $\prod_v K_v$,
the product running over the archimedean completions of~$K$.

\section{Current status}

Salvatore Mercuri has made great progress with the definition of $\A_K$ and he even
has a complete formalisation of the proof that the adele ring is locally compact. His work is in
\href{https://github.com/smmercuri/adele-ring_locally-compact}{his own repo} which
I don't want to have as a dependency of FLT, because this work should all be
in mathlib. We quote some of it here.

\section{Results about adeles of a number field that we await}

Let $K$ be a number field. Right now mathlib has the finite adeles $\A_K^\infty$ of $K$,
with its topological ring structure. It does not however have the infinite adeles $K_\infty$.
There are PRs by Salvatore Mercuri currently
in progress for defining this topological ring. Once we have it, the full ring of
adeles $\A_K$ is defined as $\A_K:=\A_K^\infty\times K_\infty$; it is a topological ring
and also a $K$-algebra. Once Mercuri's mathlib
PR~\href{https://github.com/leanprover-community/mathlib4/pull/16485}{16485} is merged, we will have
all this available to us; until then, we sorry the definitions and these facts.

Mercuri's work isn't however enough for us; we also need several theorems about this topological
$K$-algebra. These lemmas are essentially impossible to work on until we have the definition in
mathlib. We state them here without proof.

\begin{theorem}
  \lean{NumberField.AdeleRing.discrete}
  \label{NumberField.AdeleRing.discrete}
  The additive subgroup $K$ of $\A_K$ is discrete.
\end{theorem}

\begin{theorem}
  \lean{NumberField.AdeleRing.cocompact}
  \label{NumberField.AdeleRing.cocompact}
  The quotient $\A_K/K$ is compact.
\end{theorem}

\begin{theorem}
  \lean{NumberField.AdeleRing.locallyCompactSpace}
  \label{NumberField.AdeleRing.locallyCompact}
  The topological ring $\A_K$ is locally compact.
\end{theorem}

Mercuri has already proved local compactness in his repo.
As far as I know, the other
two theorems remain unformalised (but as I've mentioned, we cannot start on them until
we have a definition of the adele ring in this project, and I would prefer
that this were achieved by upstreaming Mercuri's work to mathlib).

However, there is something that we can do here even before Mercuri's work
is upstreamed: one proof I know that of discreteness
and cocompactness of $K$ in $\A_K$ reduces to the case $K=\Q$, using the ``canonical''
isomorphism $\A_K\cong\A_{\Q}\otimes_{\Q}K$. This can be proved by checking it for finite
and infinite adeles separately, so one thing which we can work on now is the finite case.
We explain this in the next section.

\section{Results about finite adeles which we can work on now}

Adeles are arithmetic objects, attached to global fields like number fields.
However (as I learnt from Maria Ines de Frutos Fernandez) finite adeles are algebraic
objects, as they can be attached to any Dedekind domain. The ``theorem'' that we need
is the following:

\begin{theorem}
  Let~$A$ be a Dedekind domain with field of fractions~$K$, and write $\mathbb{A}_{A,K}^\infty$
  for the finite adeles of $A$. Let~$L/K$ be a finite
  separable extension, and let $B$ be the integral closure of~$A$ in~$L$.
  Then there's a ``canonical'' isomorphism $\mathbb{A}_{R,K}^\infty \otimes_KL\mathbb{A}_{B,L}^\infty$.
\end{theorem}

Of course, this isn't a theorem; this is actually a \emph{definition} (the map) and a theorem about
the definition (that it's an isomorphism). Before we can prove the theorem, we need to make the
definition.

If $A$ is a Dedekind domain then the \emph{height one spectrum} of $A$ is
the nonzero prime ideals of~$A$. Note that because we stick to the literature,
rather than to common sense, fields are Dedekind domains in mathlib, and the
height one spectrum of a field is empty. The reason I don't like allowing fields
to be Dedekind domain is that geometrically the standard definition of Dedekind
domain is ``smooth affine curve, or a point''. But many theorems in algebraic geometry
begin ``let $C$ be a smooth curve''.

There are two steps to the definition of the map; the first is to define it locally
on the individual completions and the second is to glue everything together. Let's
start with the local construction.

Say we're in the AKLB setup above, with $L/K$ a finite separable extension.
Let $w$ be a finite place of $L$ lying above a finite place $v$ of $K$.
We put the $w$-adic topology on $L$ and the $v$-adic topology on~$K$. We now claim
that inclusion $i:K\to L$ is continuous with respect to these topologies. This
claim follows from

\begin{lemma} If $i:K\to L$ denotes the inclusion then $e*w(i(k))=v(k)$, where
  $e$ is the ramification index of $w/v$.
  \label{IsDedekindDomain.HeightOneSpectrum.valuation_comap}
  \lean{IsDedekindDomain.HeightOneSpectrum.valuation_comap}
\end{lemma}
\begin{proof}
  \leanok
  Standard.
\end{proof}

\begin{definition}
  \lean{IsDedekindDomain.HeightOneSpectrum.adicCompletion_comap_algHom}
  \label{IsDedekindDomain.HeightOneSpectrum.adicCompletion_comap_algHom}
  \uses{IsDedekindDomain.HeightOneSpectrum.valuation_comap}
  There's a natural continuous $K$-algebra homomorphism map $K_v\to L_w$. It is defined by completing
  the inclusion $K\to L$ at the finite places $v$ and $w$, which can be done
  by the previous lemma.
\end{definition}

Note: the formalization right now only claims that the map is a $K$-algebra homomorphism,
not the continuity. Do we have continuous $K$-algebra maps?

We can take the product of all of these maps.

\begin{definition}
  \lean{DedekindDomain.ProdAdicCompletions.baseChange}
  \label{DedekindDomain.ProdAdicCompletions.baseChange}
  \uses{IsDedekindDomain.HeightOneSpectrum.adicCompletion_comap_algHom}
  There's a natural $K$-algebra homomorphism $\prod_v K_v\to\prod_w L_w$, where the
  products run over the height one spectrums of $A$ and $B$ respectively.
\end{definition}

One then has to check that for good primes everything works on an integral level,

\begin{definition}
  The map above induces a natural $K$-algebra homomorphism $\A_K^\infty\to\A_L^\infty$
  at the level of finite adeles.
\end{definition}

Note that the finite adeles do not have the subspace topology, so one has to be slightly
careful here.
