\chapter{An overview of the proof}\label{ch_overview}

So far we have seen that, modulo Mazur's theorem (and various other things which will still take some work to formalise but which are much easier), Fermat's Last Theorem can be reduced
to the statement that there is no prime $\ell\geq 5$ and hardly-ramified
irreducible 2-dimensional Galois representation $\rho:\GQ\to\GL_2(\Z/\ell\Z)$.

In this chapter we give an overview of our strategy for proving this, and collect
various results which we will need along the way. Note that we no longer need to assume that $\rho$ comes from the $\ell$-torsion in an elliptic curve.

\section{Potential modularity.}

We will only speak about modularity for 2-dimensional representations of the
absolute Galois group of a totally real field $F$ of even degree over $\Q$, and
what we mean by the term is "associated to an automorphic representation of the
units of the totally definite quaternion algebra over $F$ ramified at no finite places".
We can furthermore even demand that the infinity type is trivial, as these are the only
forms we shall need for FLT.

Assume we have a hardly-ramified representation~$\rho$ as above. Let $K$
be the number field corresponding to the kernel of~$\rho$. Our first claim
is that there is some totally real field $F$ of even degree, Galois over $\Q$,
unramified at $\ell$, and disjoint from $K$, such that $\rho|G_F$ is modular. 
The proof of this is very long, and uses a host of machinery. For example:
\begin{itemize}
    \item Moret--Bailly's result~\cite{moret-bailly} on points on curves with prescribed 
    local behaviour;
    \item several nontrivial results in global class field theory;
    \item the Jacquet--Langlands correspondence;
    \item The assertion that irreducible 2-dimensional mod $p$ representations induced from a character are modular (this follows from converse theorems);
    \item A modularity lifting theorem.
\end{itemize}

Everything here is from the 20th century, and most of it is from the 1980s or before.
The exception is the modularity lifting theorem, which we state explicitly here.

**TODO** state modularity lifting theorem.

I am not entirely sure where to find a proof of this in the literature, although it has certainly been known to the experts for some time. Theorem 3.3 of~\cite{taylor-mero-cont} comes close, although it assumes that $\ell$ is totally split in $F$ rather than just unramified. Another near-reference is Theorem~5.2 of~\cite{toby-modularity}, although this assumes
the slightly stronger assumption that the image of $\rho$ contains $\SL_2(\Z/p\Z)$ (however it is well-known to the experts that this can be weakened to give the result we need). One reference for the proof is \href{https://math.berkeley.edu/~fengt/249A_2018.pdf}{Richard Taylor's 2018 Stanford course}. 

Given the modularity lifting theorem, the strategy to show potential modularity of $\rho$ is to use Moret--Bailly to find an appropriate totally real field $F$, an auxilary prime $p$, and an auxiliary elliptic curve over $F$ whose mod $\ell$ Galois representation is $\rho$ and whose
mod $p$ Galois representation is induced from a character. By converse theorems (for example)
the mod $p$ Galois representation is associated to an automorphic representation of
$\GL_2/F$ and hence by Jacquet--Langlands it is modular. Now we use the
modularity lifting theorem to deduce the modularity of the curve over $F$ and hence
the modularity of the $\ell$-torsion. 

\section{Compatible families, and reduction at 3}

We now use Khare--Wintenberger to lift $\rho$ to a potentially modular $\ell$-adic
Galois representation of conductor 2, and put it into an $\ell$-adic famiily using
the Brauer's theorem trick in \cite{blggt}. Finally we look at the 3-adic specialisation
of this family. Reducing mod 3 we get a representation which is flat at 3 and tame at 2,
so must be reducible because
of the techniques introduced in Fontaine's paper on abelian varieties over $\Z$ (an irreducible
representation would cut out a number field whose discriminant violates the Odlyzko bounds).
One can now go on to deduce that the 3-adic representation must be reducible, which
contradicts the irreducibility of $\rho$.

As this document grows, we will add a far more detailed discussion of
what is going on here. 






