\chapter{Miniproject: Fujisaki's Lemma}\label{Fujisaki_project}

\section{The goal}

There is an idelic compactness statement which encapsulates both finiteness of the class
group of a number field and Dirichlet's units theorem about the rank of the unit group.
In fact there is even a noncommutative version of this statement. In John Voight's
book~\cite{voightbook} this is Main Theorem 27.6.14(a) and Voight calls it Fujisaki’s lemma.
I know nothing of the history but I'm happy to adopt this name. In the quaternion algebra
miniproject we will use this compactness result to prove finite-dimensionality of a
space of quaternionic modular forms.

\section{Initial definitions}

Let $K$ be a field. A \emph{central simple $K$-algebra} is a $K$-algebra~$B$ (not necessarily
commutative) with centre $K$ such that $B$ has exactly two two-sided ideals, namely ${0}$ and $B$
(or $\bot$ and $\top$, as Lean would call them). We will be concerned
only with central simple $K$-algebras which are finite-dimensional as $K$-vector spaces, and
when $K$ is clear we will just refer to them as central simple algebras. We remark that a
4-dimensional central simple algebra is called a \emph{quaternion algebra}; we will have
more to say about these later on.

Matrix algebras $M_n(K)$ are examples of finite-dimensional central simple $K$-algebras.
If $K=\bbC$ (or more generally if $K$ is algebraically closed)
then matrix algebras are the only finite-dimensional examples
up to isomorphism. There are other examples over the reals: for example Hamilton's quaternions
$\bbH:=\R\oplus\R i\oplus\R j\oplus\R k$ with the usual rules $i^2=j^2=k^2=-1$,
$ij=-ji=k$ etc, are an example of a central simple $\R$-algebra (and a quaternion algebra), and
matrix algebras over $\bbH$ are other central simple $\R$-algebras.
For a general field $K$
one can make an analogue of Hamilton's quaternions $K\oplus Ki\oplus Kj\oplus Kk$ with the
same multiplication rules ($i^2=-1$ and so on) to describe the multiplication, and if the characteristic
of~$K$ isn't 2
then this is a quaternion algebra (which may or may not be isomorphic to $M_2(K)$ in this
generality).

Some central simple algebras~$B$ are \emph{division algebras}, meaning that they are division
rings, or equivalently that every nonzero $b\in B$ has a two-sided inverse. For example
Hamilton's quaternions are a division algebra over $\R$,
because $(x+yi+zj+tk)(x-yi-zj-tk)=x^2+y^2+z^2+t^2$, so the inverse
of a nonzero $x+yi+zj+tk$ is $(x-yi-zj-tk)/(x^2+y^2+z^2+t^2)$.
However $2\times 2$ matrices over a field~$K$, whilst being a central simple algebra
over~$K$, are never a division algebra
(even if $K=\bbC$) because a nonzero matrix with determinant zero such as
$\begin{pmatrix}1&0\\0&0\end{pmatrix}$ has no inverse.

\section{Enter the adeles}

The adeles of a number field are discussed in far more detail
in the adele miniproject \ref{Adele_miniproject}. We just recall here that if $K$ is a number field
then there are two huge commutative topological $K$-algebras called the \emph{finite adeles}
$\A_K^\infty$ and the \emph{adeles} $\A_K$ of $K$, and that they're both locally compact
as topological spaces. We also know from theorem~\ref{NumberField.AdeleRing.baseChangeEquiv}
that $\A_K\cong K\otimes_{\Q}\A_{\Q}K$ (both topologically and algebraically), meaning
that if $R$ is a $K$-algebra then $R_{\A} := R\otimes_K\A_K$ is naturally isomorphic
to $R\otimes_{\Q}\A_{\Q}$. One can
furthermore check that if $R$ is a finite $K$-algebra then the $\A_K$-module topologies and $\A_{\Q}$-module
topologies on $R_{\A}$ coincide. Indeed, the topology on $\A_K$
is the $\A_{\Q}$-module topology, as
$\A_K=\A_{\Q}\otimes_{\Q}K$ as topological $\A_{\Q}$-algebras, where the right hand side
has the $\A_{\Q}$-module topology by definition. So the claim follows from the
fact that if $A$ is a topological ring, $B$ is a topological $A$-algebra
finite as an $A$-module and with the $A$-module topology, and if
$M$ is a topological $B$-module
(and hence a topological $A$-module), then the $A$-module and $B$-module
topologies on~$M$ coincide (this is {\tt moduleTopology.trans} in the repo,
not yet PRed to mathlib).

Let $K$ be a number field and let $D/K$ be a finite-dimensional central simple $K$-algebra
(later on $D$ will be a division algebra (hence the name) but we do not need this yet).
Then $D_{\A}:=D\otimes_K\A_K$ is an $\A_K$-algebra which
is free of finite rank, and if we give $D_{\A}$ the $\A_K$-module topology then it is
a topological ring (by results in mathlib). Furthermore $D_{\A}$ is free of finite
rank over the locally compact topological ring $\A_K$ and is thus also
locally compact. So by the theory of Haar characters (see Chapter~\ref{Haar_char_project})
there is a canonical character $\delta_{D_{\A}}:D_{\A}^\times\to\R_{>0}$ measuring
how left multiplication by an element of $D_{\A}^\times$ changes the additive Haar
measure on $D_{\A}$. Let $D_{\A}^{(1)}$ denote the kernel of $\delta_{D_{\A}}$,
and give it the subspace topology coming from $D_{\A}^\times$.
Corollary~\ref{NumberField.AdeleRing.units_mem_ringHaarCharacter_ker} from the
Haar character miniproject shows that $D^\times$ (regarded as a subgroup of $D_{\A}^\times$
via the map $d\mapsto d\otimes 1$) is contained within $D_{\A}^{(1)}$,
thus the below theorem typechecks.

\begin{theorem}
  \label{NumberField.AdeleRing.DivisionAlgebra.compact_quotient}
  \lean{NumberField.AdeleRing.DivisionAlgebra.compact_quotient}
  \leanok
  If $D$ is a division algebra then
  the quotient $D^\times\backslash D_{\A}^{(1)}$
  with its quotient topology coming from $D_{\A}^{(1)}$, is compact.
\end{theorem}

The rest of this miniproject is devoted to a proof of this theorem.

\section{The proof}

We prove the theorem via a series of lemmas.

\begin{lemma}
  \label{NumberField.AdeleRing.DivisionAlgebra.Aux.existsE}
  \lean{NumberField.AdeleRing.DivisionAlgebra.Aux.existsE}
  \leanok
  There's a compact subset $E$ of $D_{\A}$
  with the property that for all $x\in D_{\A}^{(1)}$,
  the obvious map $xE\to D\backslash D_{\A}$ is not injective.
\end{lemma}

\begin{proof} We know that if we pick a $\Q$-basis for $D$
  of size $d$ then this identifies $D$ with $\Q^d$,
  $D_{\A}$ with $\A_{\Q}^d$, and $D\backslash D_{\A}$ with
  $(\Q\backslash\A_{\Q})^d$. Now $\Q$ is discrete in $\A_{\Q}$
  by theorem~\ref{NumberField.AdeleRing.discrete}, and the quotient
  $\Q\backslash \A_{\Q}$ is compact by theorem~\ref{Rat.AdeleRing.cocompact}.
  Hence $D$ is discrete in $D_{\A}$
  and the quotient $D\backslash D_{\A}$ is compact.

    Fix a Haar measure $\mu$ on $D_{\A}$ and push it forward
    to $D\backslash D_{\A}$; by compactness this quotient has finite
    and positive measure, say $m\in\R_{>0}$.
    Choose any compact $E\subseteq D_{\A}$ with measure $> m$
    (for example, choose a $\Z$-lattice $L\cong\Z^d$ in $D\cong\Q^d$,
    define $E_f:=\prod_p L_p\in D\otimes_{\Q}\A_{\Q}^\infty$,
    and define $E_{\infty}\subseteq D\otimes_{\Q}\R\cong\R^n$ to be a huge closed
    ball, large enough to ensure the measure of $E:=E_f\times E_{\infty}$ is bigger than $m$).
    Then $\mu(xE)=\mu(E)>m$ so the map can't be injective.
\end{proof}

\begin{definition}
  \label{NumberField.AdeleRing.DivisionAlgebra.Aux.E}
  \lean{NumberField.AdeleRing.DivisionAlgebra.Aux.E}
  \uses{NumberField.AdeleRing.DivisionAlgebra.Aux.existsE}
  \leanok
  We let $E$ denote any compact set satisfying the hypothesis of the previous lemma.
\end{definition}

\begin{definition}
  \label{NumberField.AdeleRing.DivisionAlgebra.Aux.X}
  \lean{NumberField.AdeleRing.DivisionAlgebra.Aux.X}
  \uses{NumberField.AdeleRing.DivisionAlgebra.Aux.E}
  \leanok
Define $X:=E-E:=\{e-f:e,f\in E\}\subseteq D_{\A}$.
\end{definition}

\begin{definition}
  \label{NumberField.AdeleRing.DivisionAlgebra.Aux.Y}
  \lean{NumberField.AdeleRing.DivisionAlgebra.Aux.Y}
  \uses{NumberField.AdeleRing.DivisionAlgebra.Aux.X}
  \leanok
Define $Y:=X.X:=\{xy:x,y\in X\}\subseteq D_{\A}$.
\end{definition}

\begin{lemma}
  \label{NumberField.AdeleRing.DivisionAlgebra.Aux.X_compact}
  \lean{NumberField.AdeleRing.DivisionAlgebra.Aux.X_compact}
  \uses{NumberField.AdeleRing.DivisionAlgebra.Aux.X}
  \leanok
  $X$ is a compact subset of $D_{\A}$.
\end{lemma}
\begin{proof}
  \leanok
  It's the continuous image of the compact set~$E\times E$.
\end{proof}

\begin{lemma}
  \label{NumberField.AdeleRing.DivisionAlgebra.Aux.Y_compact}
  \lean{NumberField.AdeleRing.DivisionAlgebra.Aux.Y_compact}
  \uses{NumberField.AdeleRing.DivisionAlgebra.Aux.Y}
  \leanok
  $Y$ is a compact subset of $D_{\A}$.
\end{lemma}
\begin{proof}
  \leanok
  It's the continuous image of the compact set~$X\times X$.
\end{proof}

\begin{lemma}
  \label{NumberField.AdeleRing.DivisionAlgebra.Aux.X_meets_kernel}
  \lean{NumberField.AdeleRing.DivisionAlgebra.Aux.X_meets_kernel}
  \uses{NumberField.AdeleRing.DivisionAlgebra.Aux.X, NumberField.AdeleRing.DivisionAlgebra.Aux.E}
  % TODO I used E in the statement of the lemma here
  \leanok
  If $\beta\in D_{\A}^{(1)}$ then
$\beta X\cap D^\times\not=\emptyset$.
\end{lemma}
\begin{proof}
  \leanok
Indeed by lemma~\ref{NumberField.AdeleRing.DivisionAlgebra.Aux.existsE},
the map $\beta E\to D\backslash D_{\A}$
isn't injective, so there are distinct
$\beta e_1,\beta e_2\in \beta E$ with $e_i\in E$ and
$\beta e_1-\beta e_2=b\in D$.
Now $b\not=0$ and $D$ is a division algebra, so $b\in D^\times$.
And $e_1-e_2\in X$ so $b=\beta(e_1-e_2)\in \beta X$, so we're done.
\end{proof}

\begin{lemma}
  \label{NumberField.AdeleRing.DivisionAlgebra.Aux.X_meets_kernel'}
  \lean{NumberField.AdeleRing.DivisionAlgebra.Aux.X_meets_kernel'}
  \uses{NumberField.AdeleRing.DivisionAlgebra.Aux.X}
  \leanok
  Similarly, if $\beta\in D_{\A}^{(1)}$ then
  $X\beta^{-1}\cap D^\times\not=\emptyset$.
\end{lemma}
\begin{proof}
  \leanok
  % TODO does this make the lines dottted? I used E in the proof here
  % TODO does \uses a theorem from another project mess up the colouring?
  \uses{NumberField.AdeleRing.DivisionAlgebra.Aux.E, addHaarScalarFactor.left_mul_eq_right_mul}
  Indeed, $\beta^{-1}\in D_{\A}^{(1)}$, and so left multiplication by $\beta^{-1}$
  doesn't change Haar measure on $D_{\A}$, so neither does right multiplication
  (by theorem~\ref{addHaarScalarFactor.left_mul_eq_right_mul}).
  So the same argument works: $E\beta^{-1}\to D\backslash D_{\A}$ is not
  injective so choose $e_1\beta^{-1}\not=e_2\beta^{-1}$ with difference $b\in D$
  and then $(e_1-e_2)\beta^{-1}\in D-{0}=D^\times$.
\end{proof}

\begin{definition}
  \label{NumberField.AdeleRing.DivisionAlgebra.Aux.T}
  \lean{NumberField.AdeleRing.DivisionAlgebra.Aux.T}
  \uses{NumberField.AdeleRing.DivisionAlgebra.Aux.Y}
  \leanok
  Let $T:=Y\cap D^\times$.
\end{definition}

\begin{lemma}
  \label{NumberField.AdeleRing.DivisionAlgebra.Aux.T_finite}
  \lean{NumberField.AdeleRing.DivisionAlgebra.Aux.T_finite}
  \uses{NumberField.AdeleRing.DivisionAlgebra.Aux.T}
  \leanok
  $T$ is finite.
\end{lemma}
\begin{proof}
  \uses{NumberField.AdeleRing.DivisionAlgebra.Aux.Y_compact}
  % TODO I used T in both the theorem and proof this time!
  It suffices to prove that $Y\cap D$ is finite.
    But $D\subseteq D_{\A}$ is a discrete additive subgroup, and hence closed.
    And $Y\subseteq D_{\A}$ is compact.
    So $D\cap Y$ is compact and discrete, so finite.
\end{proof}

\begin{definition}
  \label{NumberField.AdeleRing.DivisionAlgebra.Aux.C}
  \lean{NumberField.AdeleRing.DivisionAlgebra.Aux.C}
  \uses{NumberField.AdeleRing.DivisionAlgebra.Aux.T,
    NumberField.AdeleRing.DivisionAlgebra.Aux.X}
  \leanok
  Define $C:= (T^{-1}.X) \times X\subset D_{\A}\times D_{\A}$.
\end{definition}

\begin{lemma}
  \label{NumberField.AdeleRing.DivisionAlgebra.Aux.C_compact}
  \lean{NumberField.AdeleRing.DivisionAlgebra.Aux.C_compact}
  \uses{NumberField.AdeleRing.DivisionAlgebra.Aux.C,
    NumberField.AdeleRing.DivisionAlgebra.Aux.T_finite,
    NumberField.AdeleRing.DivisionAlgebra.Aux.X_compact}
  \leanok
  $C$ is compact.
\end{lemma}
\begin{proof}
  \leanok
   $X$ is compact and $T$ is finite.
\end{proof}

\begin{lemma}
  \label{NumberField.AdeleRing.DivisionAlgebra.Aux.antidiag_mem_C}
  \lean{NumberField.AdeleRing.DivisionAlgebra.Aux.antidiag_mem_C}
  \uses{NumberField.AdeleRing.DivisionAlgebra.Aux.C}
  \leanok
  For every $\beta\in D_{\A}^{(1)}$, there exists $b\in D^\times$
  and $\nu\in D_{\A}^{(1)}$ such that $\beta=b\nu$ and $(\nu,\nu^{-1})\in C.$
\end{lemma}
\begin{proof}
  \uses{NumberField.AdeleRing.DivisionAlgebra.Aux.E,
    NumberField.AdeleRing.DivisionAlgebra.Aux.X_meets_kernel,
    NumberField.AdeleRing.DivisionAlgebra.Aux.X_meets_kernel',
    NumberField.AdeleRing.DivisionAlgebra.Aux.C}
  By lemma~\ref{NumberField.AdeleRing.DivisionAlgebra.Aux.X_meets_kernel},
  $\beta X\cap D^\times\not=\emptyset$,
  and lemma~\ref{NumberField.AdeleRing.DivisionAlgebra.Aux.X_meets_kernel'},
  $X\beta^{-1}\cap D^\times\not=\emptyset$,
  so we can write $\beta x_1=b_1$ and $x_2\beta^{-1}=b_2$ with $b_i\in D^\times$ and $x_i\in X$.
  Note that $\beta\in D_{\A}^{(1)}$ and $b_i\in D^{\times}\subseteq D_{\A}^{(1)}$ by
  corollary~\ref{NumberField.AdeleRing.units_mem_ringHaarCharacter_ker}, so $x_i\in D_{\A}^{(1)}$
  as well. In particular $x_i\in D_{\A}^\times$ so $x_1^{-1}$ makes sense.

  Multiplying the equations defining the $x_i$ and $b_i$ we deduce that
  $x_2x_1=b_2b_1\in Y\cap D^\times=T$ (recall that $Y=X.X$ and $T=Y\cap D^\times$
  is finite); call this element $t$. Then $x_1^{-1}=t^{-1}x_2\in T^{-1}.X$,
  and $x_1\in X$, so if we set $\nu=x_1^{-1}\in D_{\A}^{(1)}$
  and $b=b_1\in D^\times$ then we have $\beta=b\nu$ and $(\nu,\nu^{-1})\in C := (T^{-1}.X)\times X$.
  We are done!
\end{proof}

We can now prove Fujisaki's theorem~\ref{NumberField.AdeleRing.DivisionAlgebra.compact_quotient}.

\begin{proof}
  \proves{NumberField.AdeleRing.DivisionAlgebra.compact_quotient}
  \uses{MeasureTheory.ringHaarChar_continuous,
    NumberField.AdeleRing.DivisionAlgebra.Aux.antidiag_mem_C,
    NumberField.AdeleRing.DivisionAlgebra.Aux.C_compact}
  Indeed, if $M$ is the preimage of $C$ under the inclusion $D_{\A}^{(1)} \to D_{\A}\times D_{\A}$
  sending $\nu$ to $(\nu,\nu^{-1})$, then $M$ is a closed subspace
    of a compact
  space so it's compact (note that $\delta_{D_{\A}}$ is continuous,
  by theorem~\ref{MeasureTheory.ringHaarChar_continuous}, so $D_{\A}^{(1)}$ is a closed subset of
  $D_{\A}^\times$ which is itself a closed subset of $D_{\A}\times D_{\A}$).
  Lemma~\ref{NumberField.AdeleRing.DivisionAlgebra.Aux.antidiag_mem_C} shows that $M$ surjects onto
  $D^\times\backslash D_{\A}^{(1)}$ which is thus also compact.
\end{proof}

We note here some useful consequences.

\begin{theorem}
  \label{NumberField.FiniteAdeleRing.DivisionAlgebra.units_cocompact}
  \lean{NumberField.FiniteAdeleRing.DivisionAlgebra.units_cocompact}
  \leanok
  $D^\times\backslash(D\otimes_K\A_K^\infty)^\times$ is compact.
\end{theorem}
\begin{proof}
  \uses{NumberField.AdeleRing.DivisionAlgebra.compact_quotient}
There's a natural map $\alpha$ from $D^\times\backslash D_{\A}^{(1)}$ to
  $D^\times\backslash (D\otimes_K \A_K^\infty)^\times$. We claim that it's
  surjective. Granted this claim, we are home, because if we put the quotient
  topology on $D^\times\backslash (D\otimes_K \A_K^\infty)^\times$ coming from
  $(D\otimes_K \A_K^\infty)^\times$ then it's readily verified that $\alpha$
  is continuous, and the continuous image of a compact space is compact.

  As for surjectivity: say $x\in (D\otimes_K \A_K^\infty)^\times$. We need to extend
  $x$ to an element $(x,y)\in (D\otimes_K \A_K^\infty)^\times\times(D\otimes_K K_\infty)^\times$
  which is in the kernel of $\delta_{D_{\A}}$. Because $\delta_{D_{\A}}(x,1)$ is some positive
  real number, it will suffice to show that if $r$ is any positive real number then we can
  find $y\in (D\otimes_K \A_K^\infty)^\times=(D\otimes_{\Q}\R)^\times$ with $\delta_{D_{\A}}(1,y)=r$,
  or equivalently (setting $D_{\R}=D\otimes_{\Q}\R$) that $\delta_{D_{\R}}(y)=r$.
  But $D\not=0$ as it is a division algebra,and hence $\Q\subseteq D$, meaning
  $\R\subseteq D_{\R}$, and if
  $x\in\R^\times\subseteq D_{\R}^\times$ then $\delta(x)=|x|^d$ with $d=\dim_{\Q}(D)$,
  as multiplication by $x$ is just scaling by a factor of $x$ on $D_{\R}\cong\R^d$.
  In particular we can set $x=y^{1/d}$.
\end{proof}
\begin{remark} In this generality the quotient might not be Hausdorff.
\end{remark}

\begin{theorem}
  \label{NumberField.FiniteAdeleRing.DivisionAlgebra.finiteDoubleCoset}
  \lean{NumberField.FiniteAdeleRing.DivisionAlgebra.finiteDoubleCoset}
  \uses{NumberField.FiniteAdeleRing.DivisionAlgebra.units_cocompact}
  \leanok
  If $U$ is an open subgroup of $(D\otimes_K \A_K^\infty)^\times$
  then the double coset space $D^\times\backslash(D\otimes_K \A_K^\infty)^\times/U$ is finite.
\end{theorem}
\begin{proof} The double cosets give a disjoint open cover of $(D\otimes_K \A_K^\infty)$
  which descends to a disjoint open cover of the quotient space
  $D^\times\backslash(D\otimes_K \A_K^\infty)^\times$. However this space is compact
  by theorem~\ref{NumberField.FiniteAdeleRing.DivisionAlgebra.units_cocompact}.
\end{proof}
