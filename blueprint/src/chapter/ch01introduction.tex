\chapter{Introduction}

Fermat's Last Theorem is the statement that if $a,b,c,n$ are positive whole numbers with $n\geq 3$,
then $a^n+b^n\not=c^n$. It is thus the claim that a family of \emph{Diophantine equations}
($a^3+b^3=c^3, a^4+b^4=c^4,\ldots$) has no positive integer solutions.
Diophantus was a Greek mathematician who lived around 1800
years ago, and he would have been able to understand the statement of the theorem (he knew about
positive integers, addition and multiplication).

Fermat's Last Theorem was explicitly raised by Fermat in 1637, and was proved by Wiles (with the
proof completed in joint work with Taylor) in 1994. There are now several proofs but all of them
go broadly in the same direction, using elliptic curves and modular forms.

Lean is an interactive theorem prover; it checks mathematical arguments with super-human accuracy.
Explaining a proof of Fermat's Last Theorem to Lean is in some sense like explaining the proof to
Diophantus; for example, the proof starts by observing that before we go any further it's convenient
to first invent/discover zero and negative numbers, and one can point explicitly at places in Lean's
source code \href{https://github.com/leanprover/lean4/blob/260eaebf4e804c9ac1319532970544a4e157c336/src/Init/Prelude.lean#L1049}{here}
and \href{https://github.com/leanprover/lean4/blob/260eaebf4e804c9ac1319532970544a4e157c336/src/Init/Data/Int/Basic.lean#L45}{here}
where these things happen. However we will adopt a more efficient approach: we will assume all of
the theorems both in core Lean and in its mathematics library
\href{https://github.com/leanprover-community/mathlib4}{\tt mathlib}, and proceed from there.
To give some idea of what this entails: {\tt mathlib} at the time of writing contains most of an
undergraduate mathematics degree and parts of several relevant Masters level courses (for example,
the definitions and basic properties of
\href{https://leanprover-community.github.io/mathlib4_docs/Mathlib/AlgebraicGeometry/EllipticCurve/Weierstrass.html}{elliptic curves}
and \href{https://leanprover-community.github.io/mathlib4_docs/Mathlib/NumberTheory/ModularForms/Basic.html}{modular forms}
are in {\tt mathlib}). Thus our task can be likened to teaching a graduate level course on
Fermat's Last Theorem to a computer. The computer is quite a challenging audience member -- it
will insist on being given all technical details of all arguments, and it will not accept proof by
intimidation or by appeal to higher authority. Most mathematicians know humans who also behave
in this manner. However, it is worse than this; in 2025 at least, the computer will only start filling
in details of arguments by itself once the arguments are mathematically utterly obvious.
Thus, currently, formalization can be a very time-consuming process.

\section{Which proof is being formalised?}

At the time of writing, these notes do not contain anywhere near a proof of FLT, or even a sketch proof,
although we are currently actively working on fixing this.

From 2024 to 2029 we will be building a proof of FLT, following a strategy
constructed by Taylor, taking into account Buzzard's comments on what would be easy or hard to do
in Lean. Our proof uses refinements of the original Taylor--Wiles method by Diamond/Fujiwara,
Khare--Wintenberger, Skinner--Wiles, Kisin, Taylor and others -- one could call it a 21st century
proof of the theorem. During this initial phase of the project, we shall also be \emph{assuming}
many nontrivial theorems without proof, as long as they were published by 31st December 1989.
To get technical for just a second -- we shall for example be assuming the existence of Galois
representations attached to weight~2 Hilbert modular forms, we will assume Langlands' cyclic base
change theorem for $\GL_2$, Mazur's theorem bounding the torsion subgroup of an elliptic curve over
the rationals, and several other nontrivial results which were known by the end of the 1980s.

The upshot of this is that, by 2029, this project should contain a complete proof that FLT
follows from results that were known to humanity in the 1980s. In particular, one naive
way of understanding the goal is that it is a ``formalization of the papers of Wiles and Taylor--Wiles,
assuming the results in the references of those papers''. However, as noted above, we will actually
be taking a slightly different path.

\section{The structure of this blueprint}

This blueprint is a \emph{nonlinear} document, comprising many chapters. The chapters
are not designed to be read in order. Each chapter is self-contained and has a well-defined goal,
stated at the top of the chapter.

After this chapter, you should next read chapter~\ref{ch_reductions}, which explains how to reduce FLT
to two highly nontrivial statements about the $p$-torsion in a certain elliptic curve (the Frey curve).
One of these statements was proved in the 1970s by Mazur, and we shall not be concentrating on
it until the end of the first phase. The other is a theorem of Wiles, and this is what we will
be concentrating on in the remainder of the blueprint.

\section{Remarks}

The actual blueprint currently also contains a lot of disorganised ideas. Currently these
should be disregarded.

For example there is chapter~\ref{ch_frey}, which goes into more details about elliptic curves.
Chapter~\ref{ch_overview} is an \emph{extremely} sketchy overview of how
the rest of the proof goes. Neither of these should be read right now.

All of the remaining chapters are experiments, and most of them are what I am currently
calling ``mini-projects''. A mini-project is a bottom-up project, typically at early graduate
student level, with a concrete goal. The ultimate goal of many of these projects is to actually
get some result into mathlib. We have had one success so far -- the Frobenius mini-project
is currently being PRed to mathlib by Thomas Browning. Currently most of my efforts are
going into running mini-projects, with the two most active ones currently being the adeles
mini-project and the quaternion algebra mini-project. These projects do not logically depend
on each other for the most part, and one can pick and choose how one reads them.

There is also an appendix, which is again very sketchy, and comprises mostly of a big
list of nontrivial theorems many of which we will be assuming without proof in the FLT
project.

The next chapter to read, where the proof begins, is chapter~\ref{ch_reductions}.
