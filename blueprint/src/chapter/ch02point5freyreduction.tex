\chapter{Proving irreducibility}\label{ch_freyreduction}

\section{Goal}

In this chapter, we reduce the claim that the $p$-torsion in the Frey
curve is reducible to three hard theorems, two of which were not known
in the 1980s.

\section{Overview}

In Chapter~\ref{ch_reductions} we reduced FLT to the
following two claims: if $\rho$ is the $p$-torsion in the Frey curve
associated to a Frey package $(a,b,c,p)$, then $\rho$ is both
irreducible (by Mazur) and reducible (by Wiles). Mazur's 1977 theorem
is currently out of scope for this project, as it is ``too old''.
We thus focus on how to prove Wiles' theorem, which was still an open problem
in the 1980s (although it had been conjectured by Serre in~\cite{serreconj}
in 1987). In this section we state three difficult theorems and
show how Wiles' theorem follows easily from them.

\section{Facts about the Frey curve}

Let $(a,b,c,p)$ be a Frey package (so in particular $p\geq5$ is prime and $a^p+b^p=c^p$),
let $E$ be the corresponding Frey curve over $\Q$, and let $\rho:\GQ\to\Aut(E(\Qbar)[p])$
be the 2-dimensional Galois representation on the $p$-torsion of~$E$. Recall that our goal
is to prove that $\rho$ is reducible.

What we need to leverage is the fact that $\rho$ has very little ramification. To give
a toy example before we start: if $K$ is a number field (i.e., a finite extension of $\Q$)
and if the extension $K/\Q$ is unramified at all primes, then an old theorem of
Minkowski tells us that $K=\Q$. We want to prove a theorem in the same vein; if
a 2-dimensional Galois representation is hardly-ramified, then it is reducible.
Below, we give a precise
definition of what it means for a 2-dimensional representation $\GQ\to\GL_2(R)$
to be \emph{hardly ramified}. Before that, we need to say precisely which rings~$R$
we will allow. We recall the following definition, due to Grothendieck:

\begin{definition} A commutative topological ring~$R$ is \emph{pseudocompact} if it satisfies
  the following properties:
  \begin{itemize}
    \item If $U\subseteq R$ is any open set containing 0, then there's an open ideal $I\subseteq U$.
    \item If $I\subseteq R$ is open, then $R/I$ is Artinian.
    \item The natural map $R\to\lim R/I$ is a topological and algebraic isomorphism, where $I$ runs
      over the open ideals of $R$. Here $R/I$ has the discrete topology and the projective
      limit has the projective limit topology.
  \end{itemize}
\end{definition}

\begin{definition} A \emph{coefficient ring} is a local pseudocompact topological ring $R$ with
  finite residue field.
\end{definition}

Examples of coefficient rings include finite fields, and integer rings of finite extensions
of $\Q_p$. There are also non-Noetherian examples, for example the projective limit of
the rings $\Z/p\Z[\eps_1,\ldots,\eps_n]/(\forall i,j,\eps_i\eps_j=0)$
which are convenient to include
for technical reasons. If $R$ is a coefficient ring with maximal ideal $\m$
and residue field $k$ of characteristic~$\ell$, then for any open ideal~$I$ there is
a ring homomorphism $\Z/\ell^n\Z\to R/I$ for all sufficiently large naturals $n$,
and thus a continuous ring homomorphism $\Z_{\ell}\to R$. The $\ell$-adic cyclotomic character
is a continuous
representation $\GQ\to\Z_{\ell}^\times$ and it thus induces a continuous
representation $\GQ\to R^\times$ which we also refer to as the cyclotomic character.

\begin{definition}
  \label{hardly_ramified}
  Let $R$ be a coefficient ring with finite residue field of characteristic $\ell\geq3$.
  Let $V$ be a finite free $R$-module of rank~2, equipped with the product topology. A
  continuous representation $\rho: \GQ\to \GL_R(V)$ is said to be \emph{hardly ramified} if it
  satisfies the following four conditions:
  \begin{enumerate}
  \item $\det(\rho):\GQ\to R^\times$ is the mod $\ell$ cyclotomic character;
  \item $\rho$ is unramified outside $2\ell$;
  \item There is a short exact sequence $0\to R\to V\to R\to 0$ restriction of $\rho$ to $\Gal(\Qbar_2/\Q)$ is reducible and has a
    1-dimensional quotient which is an unramified representation whose square is trivial;
  \item The restriction of $\rho$ to $\GQl$ comes from a finite flat group scheme.
  \end{enumerate}
\end{definition}

A well-known result, which basically goes back to Frey, is the following:

\begin{theorem}
  \label{Frey_curve_hardly_ramified}
  The $\ell$-torsion $\rho:\GQ\to\GL_2(\Z/\ell\Z)$ in the Frey curve associated to a Frey
  package $(a,b,c,\ell)$ is hardly ramified.
\end{theorem}
\begin{proof}
  First note that $\ell\geq5>3$ by definition of a Frey package. Let $\rho$
  denote the Galois representation on the $\ell$-torsion of the Frey curve.
  The fact that $\rho$ is 2-dimensional is~Corollary
  III.6.4(b) of~\cite{silverman1}, and the fact that its determinant is
  cyclotomic is Proposition~III.8.3. These results hold for elliptic curves
  in general. The remaining claims are specific to the Frey curve and lie
  deeper. The fact that $\rho$ is unramified outside $2\ell$ is a consequence
  of (4.1.12) and (4.1.13) of~\cite{serreconj}. The fact that $\rho$ at 2
  has an unramified 1-dimensional quotient of order at most 2 follows from
  the fact that the Frey curve is semistable at~2 (see (4.1.5) of~\cite{serreconj})
  and the theory of the Tate curve. Finally, the claim that $\rho$ is flat at $\ell$
  is Proposition~5 and (4.1.13) of~\cite{serreconj}.
\end{proof}

Our job then becomes to prove that a hardly ramified representation is reducible.
Note that this claim is a consequence of Serre's conjecture~\cite{serreconj},
now a theorem of Khare and Wintenberger {\bf TODO cite KW}, and indeed we shall use
methods introduced by Khare and Wintenberger to prove this special case of
Serre's conjecture.

More precisely, we will prove the following three theorems. Firstly, we show that
an irreducible hardly ramified mod $\ell$ representation lifts to an $\ell$-adic representation.

\begin{theorem} If $\ell\geq3$ is prime and $\overline{\rho}:\GQ\to\GL_2(\Z/\ell\Z)$
  \label{hardly-ramified-lifts}
  is hardly ramified and irreducible, then there exists a finite extension~$K$ of $\Q_\ell$
  with integer ring~$\mathcal{O}$ and maximal ideal $\mathfrak{m}$
  and a hardly ramified representation
  $\rho:\GQ\to\GL_2(\mathcal{O})$ whose reduction modulo~$\m$ is isomorphic to $\rho$.
\end{theorem}

Next we show that a hardly ramified $\ell$-adic representation ``spreads out'' to a compatible
family of hardly ramified $q$-adic representations for all primes $q$.

\begin{theorem}
  \label{hardly-ramified-spreads-out}
  If $\ell\geq3$ is prime, $K$ is a finite extension of $\Q_\ell$
  with integers $\O$ and if $\rho:\GQ\to\GL_2(\O)$ is a hardly ramified representation,
  then there exists a number field $M$ and, for each finite place $\mu$ of $M$
  of characteristic prime to 2, with completion $M_\mu$ having integer ring $R_\mu$,
  a hardly representation $\rho_\mu:\GQ\to\GL_2(R_\mu)$, with the following properties:
  \begin{itemize}
    \item There is some $\lambda\mid\ell$ of $M$ such that $\rho_\lambda\cong\rho$,
      the isomorphism happening over some appropriate finite extension of $M_\lambda$;
    \item If $\mu_1$ and $\mu_2$ are two finite places of $M$ with odd characteristics $m_1$
      and $m_2$, and if $p\nmid 2m_1m_2$ is prime, then $\rho_{\mu_1}$ and $\rho_{\mu_2}$
      are both unramified at~$p$ and the characteristic polynomials $\rho_{\mu_1}(\Frob_p)$
      and $\rho_{\mu_2}(\Frob_p)$ lie in $M[X]$ and are equal.
  \end{itemize}
\end{theorem}

(note: do we need $\rho$ mod $\ell$ is irreducible in this theorem?)

In particular, we can ``move'' $\rho$ to a 3-adic representation
However, we can prove a very strong theorem about hardly ramified 3-adic Galois representations.

\begin{theorem}
  \label{hardly-ramified-3adic-reducible}
  Suppose $L/\Q_3$ is a finite extension, with integer ring $\O_L$, and suppose
  $\rho_3:\GQ\to\GL_2(\O_L)$ is hardly ramified. Then $\rho_3^{ss}=1\oplus\chi_3$
  where $1$ is the trivial character and $\chi_3$ is the 3-adic cyclotomic character.
\end{theorem}

Assuming all of these theorems, we can get what we want.

\begin{theorem}[Wiles,Taylor--Wiles, Ribet,\ldots]
  \label{FLT.Wiles_Frey'}
  \lean{FLT.Wiles_Frey}
  \uses{FLT.FreyCurve}
  \leanok
  If $\overline{\rho}$ is the mod $p$ Galois representation associated to a Frey package $(a,b,c,p)$ then
  $\overline{\rho}$ is not irreducible.
\end{theorem}
\begin{proof}
  Assume for a contradiction that $\overline{\rho}$ is irreducible. By theorem~\ref{hardly-ramified-lifts},
  $\overline{\rho}$ lifts to a hardly ramified $\ell$-adic reprepresentation $\rho$. By
  theorem~\ref{hardly-ramified-spreads-out}, $\rho$ is part of a compatible family of
  $q$-adic Galois representations. By theorem~\ref{hardly-ramified-3adic-reducible},
  any 3-adic member $\rho_3$ of this family has semisimplification $1\oplus\chi_3$ and in particular
  for $p\nmid 6$ we have that the characteristic polynomial of $\rho_3(\Frob_p)=(X-p)(X-1).$
  But by compatibility of the family we know that for $p\nmid 6\ell$ the characteristic
  polynomial of $\rho(\Frob_p)$ is $(X-p)(X-1)$, and thus the characteristic polynomial
  of $\overline{\rho}(\Frob_p)$ is $(X-p)(X-1)$. By the Brauer-Nesbitt theorem, $\overline{\rho}$
  is reducible, the contradiction we seek.
\end{proof}

We now must prove these three theorems. By far the easiest is the third theorem; this follows
from old estimates of Fontaine, originally developed by him to prove that there was no
nontrivial abelian scheme over $\Z$. The other two theorems are deeper, and both use
modern variants of Wiles' $R=T$ machinery.
