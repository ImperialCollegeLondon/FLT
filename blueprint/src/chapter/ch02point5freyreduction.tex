\chapter{Proving irreducibility}\label{ch_freyreduction}

\section{Goal}

In this chapter, we reduce the claim that the $p$-torsion in the Frey
curve is reducible to three hard theorems, two of which were not known
in the 1980s.

\section{Overview}

In Chapter~\ref{ch_reductions} we reduced FLT to the
following two claims: if $\rho$ is the $p$-torsion in the Frey curve
associated to a Frey package $(a,b,c,p)$, then $\rho$ is both
irreducible (by Mazur) and reducible (by Wiles). Mazur's 1977 theorem
is currently out of scope for this project, as it is ``too old''.
We thus focus on how to prove Wiles' theorem, which was still an open problem
in the 1980s (although it had been conjectured by Serre in~\cite{serreconj}
in 1987). In this section we state three difficult theorems and
show how Wiles' theorem follows easily from them.

\section{Facts about the Frey curve}

Let $(a,b,c,p)$ be a Frey package (so in particular $p\geq5$ is prime and $a^p+b^p=c^p$),
let $E$ be the corresponding Frey curve over $\Q$, and let $\rho:\GQ\to\Aut(E(\Qbar)[p])$
be the 2-dimensional Galois representation on the $p$-torsion of~$E$. Recall that our goal
is to prove that $\rho$ is reducible.

What we need to leverage is the fact that $\rho$ has very little ramification. To give
a toy example before we start: if $K$ is a number field (i.e., a finite extension of $\Q$)
and if the extension $K/\Q$ is unramified at all primes, then an old theorem of
Minkowski tells us that $K=\Q$. We want to prove a theorem in the same vein; if
something is hardly ramified, then it is somehow trivial. Let us make a precise
definition of what it means for a 2-dimensional representation to be \emph{hardly ramified}.

\begin{definition} A \emph{coefficient ring} is a local ring $R$ with maximal ideal $\m$ and
  finite residue field $k$, such that the natural map from $R$ to the projective
  limit $\lim_{n\in\N} R/\m^n$ is an isomorphism of rings.
\end{definition}

If $R$ is a coefficient ring then we topologise it by giving $R/\m^n$ the discrete
topology and $R=\lim_n R/\m^n$ the projective limit topology.

Examples of coefficient rings include finite fields, and integer rings of finite extensions
of $\Q_p$. There are also non-Noetherian examples, which are convenient to include
for technical reasons. If $R$ is a coefficient ring with maximal ideal $\m$
and residue field $k$ of characteristic~$\ell$, then there are ring homomorphisms
$\Z/\ell^n\Z\to R/\m^n$ for all naturals $n$ and thus a continuous ring homomorphism
$\Z_{\ell}\to R$. The $\ell$-adic cyclotomic character is a continuous
representation $\GQ\to\Z_{\ell}^\times$ and it thus induces a continuous
representation $\GQ\to R^\times$ which we also refer to as the cyclotomic character.

\begin{definition}
  \label{hardly_ramified}
  Let $R$ be a coefficient ring with finite residue field of characteristic $\ell\geq3$.
  Let $V$ be a finite free $R$-module of rank~2, equipped with the product topology. A
  continuous representation $\rho: \GQ\to \GL_R(V)$ is said to be \emph{hardly ramified} if it
  satisfies the following four conditions:
  \begin{enumerate}
  \item $\det(\rho):\GQ\to R^\times$ is the mod $\ell$ cyclotomic character;
  \item $\rho$ is unramified outside $2\ell$;
  \item There is a short exact sequence $0\to R\to V\to R\to 0$ restriction of $\rho$ to $\Gal(\Qbar_2/\Q)$ is reducible and has a
    1-dimensional quotient which is an unramified representation whose square is trivial;
  \item The restriction of $\rho$ to $\GQl$ comes from a finite flat group scheme.
  \end{enumerate}
\end{definition}

A well-known result, which basically goes back to Frey, is the following:

\begin{theorem}
  \label{Frey_curve_hardly_ramified}
  The $\ell$-torsion in the Frey curve associated to a Frey package $(a,b,c,\ell)$
  is hardly ramified.
\end{theorem}
\begin{proof}
  First note that $\ell\geq5>3$ by definition of a Frey package. Let $\rho$
  denote the Galois representation on the $\ell$-torsion of the Frey curve.
  The fact that $\rho$ is 2-dimensional is~Corollary
  III.6.4(b) of~\cite{silverman1}, and the fact that its determinant is
  cyclotomic is Proposition~III.8.3. These results hold for elliptic curves
  in general. The remaining claims are specific to the Frey curve and lie
  deeper. The fact that $\rho$ is unramified outside $2\ell$ is a consequence
  of (4.1.12) and (4.1.13) of~\cite{serreconj}. The fact that $\rho$ at 2
  has an unramified 1-dimensional quotient of order at most 2 follows from
  the fact that the Frey curve is semistable at~2 (see (4.1.5) of~\cite{serreconj})
  and the theory of the Tate curve. Finally, the claim that $\rho$ is flat at $\ell$
  is Proposition~5 and (4.1.13) of~\cite{serreconj}.
\end{proof}

Our job then becomes to prove that a hardly ramified representation is reducible.
Note that this claim is a consequence of Serre's conjecture~\cite{serreconj},
now a theorem of Khare and Wintenberger {\bf TODO cite KW}, and we shall use
methods introduced by Khare and Wintenberger to prove this special case of
Serre's conjecture.

More precisely, we will take the following approach. Assume $\rho$ is irreducible
and hardly-ramified. We will first lift $\rho$ to an $\ell$-adic representation
which is also hardly-ramified
This claim follows from the following three theorems.

{\bf TODO}

1) Lift rhobar to char 0
2) Put in a family
3) 3-adic hardly-ramified rep is extn of 1 by cyclo
