\chapter{An overview of the proof of Fermat's Last Theorem.}

Fermat's Last Theorem is the statement that if $a,b,c,n$ are positive integers with $n\geq 3$, then $a^n+b^n\not=c^n$. It is thus a statement about a family of \emph{Diophantine equations} ($a^3+b^3=c^3, a^4+b^4=c^4,\ldots$). Diophantus was a Greek mathematician who lived around 1800 years ago, and he would have been able to understand the statement of the theorem (he knew about positive integers, addition and multiplication).

Explaining a proof of Fermat's Last Theorem to Lean is in some sense like explaining the proof to Diophantus; for example, the proof starts by observing that before we go any further it's convenient to first invent/discover zero and negative numbers, and one can point explicitly at places in Lean's source code \href{https://github.com/leanprover/lean4/blob/260eaebf4e804c9ac1319532970544a4e157c336/src/Init/Prelude.lean#L1049}{here} and \href{https://github.com/leanprover/lean4/blob/260eaebf4e804c9ac1319532970544a4e157c336/src/Init/Data/Int/Basic.lean#L45}{here} where these things happen. However we will adopt a more efficient approach: we will assume all of the theorems both in Lean and in its mathematics library {\tt mathlib}, and proceed from there. 

The proof is by contradiction. We start by making some standard reductions. For convenience we say that a \emph{Frey package} is a triple $(a,b,c)$ of nonzero pairwise coprime integers with $a\equiv3\pmod4$ and $b\equiv0\pmod2$, and an odd prime $p$, such that $a^p+b^p=c^p$.

\begin{lemma}\label{WLOG_n_prime}\lean{???FLT.reduction_to_prime}
  If there is no Frey package, then Fermat's Last Theorem follows.
\end{lemma}
\begin{proof} We prove the contrapositive.
  Assume that $x,y,z,n$ are positive integers with $n\geq3$ and $x^n+y^n=z^n$. It is easy to check that $n$ is either a multiple of~4 or a multiple of an odd prime. Furthermore, if $n=rs$ is a factorization of $n$ then $(x^r)^s+(y^r)^s=(z^r)^s$, meaning that we may assume that either $n=4$ or that $n=p$ is an odd prime. An old result of Fermat himself (proved as {\tt fermatLastTheoremFour} in {\tt mathlib}) says that $x^4+y^4=z^4$ has no positive integer solutions, meaning that we can assume that $n=p$ is an odd prime.

  If the greatest common divisor of $x,y,z$ is $d$ then $x^p+y^p=z^p$ implies $(x/d)^p+(y/d)^p=(z/d)^p$, meaning that we can assume that no prime divides all of $x,y,z$. In fact if some prime divides two of the integers $x,y,z$ then by $x^p+y^p=z^p$ and unique factorization it must divide all three of them; thus we may even assume that $x,y,z$ are pairwise coprime. In particular we may assume that they are not all even, and now reducing modulo~2 shows that precisely one of them must be even. In fact we may assume that $y$ is the even number: if $x$ is even then we can switch $x$ and $y$, and if $z$ is even then we can replace $z$ by $-y$ and $y$ by $-z$ (using that $p$ is odd).

  Now $x$ is odd, so it is either 1 or~3 mod~4. If $x$ is 3 mod~4 then we set $a=x$, $b=y$ and $c=z$; if however $x$ is 1 mod~4 we set $a=-x$, $b=-y$ and $c=-z$. In either case we are done.
\end{proof}

\begin{lemma}\label{WLOG_p_at_least_5}\lean{???fermatLastTheoremThree} If $(a,b,c,p)$ is a Frey package then $p\not=3$.
\end{lemma}
\begin{proof} This an old result due essentially to Euler, which has already been formalised in Lean (although at the time of writing the result is not in mathlib).
\end{proof}

To continue, we need some of the theory of elliptic curves. So let $f(X)$ denote any monic cubic polynomial with rational coefficients and whose three complex roots are distinct, and let us consider the equation $E:Y^2=f(X)$, which defines a curve in the $(X,Y)$ plane. This curve (or strictly speaking its projectivisation) is a so-called elliptic curve. A good place to start learning about the general theory of elliptic curves is Wikipedia {\bf TODO: is it?}. 

If $K$ is now any field of characteristic 0, then we write $E(K)$ to be the set of solutions to $Y^2=f(X)$ with $X,Y\in K$, together with an additional ``point at infinity''. It is an extraordinary fact that $E(K)$ naturally has the structure of an abelian group, and we shall use $+$ to denote the group law. This group structure has the property that three distinct points $P,Q,R$ in the $(X,Y)$ plane sum to zero if and only if they are collinear, and the point at infinity is the identity of the group.

If $K$ and $L$ are two fields of characteristic zero and $f:K\to L$ is a field homomorphism, the map from $E(K)$ to $E(L)$ induced by $f$ is an additive group homomorphism. 

We will now define the key \emph{Galois representation} associated to a Frey package. If $A$ is any additive abelian group, then the subgroup $A[p]$ of elements $g$ such that $pg=0$ is naturally a vector space over the field $\Z/p\Z$. Given a Frey package $(a,b,c,p)$ we may consider the corresponding \emph{Frey curve}, considered by Frey and, before him, Hellgouarch. This is the elliptic curve $E$ defined by the equation $Y^2=X(X-a^p)(X+b^p)$; the roots of the cubic are distinct because $a,b,c$ are nonzero. Now let $\Qbar$ denote an algebraic closure of the rationals (for example, the algebraic numbers in $\bbC$) and let $G=\GQ$ denote the group of field isomorphisms $\Qbar\to\Qbar$. By the remarks above, each element $g\in G$ induces a group homomorphism $E(\Qbar)\to E(\Qbar)$, which is easily checked to be an isomorphism because its inverse is the map induced by $g^{-1}$. The group $G$ thus acts on $E(\Qbar)$, and hence on $V:=E(\Qbar)[p]$, the $p$-torsion in $E(\Qbar)$ (that is, the elements $e\in E(\Qbar)$ such that $p\times e=0$, where $p\times e$ is defined to be $e+e+\cdots+e$). The abelian group $V$ is naturally a vector space over the field $\Z/p\Z$, so we may consider this action of $\GQ$ on $V$ as being a \emph{representation} of $\GQ$.

Recall that a representation of a group $G$ on a vector space $W$ is said to be \emph{irreducible} if there are precisely two $G$-stable subspaces of $W$, namely $0$ and $W$. The representation is said to be \emph{reducible} otherwise.

Now say Fermat's Last Theorem is false. Take a Frey package $(a,b,c,p)$ and consider the corresponding mod $p$ representation of $\GQ$ coming from the $p$-torsion in the Frey curve as described above. Let's call this representation $\rho$. Is this representation reducible or irreducible?

\begin{theorem}[Mazur]\label{mazur_frey}\lean{???} $\rho$ cannot be reducible.\end{theorem}
\begin{proof} This follows from a profound result of Mazur from 1979 {\bf todo reference}, which boils down to the fact that the torsion subgroup of an elliptic curve over $\Q$ can have size at most~16. In fact some work still needs to be done to deduce the theorem from Mazur's result. We omit the argument for now -- it is explained in Proposition~6 of~\cite{serreconj}.
\end{proof}

\begin{theorem}[Wiles,Taylor--Wiles]\label{wiles_frey}\lean{???} $\rho$ cannot be irreducible either.\end{theorem}
\begin{proof} This is the main content of Wiles' magnum opus. We omit the argument for now, although later on in this project we will have a lot to say about it.
  \end{proof}

These two theorems together give the contraction required. We deduce

\begin{corollary}\label{FLT}\lean{???} Fermat's Last Theorem is true.\end{corollary}
\begin{proof}\end{proof}

In the remainder of this project we shall be filling in many of the details of these arguments.
