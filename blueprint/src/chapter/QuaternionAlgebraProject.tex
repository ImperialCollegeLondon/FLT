\chapter{Miniproject: Quaternion algebras}\label{Quat_alg_project}

\section{The goal}

At the time of writing, {\tt mathlib} still does not have a proof that the space
of classical modular forms of a fixed weight, level and character is finite-dimensional.
The main result in this miniproject is to prove that certain spaces of quaternionic modular forms
of a fixed weight, level and character are finite-dimensional. We need finiteness results
like this in order to control the Hecke algebras which act on these spaces; these Hecke
algebras are the ``$T$''s which are isomorphic to the ``$R$''s in the $R=T$ theorem which
is the big first target for the FLT project.

\section{Initial definitions}

Our first goal is the definition of these spaces of quaternionic modular forms. We start with
some preliminary work towards this.

Let $K$ be a field. A \emph{central simple $K$-algebra} is a $K$-algebra~$D$ with
centre $K$ such that $D$ has no nontrivial two-sided ideals. A \emph{quaternion algebra}
over $K$ is a central simple $K$-algebra of $K$-dimension~4.

Matrix algebras $M_n(K)$ are examples of central simple $K$-algebras, so
$2\times 2$ matrices $M_2(K)$ are an example of a quaternion algebra over $K$.
If $K=\bbC$ then $M_2(\bbC)$ is the only example, up to isomorphism, but there are
two examples over the reals, the other being Hamilton's quaternions
$\bbH:=\R\oplus\R i\oplus\R j\oplus\R k$ with the usual rules $i^2=j^2=k^2=-1$,
$ij=-ji=k$ etc. For a general field $K$ one can make an analogue of Hamilton's
quaternions $K\oplus Ki\oplus Kj\oplus Kk$ with these same rules to describe the
multiplication, and if the characteristic of~$K$ isn't 2 then this is a quaternion algebra
(which may or may not be isomorphic to $M_2(K)$ in this generality). If $K$ is a number field
then there are infinitely many isomorphism classes of quaternion algebras over $K$.

A fundamental fact about central simple algebras is that if $D/K$
is a central simple $K$-algebra and $L/K$ is an extension of fields, then $D\otimes_KL$
is a central simple $L$-algebra. In particular if $D$ is a quaternion algebra over $K$
then $D\otimes_KL$ is a quaternion algebra over $L$. Some Imperial students have established
this fact in ongoing project work.

A \emph{totally real field} is a number field~$F$ such that the image of every ring
homomorphism $F\to\bbC$ is a subset of $\R$. We fix once and for all a totally real field~$F$ and a
quaternion algebra $D$ over $F$. We furthermore assume that $D$ is \emph{totally definite}, that is,
that for all field embeddings $\tau:F\to\R$ we have $D\otimes_{F,\tau}\R\cong\bbH$.

The high-falutin' explanation of what is about to happen is that the units $D^\times$ of $D$
can be regarded as a connected reductive algebraic group over $F$, and we are going to define spaces
of automorphic forms for this algebraic group. For a general connected reductive algebraic group,
part of the definition of an automorphic form is that it is the solution to some differential
equations (for example modular forms are automorphic forms for the algebraic group $\GL_2$ over
$\Q$, and the definition of a modular form involves holomorphic functions, which are solutions to
the Cauchy--Riemann equations). However under the assumption that $F$ is totally real and $D/F$ is
totally definite, the associated symmetric space is 0-dimensional, meaning that no differential
equations are involved in the definition
of an automorphic form in this setting. As a consequence, the definition we're about to give
makes sense not just over the complex numbers but over any commutative ring $R$, which will
be crucial for us as we will need to think about, for example, mod~$p$ and more generally
mod~$p^n$ automorphic forms in this setting.

\section{The adelic viewpoint}

Having made assumptions on $D$ which makes the theory far less technical, we will now
make it a little more technical by using the modern adelic approach to the theory.
The adeles of a number field are discussed in far more detail
in the adele miniproject \ref{Adele_miniproject}. We just recall here that if $K$ is a number field
then there are two huge topological $K$-algebras called the \emph{finite adeles}
$\A_K^\infty$ and the \emph{adeles} $\A_K$ of $K$.

Let us now fix a \emph{weight}, a \emph{level} and a \emph{character}, and we will define
a space of automorphic forms for $D^\times$ of this weight, level and character. If you know
something about the theory of classical modular forms you will have seen these words used in that
theory. In the adelic set-up these words have slightly different interpretations. We will define
automorphic forms over $R$, an arbitrary commutative base ring. If you are thinking about analogies
with spaces of classical modular forms then you could imagine the case $R=\bbC$.

A \emph{weight} is a finitely-generated $R$-module~$W$ with an action of $D^\times$. For applications
to Fermat's Last Theorem we only need to consider the case where $W=R$ and the action is trivial,
but there is no harm in setting things up in more generality. The case $W=R=\bbC$ corresponds (via
the highly non-trivial Jacquet-Langlands correspondence) when $F=\Q$ to the case of modular forms
of weight 2, and for general~$F$ corresponds to Hilbert modular forms of parallel weight~2.

A \emph{level} is a compact open subgroup~$U$ of $(D\otimes_F\A_F^\infty)^\times$. These are plentiful.
The ring $D\otimes_F\A_F^\infty$ is a topological ring, and this fact is currently in the process
of being PRed to mathlib. Hence the units $(F\otimes_F\A_F^\infty)^\times$ of this ring are a topological
group. This group is locally profinite, and hence has many compact open subgroups; we will see
explicit examples later on.

A \emph{character} is a group homomorphism $\chi$ from $(\A_F^\infty)^\times$ to $R^\times$. For many
of the applications, $\chi$ can also be trivial, although we will crucially have to allow
certain nontrivial characters of $p$-power order when we are doing the patching argument needed
to prove the modularity lifting theorem which is the big first target of the FLT project.
We regard $\A_F^\infty$ as a subring of $D\otimes_F\A_F^\infty$, which is possible because
$F$ is a subring of $D$. More precisely we embed $\A_F^\infty$ into $D\otimes_F\A_F^\infty$
via the map sending $g$ to $1\otimes g$. Because $F$ is in the centre of~$D$, we have
that $\A_F^\infty$ is in the centre of $D\otimes_F\A_F^\infty$ and thus $(\A_F^\infty)^\times$
is a central subgroup of $(D\otimes_F\A_F^\infty)^\times$.

We fix a base commutative ring $R$, a weight $W$, a level $U$ and a character $\chi$.

\begin{definition}
  \lean{TotallyDefiniteQuaternionAlgebra.AutomorphicForm}
  \label{TotallyDefiniteQuaternionAlgebra.AutomorphicForm}
  \leanok
  An \emph{automorphic form} of weight $W$, level $U$ and character $\chi$ for $D$ is
  a function $f:(D\otimes_F\A_F^\infty)^\times\to W$ satisfying the following axioms:
  \begin{itemize}
    \item $f(\delta g)=\delta\cdot f(g)$ for all $\delta\in D^\times$ and $g\in (D\otimes_F\A_F^\infty)^\times$
    (this makes sense because $W$ has an action of $D^\times$).
    \item $f(gz)=\chi(z)f(g)$ for all $g\in (D\otimes_F\A_F^\infty)^\times$ and $z\in(\A_F^\infty)^\times$
    (this makes sense because $W$ is an $R$-module).
    \item $f(gu)=f(g)$ for all $g\in (D\otimes_F\A_F^\infty)^\times$ and $u\in U$.
  \end{itemize}
\end{definition}

Let $S_{W,\chi}(U;R)$ denote the space of automorphic forms of weight $W$, level $U$ and character
$\chi$. Two basic observations are

\begin{definition}
  \lean{TotallyDefiniteQuaternionAlgebra.AutomorphicForm.addCommGroup}
  \label{TotallyDefiniteQuaternionAlgebra.AutomorphicForm.addCommGroup}
  \uses{TotallyDefiniteQuaternionAlgebra.AutomorphicForm}
  \leanok
  Pointwise addition $(f_1+f_2)(g):=f_1(g)+f_2(g)$ makes $S_{W,\chi}(U;R)$ into an additive
  abelian group.
\end{definition}

\begin{definition}
  \lean{TotallyDefiniteQuaternionAlgebra.AutomorphicForm.module}
  \label{TotallyDefiniteQuaternionAlgebra.AutomorphicForm.module}
  \uses{TotallyDefiniteQuaternionAlgebra.AutomorphicForm,
    TotallyDefiniteQuaternionAlgebra.AutomorphicForm.addCommGroup}
  \leanok
  Pointwise scalar multiplication $(r\cdot f)(g):= r\cdot(f(g))$ makes
  $S_{W,\chi}(U;R)$ into an $R$-module.
\end{definition}

These spaces $S_{W,\chi}(U;R)$ are sometimes referred to as spaces of ``quaternionic modular forms''.
The Hecke algebras involved in the main modularity lifting theorems needed in the FLT project
will be endomorphisms of these spaces.

\section{Statement of the main result of the miniproject}

The point of this miniproject is the following goal:

\begin{theorem}
  \lean{TotallyDefiniteQuaternionAlgebra.AutomorphicForm.finiteDimensional}
  \label{TotallyDefiniteQuaternionAlgebra.AutomorphicForm.finiteDimensional}
  \uses{TotallyDefiniteQuaternionAlgebra.AutomorphicForm.module}
  If $R$ is a field~$k$ and the weight $W$ is a finite-dimensional $k$-vector space
  then the space $S_{W,\chi}(U;k)$ is a finite-dimensional $k$-vector space.
\end{theorem}

This is an analogue of the result that classical modular forms of a fixed
level, weight and character are finite-dimensional. In fact, by delicate results
of Jacquet and Langlands this result (in the case $k=\bbC$) implies many cases of that classical claim,
although of course the Jacquet--Langlands theorem is much much harder to prove than the classical
proof of finite-dimensionality.

The finite-dimensionality theorem is in fact an easy consequence of a finiteness assertion
which is valid in far more generality, namely for division algebras over number fields.
We state and prove this result in this generality. Let $K$ be a number field and let $B/K$
be a finite-dimensional central simple $K$-algebra. Assume furthermore that $B$ is a
\emph{division algebra}, that is, that every nonzero element of $B$ is a unit. The finiteness
theorem we want is this.

\begin{theorem}
  \lean{DivisionAlgebra.finiteDoubleCoset}
  \label{DivisionAlgebra.finiteDoubleCoset}
  If $U\subseteq (B\otimes_K\A_K^\infty)^\times$ is a compact open subgroup,
  then the double coset space $B^\times\backslash(B\otimes_K\A_K^\infty)^\times/U$ is a
  finite set.
\end{theorem}

I (kmb) had always imagined that this latter finiteness statement was ``folklore'' or
``a standard consequence of results about automorphic forms'', but in John Voight's
book~\cite{voightbook} this is Main Theorem 27.6.14(b) and Voight calls it Fujisaki’s lemma.

Let's prove Theorem~\ref{TotallyDefiniteQuaternionAlgebra.AutomorphicForm.finiteDimensional},
the finite-dimensionality of the space of quaternionic modular forms,
assuming Fujisaki's lemma.
\begin{proof}
  \proves{TotallyDefiniteQuaternionAlgebra.AutomorphicForm.finiteDimensional}
  \uses{DivisionAlgebra.finiteDoubleCoset}
  Choose a set of coset representative $g_1,g_2,\ldots,g_n$ for
  $D^\times\backslash(D\otimes_F\A_F^\infty)^\times/U$. My claim is that
  the function $S_{W,\chi}(U;k)\to W^n$ sending $f$ to $(f(g_1),f(g_2),\ldots,f(g_n))$
  is injective and $k$-linear, which suffices by finite-dimensionality of $W$.
  $k$-linearity is easy, so let's talk about injectivity.

  Say $f_1$ and $f_2$ are two elements of $S_{W,\chi}(U;k)$ which agree on
  each $g_i$. It suffices to prove that $f_1(g)=f_2(g)$ for all
  $g\in(D\otimes_F\A_F^\infty)^\times$. So say $g\in(D\otimes_F\A_F^\infty)^\times$,
  and write $g=\delta g_iu$ for $\delta \in D^\times$ and $u\in U$.
  Then $f_1(g)=f_1(\delta g_iu)=\delta\cdot f_1(g_i)$ (by hypotheses (i) and (iii)
  of the definition of $S_{W,\chi}(U;k)$), and similarly $f_2(g)=\delta\cdot f_2(g_i)$
  and because $f_1(g_i)=f_2(g_i)$ by assumption, we deduce $f_1(g)=f_2(g)$ as required.
\end{proof}

It thus remains to prove Fujisaki's lemma. Note that we will be assuming the results
from the adele mini-project~\ref{Adele_miniproject}.

\section{Proof of Fujisaki's lemma}

See Main Theorem 27.6.14 of Voight. TODO: make these into issues.
