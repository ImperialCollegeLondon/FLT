\chapter{Miniproject: Haar Characters}\label{Haar_char_project}

\section{The goal}

The goal of this miniproject is to develop the theory (i.e., the basic API) of Haar characters.
``Haar character'' is a name I've made up to describe a certain character of the units of a locally
compact topological ring. The main result we need here is that if $B$ is a finite-dimensional
algebra over a number field~$K$, then $B^\times$ is in the kernel of the Haar character
of $B\otimes_K\A_K$, where $\A_K$ is the ring of adeles of~$K$. Most if not all of this
should probably be in mathlib.

KMB would like to heartily thank S\'ebastian Gou\"ezel for the help he gave during the preparation
of this material.

\section{Initial definitions}

\subsection{Scaling Haar measure on a group}

Let $A$ be a locally compact topological additive abelian group. There's then a regular additive
Haar measure $\mu$ on $A$, unique up to a positive scalar factor. If $\phi:(A,+)\cong(A,+)$ is a
homeomorphism and an additive automorphism of $A$, then we can pull back $\mu$
to get a second measure $\phi^*\mu$ on $A$, with the property that
$\phi^*\mu(X)=\mu(\phi(X))$ for any Borel subset $X$ of $A$.

Now $\phi^*\mu$ is a translation-invariant and regular measure,
and hence also a Haar measure on $A$. It must thus differ from
$\mu$ by a positive scalar factor, which we call $d_A(\phi)$.
There is a choice of normalization here between $d_A(\phi)$
and $d_A(\phi)^{-1}$, so let us be more precise.

\begin{definition}
  \label{MeasureTheory.mulEquivHaarChar}
  \lean{MeasureTheory.mulEquivHaarChar}
  If $A$ is a locally compact topological additive abelian group,
  if $\mu$ is a regular Haar measure on $A$, and if $\phi:A\to A$ is an
  additive homeomorphism, then we let $d_A(\phi)$ denote the unique positive
  real number such that $\phi^*\mu(X)=d_A(\phi)\mu(X)$ for any Borel set~$X$.
\end{definition}

Here are some basic results about this construction.

\begin{lemma}
  \label{MeasureTheory.mulEquivHaarChar_eq}
  \lean{MeasureTheory.mulEquivHaarChar_eq}
  \mathlibok
  $d_A(\phi)$ is independent of choice of regular Haar measure.
\end{lemma}
\begin{proof}
  \mathlibok
  If $\mu'$ is a second choice then $\mu'=\lambda\mu$ for some
  positive real $\lambda$, and the $\lambda$s on each side of $\phi^*\mu'(X)=d_A(\phi)\mu'(X)$
  cancel.
\end{proof}

\begin{lemma}
  \label{MeasureTheory.mulEquivHaarChar_mul_integral}
  \lean{MeasureTheory.mulEquivHaarChar_mul_integral}
  If $f:A\to\R$ is a Borel measurable function then
  $d_A(\phi)\int f(x)d\phi^*\mu(x)=\int f(x)d\mu(x)$.
\end{lemma}
\begin{proof} This also follows immediately from the definition.
\end{proof}

\begin{lemma}
  \label{MeasureTheory.distribHaarChar_mul_volume}
  If $X$ is a Borel set then $\mu(\phi(X))=d_A(\phi)\mu(X)$.
\end{lemma}
\begin{proof} This follows immediately from the definition.
\end{proof}

Note that if $X$ is a Borel subset of $A$ with positive finite measure then we can read
off $d_A(\phi)$ by $d_A(\phi):=\mu(\phi(X))/\mu(X)$. A nice special case is when
$\mu(X)=1$, in which case we have $d_A(\phi)=\mu(\phi(X))$ for all $\phi$.
Similarly, if $f$ is a measurable function with $0<\int f(x)d\mu(x)<\infty$ then
we can read off $d_A(\phi)$ by $d_A(\phi)=(\int f(x)d\mu(x))/(\int f(x)\phi^*\mu(x))$.

\subsection{Scaling Haar measure on a ring}

Now let $R$ be a locally compact topological ring. The \emph{Haar character} of $R$,
or more precisely the \emph{left Haar character} of $R$, is a group homomorphism
$R^\times\to\R^\times$ defined in the following way. If $u\in R^\times$ then left multiplication
by $u$, namely the map $\ell_u:(R,+)\to(R,+)$ defined by $\ell_u(r)=ur$, is a homeomorphism and
an additive automorphism of $(R,+)$, so the preceding theory applies to $\ell_u$.

\begin{definition}
  We define $\delta_R(u)$ (or just $\delta(u)$ when the ring $R$ is clear) to be $d_R(\ell_u)$.
\end{definition}

The lemma in the previous subsection immediately apply

\begin{lemma}
  If $X$ is a Borel subset of $R$ and $r\in R^\times$ then $\mu(rX)=\delta_R(r)\mu(X)$.
\end{lemma}
\begin{proof} Immediate from lemma TODO ref.
\end{proof}

\begin{lemma}
  If $f:R\to\R$ is a Borel measurable function and $r\in R^\times$ then
  $\delta_R(r)\int f(rx)d\mu(x)=\int f(x)d\mu(x)$.
\end{lemma}
\begin{proof}
  Immediate from lemma TODO ref
\end{proof}

\begin{corollary}
  The function $\delta_R:R^\times\to\R_{>0}$ is continuous.
\end{corollary}
\begin{proof}
  Fix a Haar measure $\mu$ on $R$ and a continuous real-valued function
  on $R$ with compact support and such that $\int f(x) d\mu(x)\not=0$.
  Then $u \mapsto \int f(ux) d\mu(x)$ is a continuous function
  of $R\to\R$ (because a continuous function with compact support is uniformly
   continuous) and thus gives a continuous function $R^\times\to\R$.
   Thus the function $u\mapsto (\int f(ux) d\mu(x))/(\int f(x)d\mu(x))$ is
   a continuous function from $R^\times$ to $\R$ taking values in $\R_{>0}$.
   Hence $\delta_R$ is continuous, from lemma TODO
\end{proof}

\section{Examples}

We discuss some examples of Haar characters.

\begin{lemma}
  \label{distribHaarChar_real}
  \lean{distribHaarChar_real}
  If $R=\R$ then $\delta_R(u)=|u|$.
  \leanok
\end{lemma}
\begin{proof}\leanok Take $\mu$ to be Lebesgue measure and $X=[0,1]$.
We have $\delta(u)=\mu(uX)$. If $u>0$ then $u[0,1]=[0,u]$ which has measure $u=|u|$,
and if $u<0$ then $u*[0,1]=[u,0]$ which has measure $-u=|u|$.
\end{proof}

\begin{lemma}
  \label{distribHaarChar_complex}
  \lean{distribHaarChar_complex}
  \leanok
  If $R=\bbC$ then $\delta_R(u)=|u|^2$.
\end{lemma}
\begin{proof}
  \leanok
  Multiplication by a positive real $r$ sends a unit square to a square of area $r^2=|r|^2$.
  Multiplication by $e^{i\theta}$ is a rotation and thus does not change area.
  The general case follows.
\end{proof}

\begin{lemma}
  \label{distribHaarChar_padic}
  \lean{distribHaarChar_padic}
  \leanok
  If $R=\Q_p$ then $\delta_R(u)=|u|_p$, the usual $p$-adic norm.
\end{lemma}
\begin{proof}
  \leanok
  Normalise Haar measure so that $\mu(\Z_p)=1$.
  If $u$ is a $p$-adic unit then $u\Z_p=\Z_p$ so multiplication by $u$ didn't change
  Haar measure. If however $u=p$ then $u\Z_p$ has index $p$ in $\Z_p$ and, because
  $\mu(i+p\Z_p)=\mu(p\Z_p)$ we have that $\mu(\Z_p)=p\mu(p\Z_p)$ and thus $\delta(p)=p^{-1}$.
  These elements generate $\Q_p^\times$ and two characters which agree on generators
  of a group must agree on the group.
\end{proof}

\begin{remark}

If $R$ is a finite extension of $\Q_p$ then $\delta_R(u)$
is the norm on $R$ normalised in the following way:
$\delta_R(\varpi)=q^{-1}$, where $\varpi$ is a uniformiser
and $q$ is the size of the (finite) residue field. The proof is
the same as for $\Q_p$, but we won't need this.
\end{remark}

\section{Continuity}

\begin{theorem}
  \label{DistribHaarChar.continuous}
  If $R$ is a locally compact topological ring then
  $\delta_R:R\to\R_{>0}$ is continuous.
\end{theorem}
\begin{proof} (Taken from Sebastien Gouezel's proof \href{https://leanprover.zulipchat.com/#narrow/channel/217875-Is-there-code-for-X.3F/topic/.22Haar.20measure.20is.20continuous.22/near/487299979}{Zulip}).
  Fix a Haar measure $\mu$ on $R$ and a continuous real-valued function
  on $R$ with compact support and such that $\int f(x) d\mu(x)\not=0$.
  Then $$u \mapsto \int f(ux) d\mu(x)$$ is obviously a continuous function
  of $u$ (because a continuous function with compact support is uniformly
   continuous). And you can read the value of $\delta(u)$ on how $u$
   rescales the integral of continuous functions, so continuity of $\delta$
   follows.
\end{proof}

% Note that if $R$ is non-commutative then there is also a right Haar character $\delta'(u)$
% defined as $\delta'(u)=d_R(\psi)$ where $\psi(r)=ru$. In general these characters do not
% coincide; for example if $R$ is the ring of upper-triangular $2\times 2$ matrices with
% real coefficients then left multiplication by $\begin{pmatrix}2&0\\0&1\end{pmatrix}$ sends $\begin{pmatrix}a&b\\0&c\end{pmatrix}$
% to $\begin{pmatrix}2a&2b\\0&c\end{pmatrix}$ and thus scales Haar measure by a factor of 4, but right multiplication
% by $\begin{pmatrix}2&0\\0&1\end{pmatrix}$ sends it to $\begin{pmatrix}2a&&b\\0&&c\end{pmatrix}$ and thus only scales Haar measure
% by a factor of 2. We will see later on that for (inner forms of) matrix algebras this phenomenon
% does not occur.

\section{Algebras}

  Say $F$ is a locally compact topological field (for example $\R$ or $\bbC$ or $\Q_p$), $V$
  is a finite-dimensional $F$-vector space, and $\phi:V\to V$ is an invertible $F$-linear map.
  Then $V$ with its module topology (which is the product topology if one picks a basis)
  is a locally compact topological abelian group, and $\phi$ is additive.
  One can check that linearity implies continuity (this is {\tt IsModuleTopology.continuous\_of\_linearMap} in mathlib),
  so in fact $\phi$ is a homeomorphism
  and our theory applies. The following lemma gives a formula for the scale factor $d_V(\phi)$.

\begin{lemma}
  \label{addHaarScalarFactor_eq_distribHaarChar_det}
  %\lean{addHaarScalarFactor_eq_distribHaarChar_det}
  $d_V(\phi)=\delta_F(\det(\phi))$, where $\det(\phi)\in F$ is the determinant of $\phi$ as an $F$-linear map.
\end{lemma}
\begin{proof}
The proof should be inspired by \href{https://leanprover-community.github.io/mathlib4\_docs/Mathlib/MeasureTheory/Measure/Lebesgue/Basic.html\#Real.map\_matrix\_volume\_pi\_eq\_smul\_volume\_pi}{\tt Real.map\_matrix\_volume\_pi\_eq\_smul\_volume\_pi},
which crucially uses the induction principle \href{https://leanprover-community.github.io/mathlib4\_docs/Mathlib/LinearAlgebra/Matrix/Transvection.html\#Matrix.diagonal\_transvection\_induction\_of\_det\_ne\_zero}{\tt Matrix.diagonal\_transvection\_induction\_of\_det\_ne\_zero}.
In short, one needs to check it for diagonal matrices and for matrices which are the identity
except that one off-diagonal entry is non-zero. Note that we only need the result for $F=\R$
and $F=\mathbb{Q}_p$, so if it helps we can assume that $F$ is second countable.
\end{proof}

  Now say $F$ is a locally compact topological field, and $R$ is a (possibly non-commutative)
  $F$-algebra, finite-dimensional as an $F$-vector space. Then both left and right multiplication
  by a fixed element of $R$ are $F$-linear. We used left multiplication in our definition
  of the Haar character. If $u\in R^\times$ then let us denote by $\ell_u:R\to R$
  the function $r\mapsto ur$. This is an $F$-linear automorphism of $R$, and hence has a
  determinant $\det(\ell_u)\in F^\times$.
  The preceding lemma then immediately implies

  \begin{corollary}
  \label{distribHaarChar_algebra}
  %\lean{distribHaarChar_algebra}
  If $u\in R^\times$ then $\delta_R(u)=\delta_F(\det(\ell_u))$.
\end{corollary}
\begin{proof}
  Follows immediately from the preceding lemma.
\end{proof}

\section{Left and right Haar characters}

If $R$ is a non-commutative ring, then left and right multiplication by an element
can scale Haar measure in different ways.
For example if $R$ is the upper-triangular $2\times 2$ matrices with real
entries, then left multiplication by $\begin{pmatrix}a&0\\0&1\end{pmatrix}$
sends $\begin{pmatrix}x&y\\0&z\end{pmatrix}$ to $\begin{pmatrix}ax&ay\\0&z\end{pmatrix}$
and thus scales $R$'s additive
Haar measure by $|a|^2$, but right multiplication by $\begin{pmatrix}a&0\\0&1\end{pmatrix}$
sends $\begin{pmatrix}x&y\\0&z\end{pmatrix}$ to $\begin{pmatrix}ax&y\\0&z\end{pmatrix}$
and thus scales $R$'s additive Haar measure by a factor of $|a|$.

What's going on here is that if we regard left and right multiplication as $\R$-linear
maps from $R$ to $R$, then their associated matrices with respect to the obvious basis
are $diag(a,a,1)$ and $diag(a,1,1)$, which have different determinants.

However, if $k$ is now any field and if $B$ is a finite-dimensional central
simple algebra over $k$ (for example a quaternion algebra, the case we'll care about later),
and if $u\in B^\times$ then $x\mapsto ux$ and $x\mapsto xu$
are both $k$-linear endomorphisms of $B$, and I claim that they have
the same determinant.

\begin{lemma}
  \label{left_det_eq_right_det}
  %\lean{left_det_eq_right_det}
  If $u\in B^\times$, if $\ell_u:B\to B$ sends $x$ to $ux$ and if $r_u:B\to B$
  sends $x$ to $xu$ then $\det(\ell_u)=\det(r_u)$ as $k$-linear endomorphisms of $B$.
\end{lemma}
\begin{proof}
  Determinants are unchanged by base extension, so WLOG $k$ is algebraically closed.
  Then it's known that $B$ must be a matrix algebra, say $M_n(k)$. Now $u$ can be thought
  of as a matrix which has its own intrinsic determinant $d$, and $B$ as a left $B$-module
  becomes a direct sum of $n$ copies of $V$, the standard $n$-dimensional representation of $B$.
  Thus $\det(\ell_u)=d^n$. Similarly $\det(r_u)=d^n$ and in particular they are equal.
\end{proof}

\begin{corollary}
  \label{distribHaarChar_eq_addHaarScalarFactor_right_of_isCentralSimple}
  %\lean{distribHaarChar_eq_addHaarScalarFactor_right_of_isCentralSimple}
  If $B$ is a central simple algebra over a locally compact field $F$, and if $u\in B^\times$,
  then $\delta_B(u):=d_B(\ell_u)$ is equal to $d_B(r_u)$.
\end{corollary}
\begin{proof}
  If $\ell_u$ and $r_u$ denote left and right multiplication by $u$ on $B$, then we have
  seen in lemma~\ref{addHaarScalarFactor_eq_distribHaarChar_det} that $d_B(r_u)=\delta_F(\det(r_u))$.
  Lemma~\ref{left_det_eq_right_det} tells
  us that this is $\delta_F(\det(\ell_u))$ and this is $\delta_B(u)$ by
  corollary~\ref{distribHaarChar_algebra}.
\end{proof}

\section{Products}

Here are two facts which we will need about products.

\begin{lemma}
  \label{addHaarScalarFactor_prod}
  %\lean{addHaarScalarFactor_prod}
  If $(A,+)$ and $(B,+)$ are locally compact topological abelian groups,
  and if $\phi:A\to A$ and $\psi:B\to B$ are additive homeomorphisms,
  then $\phi\times\psi:A\times B\to A\times B$ is an additive homeomorphism (this is
  obvious), and
  $d_{A\times B}(\phi\times\psi)=d_A(\phi)d_B(\psi)$.
\end{lemma}
\begin{proof}
  We only need this result in the case where both $A$ and $B$ are second-countable, in which case
  {\tt Prod.borelSpace} can be used to show that Haar measure on $A\times B$ is the product of
  Haar measures on $A$ and $B$, and in this case the result follows easily. Without this assumption,
  the product of these measures may not even be a Borel measure and one has to be more careful.
  The proof in this case is explained \href{https://leanprover.zulipchat.com/#narrow/channel/116395-maths/topic/Product.20of.20Borel.20spaces/near/487257981}{here}.
\end{proof}

\begin{lemma}
  \label{addHaarScalarFactor_pi_finite}
  %\lean{addHaarScalarFactor_pi_finite}
  If $A_i$ are a finite collection of locally compact topological abelian groups,
  with $\phi_i:A_i\to A_i$ additive homeomorphisms, then $d_{\prod_i A_i}(\prod_i\phi_i)=\prod_i d_{A_i}(\phi_i)$.
\end{lemma}
\begin{proof}
  Induction on the size of the finite set, using the previous lemma.
\end{proof}

\begin{lemma}
  \label{distribHaarChar_prod}
  %\lean{distribHaarChar_prod}
  If $R$ and $S$ are locally compact topological rings, then $\delta_{R\times S}(r,s)=\delta_R(r)\times\delta_S(s)$.
\end{lemma}
\begin{proof}
  Follows immediately from lemma~\ref{addHaarScalarFactor_prod}.
\end{proof}

\begin{lemma}
  \label{distribHaarChar_pi_finite}
  %\lean{distribHaarChar_pi_finite}
  If $R_i$ are a finite collection of locally compact topological rings,
  and $u_i\in R_i^\times$ then $\delta_{\prod_i R_i}((u_i)_i)=\prod_i\delta_{R_i}(u_i)$.
\end{lemma}
\begin{proof}
  Follows immediately from lemma~\ref{addHaarScalarFactor_pi_finite}.
\end{proof}

Now say $A=\prod'_i A_i$ is the restricted product of a collection of topological abelian additive groups $A_i$
  equipped with compact open subgroups $C_i$. Say $\phi_i:A_i\to A_i$ are additive homeomorphisms
  with the property that $\phi_i(C_i)=C_i$ for all but finitely many $i$. Then the restricted
  product $\prod'_i\phi_i$ is an additive homeomorphism $\phi:A\to A$. Note also that
  $d_{A_i}(\phi_i)=1$ for all the $i$ such that $\phi_i(C_i)=C_i$, as $d_{A_i}(\phi_i)$ can be
  computed as $\mu(\phi_i(C_i))/\mu(C_i)$ and $\mu(C_i)$ is guaranteed to have positive finite measure
  as it is open and compact. Thus the product $\prod_i\delta(\phi_i)$ is a finite product (and in
  particular cannot diverge to $0$). Moreover, we have

\begin{lemma}
  \label{addHaarScalarFactor_restricted_product}
  %\lean{addHaarScalarFactor_restricted_product}
  With $A$, $A_i$, $C_i$, $\phi_i$, $\phi$ defined as above, we have
  $\delta_A(\phi)=\prod_i\delta_{A_i}(\phi_i)$.
\end{lemma}
\begin{proof}
  Assume $\phi_i(C_i)=C_i$ for all $i\not\in S$, a finite set, and work in the
  open subgroup $\prod_{i\in S}A_i\times\prod_{i\not\in S}C_i$. Then $\phi$ is an automorphism of
  this open subgroup of $A$, and in particular we can compute $\delta(\phi)$ using
  this group instead. Because $\prod_{i\not\in S}\phi_i$ is an automorphism of a compact
  group we must have $\delta(\phi_{i\not\in S}\phi_i)=1$. Finally
  $\delta(\prod_{i\in S}\phi_i)=\prod_{i\in S}\delta(\phi_i)$ by the previous result.
\end{proof}

As a special case, if $R$ is the restricted product of a collection of topological rings $R_i$
  (not necessarily commutative) each equipped with a compact open subring $C_i$, then
  we have

\begin{corollary}
  \label{distribHaarChar_restricted_product}
  %\lean{distribHaarChar_restricted_product}
  If $u=(u_i)_i\in R^\times$ then $\delta_R(u)=\prod_i\delta_{R_i}(u_i)$.
\end{corollary}
\begin{proof}
  By definition of restricted product we have $u_i\in C_i$ for all but finitely many $i$.
  Note also that $u$ has an inverse $v=(v_i)_i$ with $v_i\in C_i$ for all but finitely many $i$.
  The fact that $u_iv_i=v_iu_i=1$ means that $u_i,v_i\in C_i^\times$ for all but finitely many $i$.
  Thus the previous lemma applies.
\end{proof}

\section{Adeles}

We finish this miniproject by proving two results about Haar characters for
adelic rings. So let $K$ be a field and let $\A_K$ be the adeles of $K$.
We will prove some theorems about $\A_K$-algebras $R$ which are finite
and free as $\A_K$-modules. Such algebras can be given the $\A_K$-module topology
and this makes them into topological rings. In fact we shall only be concerned
with algebras of the form $B\otimes_K\A_K$ where $B$ is a finite-dimensional
$K$-algebra. So fix such a $B$, and write $B_{\A}$ for $B\otimes_K\A_K$.

\begin{theorem}
  \label{addHaarScalarFactor.left_mul_eq_right_mul}
  Let $B$ be a finite-dimensional central simple $K$-algebra.
  Say $u\in B_{\A}^\times$, and define $\ell_u$ and $r_u:B_{\A}\to B_{\A}$ by
  $\ell_u(x)=ux$ and $r_u(x)=xu$. Then $d_{B_{\A}}(\ell_u)=d_{B_{\A}}(r_u)$.
\end{theorem}
\begin{proof} If $u=(u_v)$ as $v$ runs through the places of $K$ then
  $d_{B_{\A}}(\ell_u)=\prod_v d_{B_v}(\ell_{u_v})$ by
  theorem~\ref{addHaarScalarFactor_restricted_product} (and the product is finite).
  By corollary~\ref{distribHaarChar_eq_addHaarScalarFactor_right_of_isCentralSimple}
  this equals $\prod_v d_{B_v}(r_{u_v})$, and again by
  theorem~\ref{addHaarScalarFactor_restricted_product} this is $d_{B_{\A}}(r_u)$.
\end{proof}

That theorem only works for inner forms of matrix algebras, but the below theorem,
a generalization of the adelic product formula, is valid for any finite-dimensional
$K$-algebra.
We have seen in lemmas~\ref{distribHaarChar_real} and~\ref{distribHaarChar_padic}
  that if $v$ is a place of $\Q$ (i.e., a prime
  number or $+\infty$) then $\delta_{\Q_v}=|\cdot|_v$.
  We can deduce from this, lemma~\ref{distribHaarChar_restricted_product} (at the finite places)
  and lemma~\ref{distribHaarChar_prod} (to include the infinite place) that if $(x_v)_v\in\A_{\Q}^\times$
  then $\delta_{\A_{\Q}}((x_v)_v)=\prod_v|x_v|_v$, where the product
  runs over all places of $\Q$.
  The \emph{product formula for $\Q$} (which we have in mathlib) says
  that if $x\in\Q^\times\subseteq\A_{\Q}^\times$ then $\prod_v|x|_v=1.$
  A quick proof: if $x=\pm\prod_pp^{e_p}$ then $\prod_p|x|_p=\prod_pp^{-e_p}$
  and $|x|_\infty=\prod_pp^{e_p}$ so they cancel. We conclude that $\delta_{\A_{\Q}}(\Q^\times)=\{1\}$.

  Next we generalize this to finite-dimensional $\Q$-vector spaces.

  So say $V$ is an $N$-dimensional $\Q$-vector space,
  and define $V_{\A}:= V\otimes_{\Q}\A_{\Q}$ with its $\A_{\Q}$-module topology.
  If we choose an isomorphism $V\cong\Q^N$ then $V_{\A}\cong\A_{\Q}^N$
  as an additive topological abelian group. In particular, $V_{\A}$ is locally compact.

  Fix a $\Q$-linear automorphism $\phi:V\to V$. By base extension $\phi$ induces
  an $\A_{\Q}$-linear automorphism $\phi_{\A}$ of $V_{\A}$ which is also a homeomorphism of $V_{\A}$
  if $V_{\A}$ is given the module topology as an $\A_{\Q}$-module. Our goal is

  \begin{theorem}
    \label{addHaarScalarFactor_eq_one}
    %\lean{addHaarScalarFactor_eq_one}
    In the above situation ($V$ a finite-dimensional $\Q$-vector space, $\phi:V\to V$ is $\Q$-linear,
    $\phi_{\A}$ the base extension to $V_{\A}:=V\otimes_{\Q}{\A_{\Q}}$, a continuous linear
    endomorphism of $V_{\A}$ with the $\A_{\Q}$-module topology), we have $d_{V_{\A}}(\phi_{\A})=1.$
  \end{theorem}
  \begin{proof}
    Notation: write $\A_{\Q}=\A_{\Q}^\infty\times\R$, where $\A_{\Q}^\infty$ is the finite adeles of $\Q$.
    Define $V_f:= V\otimes_{\Q}\A_{\Q}^\infty$ and for $v$ a place of $\Q$ (that is either $v=p$ prime
    or $v=\infty$) write $V_v:=V\otimes_{\Q}\Q_v$, so for example $V_\infty=V\otimes_{\Q}\R$.
    Then $V_{\A}=V_f\times V_\infty$. Fix once and for all a $\Z$-lattice $L\subseteq V$
    (that is, a spanning $\Z^N$ in the $\Q^N$).
    For $p$ a prime, define $C_p:=L\otimes_{\Z}\Z_p\subseteq V_p$. It can be checked that
    as a topological abelian group $V_f$ is the restricted product $\prod'_p V_p$ over the
    subgroups $C_p$.

    We have $\phi:V\to V$. For $v$ a place of $\Q$, let $\phi_v:V_v\to V_v$ denote the $\Q_v$-linear
    base extension of $\phi$ to $V_v$. We have $d_{V_{\A}}(\phi_{\A})=d_{V_f}(\prod'_p \phi_p)\times d_{V_\infty}(\phi_\infty)$,
    where $\prod'_p \phi_p$ is the restricted product of the $\phi_p$ as an endomorphism of $V_f$.
    Thus $d_{V_{\A}}(\phi_{\A})=\prod_p d_{V_p}(\phi_p)\times d_{V_\infty}(\phi_\infty)=\prod_v d_{V_v}(\phi_v)$, where
    we note that all but finitely many of the $d_{V_p}(\phi_p)$ are 1.

    By Lemma~\ref{addHaarScalarFactor_eq_distribHaarChar_det} we have that
    $d_{V_v}(\phi_v)=\delta_{\Q_v}(\det(\phi_v))$, hence $d_{V_{\A}}(\phi_{\A})=\prod_v\delta_{\Q_v}(\det(\phi_v))$.
    But $\det(\phi_v)$ is equal to the determinant of $\phi$ on $V$ as $\Q$-vector space (because
    base change does not change determinant),
    which is some nonzero rational number $q$. Thus $d_{V_{\A}}(\phi_{\A})=\prod_v\delta_{\Q_v}(q)=1$
    by the product formula for $\Q$.
  \end{proof}

  \begin{corollary}
    \label{distribHaarCharacter_kernel_tensor_adeleRing}
    \lean{distribHaarCharacter_kernel_tensor_adeleRing}
    If $B$ is a finite-dimensional $\Q$-algebra (for example a number field, or a quaternion algebra over a number field),
    if $B_{\A}$ denotes the ring $B\otimes_{\Q}\A_{\Q}$, and if $b\in B^\times$,
    then $\delta_{B_{\A}}(b)=1$.
  \end{corollary}
  \begin{proof}
    Follows immediately from the previous theorem.
  \end{proof}

  \begin{corollary} If $B$ is a finite-dimensional $\Q$-algebra and
    if $b\in B^\times$ then right multiplication by $b$
    does not change Haar measure on $B_{\A}$.
  \end{corollary}
  \begin{proof}
    Follows immediately from the previous theorem.
  \end{proof}
