\chapter{An example of an automorphic form}

\section{Introduction}
Parts of this document can serve as an introduction to some of the mathematics used in the proof of
Fermat's Last Theorem. The original proof of Wiles used modular forms, but we also need a
variant of modular forms which we'll call quaternionic modular forms. A quaternionic modular
form, in the cases of interest to us, is just a function from a certain double coset space
to the complex numbers. The fact that the double coset space is finite corresponds to the statement
that the corresponding space of quaternionic modular forms is finite.

Note that mathlib has modular forms, but it doesn't have enough complex analysis to deduce
that the space of modular forms of a given weight and level is finite-dimensional. If this
chapter, which contains a worked exercise computing a space of modular forms, is completed
before mathlib gets the necessary complex analysis, then the first nonzero space of modular forms
to be proved finite-dimensional in Lean will be a space of quaternionic modular forms.


\section{$\Zhat$}

We need something called $\Zhat$, the profinite completion of $\Z$. First a low-level explanation
of this object. Given an integer $z$, we can reduce it mod $N$ for every positive natural
number and get elements $z_N=\overline{z}\in\Z/N\Z$. What can we say 
about the relationship between the elements $z_N\in\Z/N\Z$? These elements cannot just be
arbitrary: for example we can't have $z_{10}=6$ and $z_2=1$, because $z_{10}=6$ tells us that
$z$ ends in a 6 when written in base 10, and in particular it's even, so $z_2$ must be~0.
The general rule is that if $D\mid N$ then $z_D$ can be read off from $z_N$ (just divide it by~$D$
and take the remainder).

An element of the ring $\Zhat$ is a collection $z_1$, $z_2$, $z_3$, \ldots with $z_N\in\Z/N\Z$
and such that the colletion is compatible, in the sense that if $D\mid N$ then $z_N$ reduces mod~$D$
to $z_D$.

Examples of course are given by integers, where we define $z_N$ to be $z$ mod $N$ for all $N$.
But there are many other examples; for example an infinite sum of factorials 
$0!+1!+2!+3!+4!+5!+\cdots$ gives us an element of $\Zhat$, because if you want to know what
this sum is modulo $N$ to just wait until it gets up to $N!$ because all terms from then on
are multiples of~$N$ and hence can be ignored mod~$N$; the point is that this sum is finite
mod~$N$ for any~$N$, and it's easily checked that the resulting colletion of numbers mod~$N$
are compatible.

It is easily checked that pointwise addition/negation/zero/one/multiplication make this
collection into a commutative ring. 

People who have seen some more advanced algebra might want to think of $\Zhat$ as
the profinite completion of $\Z$, that is, the projective limit of the finite quotients
$\Z/N\Z$ of $\Z$. It may or may not help to know that $\Zhat=\prod_p\Z_p$, the product over all
primes~$p$ of the $p$-adic integers.

We are going to "complete" various objects by tensoring them over $\Z$ with $\Zhat$.

\section{Warm-up: the additive structure of the completion of $\Q$}

We now consider $\Qhat:=\Q\otimes_{\Z}\Zhat$. This is some sort of a ``completion'' of $\Q$.
A reminder: a general element of $A\otimes B$ is not of the form $a\otimes b$, but
is more generally a finite sum of such elements. However we don't have any such
problems with $\Q\otimes_{\Z}\Zhat$, as the below lemma shows.

\begin{lemma}\label{Qhat.canonicalForm} Every element of $\Qhat:=\Q\otimes_{\Z}\Zhat$
can be written as $q\otimes z$ with $q\in\Q$ and $z\in\Zhat$.
\end{lemma}
\begin{proof} Take general element $\sum_i q_i\otimes z_i\in\Qhat$. Now choose a large 
    positive integer $N$, the lowest
common multiple of all the denominators showing up in the $q_i$, and then rewrite
the general element as $\sum_i \frac{n_i}{N}\otimes z_i$ with $n_i\in\Z$. Now recall that a fundemantal
equality in the tensor product $A\otimes_\ZB$ is that $na\otimes b=a\otimes nb$ for $n\in\Z$.
Applying this rule, we see that our general element is $\sum_i \frac{1}{N}\otimes n_i z_i$
and hence equal to $\frac{1}{N}\otimes(\sum_i n_i z_i)$ and thus of the form $q\otimes z$ with $q\in\Q$
and $z\in\Zhat$.
\end{proof}

Be careful though: just because every element of $\Qhat$ can be written as $q\otimes z$, this
reprsentation may not be unique. For example $2\otimes 1=1\otimes 2$.

The question we address in this section is how to uderstand $\Q\otimes_{\Z}\Zhat$ as an additive
abelian group, and in particular in terms of its two obvious subgroups $\Q\otimes_{\Z}1=\Q\otimes_{\Z}\Z$,
isomorphic to $\Q$, and $1\otimes\Zhat=\Z\otimes_{\Z}\Zhat$, isomorphic to $\Zhat$. I claim that these two
subgroups generate $\Q\otimes_\Z$, and their intersection is just the copy $\Z\otimes_{\Z}\Z$ of $\Z$ in $\Zhat$.
To prove this we need to understand what the multiples of~$N$ are in $\Zhat$.

\begin{lemma}{ZHat.multiples} The multiples of~$N$ in $\Zhat$ are precisely the compatible collections $(z_i)_i$ with
$z_N=0$.
\end{lemma}
\begin{proof}
    Clearly $z_N=0$ is a necessary condition to be a multiple of~$N$. To see it is sufficient, take a general $(z_i)$
    which is a multiple of~$N$, note that $z_N=0$, and now define a new element $(y_j)_j$ of $\Zhat$
    by $y_j=z_{Nj}/N$. Just to clarify what this means: $z_{Nj}\in\Z/Nj\Z$ maps to $z_N=0$ in $\Z/N\Z$,
    so is in the subgroup $N\Z/Nj\Z$ of $\Z/Nj\Z$, which is isomorphic (via "division by $N$") to the
    ring $\Z/j\Z$; this is how we construct $y_j$. It is easily checked that the $y_j$ are compatible
    and that $Ny=z$.
\end{proof}



\end{proof}