\chapter{An example of an automorphic form}

\section{Introduction}

Before we launch into the general theory of quaternionic modular forms and their
profound relationship with arithmetic, I thought it would be interesting to do a 
concrete worked example in Lean of the space of weight 2 level 2
modular forms for the units in everybody's favourite quaternion algebra over $\Q$, namely
$\Q\oplus\Q i\oplus\Q j\oplus\Q k$ with $i^2=j^2=k^2=ijk=-1$ as usual. The calculation will
boil down to the fact that every prime (and hence every natural) is the sum of four
natural squares.

Parts of the FLT blueprint can serve as an introduction to some of the mathematics used in the proof of
Fermat's Last Theorem. The original proof of Wiles used modular forms, but we also need a
variant of modular forms which we'll call quaternionic modular forms. A quaternionic modular
form, in the cases of interest to us, is just a function from a certain finite double coset space to the complex numbers. Proving that the double coset space is the same thing as
proving that that the corresponding space of quaternionic modular forms is finite-dimensional.

Note that mathlib has modular forms, but it doesn't have enough complex analysis to deduce
that the space of modular forms of a given weight and level is finite-dimensional. If this
chapter, which contains a worked exercise computing a space of quaternionic modular forms, is completed
before mathlib gets the necessary complex analysis, then the first nonzero space of modular forms
to be proved finite-dimensional in Lean will be a space of quaternionic modular forms.


\section{$\Zhat$}

We need something called $\Zhat$, the profinite completion of $\Z$. First a low-level explanation
of this object. Given an integer $z$, we can reduce it mod $N$ for every positive natural
number and get elements $z_N=\overline{z}\in\Z/N\Z$. What can we say 
about the relationship between the elements $z_N\in\Z/N\Z$? These elements cannot just be
arbitrary: for example we can't have $z_{10}=6$ and $z_2=1$, because $z_{10}=6$ tells us that
$z$ ends in a 6 when written in base 10, and in particular it's even, so $z_2$ must be~0.
The general rule is that if $D\mid N$ then $z_D$ can be read off from $z_N$ (just divide it by~$D$
and take the remainder).

An element of the ring $\Zhat$ is a collection $z_1$, $z_2$, $z_3$, \ldots with $z_N\in\Z/N\Z$
and such that the collection is compatible, in the sense that if $D\mid N$ then $z_N$ reduces mod~$D$
to $z_D$.

Examples of course are given by integers, where we define $z_N$ to be $z$ mod $N$ for all $N$. This gives us
a natural injection from $\Z$ to $\Zhat$. But $\Zhat$
is much much larger than $\Z$; it has the same cardinality as the reals in fact. In fact, is not difficult to write down explicit examples of elements of $\Zhat$ which aren't in $\Z$.

\begin{example} The infinite sum $0!+1!+2!+3!+4!+5!+\cdots$ is in fact a finite sum modulo $N$ for every positive integer $N$, because all terms after the $N$th are multiples of $N$.
    Let $e_N$ be the value of this finite sum modulo $N$. Then the collection $(e_N)_N$ is an element of $\Zhat$ which is not in $\Z$. 
\end{example}
The proof that the $e_N$ are compatible is some not-too-bad-looking arithmetic and finite sum manipulation. I am not sure I
know how to prove it's not in $\Z$ but it looks doable.

It is easily checked that pointwise addition/negation/zero/one/multiplication makes 
$\Zhat$ into a commutative ring. 

\section{More advanced remarks on $\Zhat$ versus $\Q$}

This section can be skipped on first reading.

People who have seen some more advanced algebra might recognise this construction of $\Zhat$
as being the profinite completion of the additive abelian group $\Z$, so it is a fundamental
object of mathematics in some sense. But usually after $\Z$ we go to $\Q$, a multiplicative
localisation, and only complete after that; $\Z$ is not dense in $\R$ or in the closed unit disc in $\R$. However $\Q$ is dense in both. Yet $\Z$ is dense both in the $p$-adic integers $\Z_p$ and in the profinite completion $\Zhat$ of $\Z$; the nonarchimedean theory of completion works at integral level. 

Even though $\Q$ is a divisible abelian group and hence its profinite completion vanishes,
we can still attempt to "locally profinitely complete it" by defining $\Qhat:=\Q\otimes_{\Z}\Zhat$. This object is knows as the \emph{finite adeles} of $\Q$. More generally if $F$ is
any number field then $F\otimes_{\Z}\Zhat$ is the ring of finite adeles of $F$. To get to
the full ring of adeles of a number field~$F$ you need to take the product with the
ring of infinite adeles of $F$, which is $F\otimes_{\Q}\R$: some kind of universal
archimedean completion of $F$.

Now let's attempt to profinitely complete $\Q$, by defining $\Qhat:=\Q\otimes_{\Z}\Zhat$.
We're going to start by trying to make sense of this ring. First we will understand
its additive structure. Then we'll understand the multiplicative structure of its units.

This is a bit more

\section{Warm-up: the additive structure of the completion of $\Qhat$}

We now consider $\Qhat:=\Q\otimes_{\Z}\Zhat$. This is some sort of a ``completion'' of $\Q$; it is also known as
the finite adele ring of $\Q$. It has two obvious
subrings
A reminder: a general element of $A\otimes B$ is not of the form $a\otimes b$, but
is more generally a finite sum of such elements. However we don't have any such
problems with $\Q\otimes_{\Z}\Zhat$, as the below lemma shows.

\begin{lemma}\label{Qhat.canonicalForm} Every element of $\Qhat:=\Q\otimes_{\Z}\Zhat$
can be written as $q\otimes z$ with $q\in\Q$ and $z\in\Zhat$.
\end{lemma}
\begin{proof} Take general element $\sum_i q_i\otimes z_i\in\Qhat$. Now choose a large 
    positive integer $N$, the lowest
common multiple of all the denominators showing up in the $q_i$, and then rewrite
the general element as $\sum_i \frac{n_i}{N}\otimes z_i$ with $n_i\in\Z$. Now recall that a fundemantal
equality in the tensor product $A\otimes_{\Z} B$ is that $na\otimes b=a\otimes nb$ for $n\in\Z$.
Applying this rule, we see that our general element is $\sum_i \frac{1}{N}\otimes n_i z_i$
and hence equal to $\frac{1}{N}\otimes(\sum_i n_i z_i)$ and thus of the form $q\otimes z$ with $q\in\Q$
and $z\in\Zhat$.
\end{proof}

Be careful though: just because every element of $\Qhat$ can be written as $q\otimes z$, this
reprsentation may not be unique. For example $2\otimes 1=1\otimes 2$.

The question we address in this section is how to understand $\Qhat$ as an additive
abelian group, and in particular in terms of its two obvious subgroups $\Q$ and $\Zhat$. More precisely we are talking about the subgroups \Q\otimes_{\Z}1=\Q\otimes_{\Z}\Z$,
isomorphic to $\Q$, and $1\otimes\Zhat=\Z\otimes_{\Z}\Zhat$, isomorphic to $\Zhat$. 

Let us first understand the intersection of these two additive subgroups. Let's start with their infimum.
Note that all tensors throughout this section are over $\Z$
and so we'll drop the subscript ${\Z}$ from the notation.

\begin{lemma}\label{Qhat.intersection} The intersection of $\Q$ and $\Zhat$ in $\Qhat$ is $\Z$.
\end{lemma}
All tensors 
One could rewrite this question as asking for the intersection of $\Q\otimes

I claim that these two
subgroups $\Q$ and $\Zhat$generate $\Q\otimes_{\Z}\Zhat$, and their intersection is just the copy $\Z\otimes_{\Z}\Z$ of $\Z$ in $\Zhat$.
To prove this we need to understand what the multiples of~$N$ are in $\Zhat$.

\begin{lemma}{ZHat.multiples} The multiples of~$N$ in $\Zhat$ are precisely the compatible collections $(z_i)_i$ with
$z_N=0$.
\end{lemma}
\begin{proof}
    Clearly $z_N=0$ is a necessary condition to be a multiple of~$N$. To see it is sufficient, take a general $(z_i)$
    which is a multiple of~$N$, note that $z_N=0$, and now define a new element $(y_j)_j$ of $\Zhat$
    by $y_j=z_{Nj}/N$. Just to clarify what this means: $z_{Nj}\in\Z/Nj\Z$ maps to $z_N=0$ in $\Z/N\Z$,
    so is in the subgroup $N\Z/Nj\Z$ of $\Z/Nj\Z$, which is isomorphic (via "division by $N$") to the
    ring $\Z/j\Z$; this is how we construct $y_j$. It is easily checked that the $y_j$ are compatible
    and that $Ny=z$.
\end{proof}


